\section{Elements of a Crime}

\begin{enumerate}
    \item Every crime has two elements: \emph{actus reus} and \emph{mens rea}.
    \item Every crime also has attendant circumstances.
\end{enumerate}

\subsection{Actus Reus}

\begin{enumerate}
    \item Literally, ``guilty act.'' There is no universally accepted definition. In murder, for instance, some would consider it to be the pulling of the trigger. Others would consider it to be the death itself. The most common definition would consider it to be both.
    \item ``Omissions are not accidents.''---Marianne Moore.
    \item What constitutes an act? When does the act begin? See Model Penal Code § 2.01.
    \item If someone holds a gun to your head and tells you to act, your act is voluntary. An act is something you do willfully.
    \item Thought crimes are not punishable (\emph{Minority Report}, \emph{Firestarter}).
    \item The \textbf{harm principle}: we punish acts that are socially and individually harmful.
\end{enumerate}

\subsubsection{Voluntary vs. Involuntary Action: \emph{Martin v. State}}

\begin{enumerate}
    \item Police officers took a drunk man from his home and onto a public highway, where they then arrested him for public drunkenness. The court held that public drunkenness cannot be established when the accused was involuntarily carried to a public place.
\end{enumerate}

\subsubsection{Proving Involuntary Action: \emph{State v. Utter}}

\begin{enumerate}
    \item Defendant was drunk and stabbed his son. He had no memory of the stabbing. He argued that his service in the army had caused him to develop a ``conditioned response'' which makes him react violently and involuntarily to people approaching unexpectedly from behind.
    \item The court reasoned that an ``act'' requires voluntary action---that is, ``act'' is synonymous with ``voluntary act.'' An involuntary or unconscious act cannot induce guilt---that is, it is not an ``act'' at all.
    \item The court found that the defendant's theory of conditioned response should have been presented to a jury \emph{if there was substantial evidence to support it.} However, because the jury could not possibly know or infer what had happened in the room at the time of the stabbing, the question should not be sent to the jury.
\end{enumerate}

\subsubsection{Legal Duty: \emph{People v. Beardsley}}

\begin{enumerate}
    \item While his wife was away, the defendant was drinking heavily at his house with another woman. The woman took several tablets of morphine and became unresponsive. The defendant put her in a basement room in his house (which another man was renting). The woman died that evening.
    \item The issue is whether the defendant had a legal duty to protect the woman. If he omitted to perform his duty, he would be criminally liable for manslaughter. The prosecution argued that the defendant was in the role of the woman's guardian. The court reasoned, however, that if the defendant had been drinking with a man and that man attempted suicide, the defendant would not have had a duty to protect him---so it should make no difference that he was with a woman.
    \item The lower courts convicted the defendant of manslaughter, but the Michigan Supreme Court here reversed.
    \item Criminal law is reluctant to create positive responsibilities, but there are a few common law relationships where such responsibilities exist:
    \begin{enumerate}
        \item Parent-child.
        \item Spouse-spouse.
        \item Master-servant.
    \end{enumerate}
\end{enumerate}

\subsubsection{Failure to Act: \emph{Barber v. Superior Court}}

\begin{enumerate}
    \item A patient suffered cardiac arrest after surgery. Doctors managed to save him, but he suffered significant brain damage. He remained in a vegetative state on life support with little chance of recovery. His family decided to remove him from life support, and he died a few days later. 
    \item The question is whether his doctors had a duty to keep him alive---since omitting to perform that duty would make them liable for murder. \textbf{``There is no criminal liability for failure to act unless there is a legal duty to act.''} 
    \item The court reasoned first that removing the man from life support constituted an omission, not a positive act.
    \item The court held that the decision of whether to continue treatment was left to the family. Therefore, the doctors did not unlawfully fail to perform a legal duty.
\end{enumerate}

\subsubsection{\emph{Lawrence v. Texas}}

\begin{enumerate}
    \item The issue was whether a Texas law criminalizing sodomy violates the Fourteenth Amendment's Due Process Clause and equal protection guarantee.
    \item Justice Kennedy:
    \begin{enumerate}
        \item The statute violates individuals' rights to privacy and liberty.
        \item The Supreme Court had previously ruled on a similar Georgia statute outlawing sodomy in \emph{Bowers v. Hardwick}: ``\emph{Bowers} was not correct when it was decided, and it is not correct today. It ought not to remain binding precedent.''
        \item The equal protection guarantee ensures that homosexuals are entitled to the same privacy protections as heterosexuals.
        \item The Constitution limits states' power to outlaw social harms.
    \end{enumerate}
\end{enumerate}

\subsection{\emph{Mens Rea}}

\begin{enumerate}
    \item ``Guilty mind.''
    \item \emph{United States v. Cordoba-Hincapie} gave a brief history of the evolution from ancient English strict liability to the modern requirement of a guilty state of mind.
    \item There are two usages of \emph{mens rea}:
    \begin{enumerate}
        \item \textbf{Culpability}: a morally culpable state of mind in general.
        \item \textbf{Elemental}: the mental state specified in the definition of the crime.
    \end{enumerate}
\end{enumerate}

\subsubsection{General Culpability: \emph{Regina v. Cunningham}}

\begin{enumerate}
    \item The defendant stole a coin-operated gas meter from the basement of his mother-in-law's house, causing noxious gas to escape and partially asphyxiate his neighbor.
    \item The issue was whether his action was malicious. A lower court convicted the defendant on the definition of malice as ``wickedness,'' i.e., a generally culpable state of mind. This court defined malice as (1) an \textbf{intention} to do the specific harm, or (2) \textbf{recklessness} (i.e., he foresaw that the harm might occur, but did it anyway). In this case, there was no malice directed at Mrs. Wade. The court overturned the conviction.
\end{enumerate}

\subsubsection{Transferred Intent: \emph{People v. Conley}}

\begin{enumerate}
    \item In a fight after a high school party, the defendant smashed a wine bottle into the victim's face, causing permanent disability. He intended to hit someone else (who ducked), but the court found that the defendant's words and demeanor nonetheless intended his action to cause permanent disability.
    \item The common law definition of intent includes both the actor's conscious goal and the results that are ``virtually certain to occur.''\footnote{Casebook p. 155.}---similar to substantial certainty in intentional torts.
    \item A person ``intends the natural and probable consequences of his actions.'' The Fourteenth Amendment prevents courts from presuming this, but juries can use common sense to recognize it.
    \item \textbf{Transferred intent}  allows transfer from one victim to another. Transfer between different types of harms is less clear cut. Courts often apply it, but not always.
    \item There is dispute about the meaning of ``general intent'' and ``specific intent.'' Some versions include:
    \begin{enumerate}
        \item General: the definition of the crime sets out no specific mental state, so the prosecutor needs only to prove a generally culpable state of mind. Specific: the definition of a crime explicitly sets out a mental state.
        \item General: reserved for crimes that permit conviction on the basis of a less culpable mental state (e.g., negligence or recklessness). Specific: denotes an offense that includes a definition of intent or knowledge.
        \item \textbf{The most common version}---general: any mental state related to the act that constitutes the social harm. Specific: an additional ``special mental element'': (1) intent to commit a future act (e.g., intent to sell), (2) special motive (e.g., offensive contact intended to cause humiliation), or (3) knowledge of attendant circumstances (e.g., sale of obscene material to a minor).
    \end{enumerate}
    \item At common law, there are ten specific intent crimes (BAFFLEPACK):
    \begin{enumerate}
        \item Burglary.
        \item Assault. 
        \item False pretenses.
        \item Forgery.
        \item Larceny.
        \item Embezzlement.
        \item Premeditated murder.
        \item Attempt.
        \item Conspiracy/solicitation.
        \item Kidnapping for ransom.
    \end{enumerate}
\end{enumerate}

\subsubsection{MPC § 2.02: General Requirements of Culpability}

\begin{enumerate}
    \item The MPC requires ``elemental'' culpability---i.e., the specific state of mind required in the definition of the crime, rather than a generally morally culpable state of mind.
    \item The MPC abandons the elemental-culpable distinction. Most jurisdictions have adopted it in whole or in part.
    \item There are four levels of culpability:\footnote{MPC § 2.02(2).}
    \begin{enumerate}
        \item \textbf{Purpose}: An actor intends to perform a specific action or to cause a specific result.
        \item \textbf{Knowledge}: An actor is aware of factual circumstances that establish criminal culpability, and if the element involves a result of his conduct, he is practically certain that the result will occur. 
        \item \textbf{Recklessness}: An actor creates and recognizes a substantial, unjustifiable risk and acts anyway. The jury should decide whether the risk is substantial and unjustifiable and whether \emph{disregard of the risk} deserves condemnation. 
        \item \textbf{Negligence}: An actor inadvertently creates a substantial, unjustifiable risk of which he should have been aware. The jury should decide whether the risk is substantial and unjustifiable and whether the defendant's \emph{failure to perceive the risk} deserves condemnation.
    \end{enumerate}
    \item If a law does not specify a culpable state of mind (i.e., no \emph{mens rea}, culpability is established if the person acted purposefully, knowingly, or recklessly. Negligence is excluded unless the law specifically prescribes it (although many jurisdictions do not exclude negligence).
\end{enumerate}

\subsubsection{Knowledge of Attendant Circumstances}

\begin{enumerate}
    \item Willful blindness: suspecting the truth but not investigating it.
\end{enumerate}

\subsubsection{Defining Knowledge: \emph{State v. Nations}}

\begin{enumerate}
    \item The defendant, Sandra Nations, operated a bar where a sixteen-year-old girl was dancing for money. A Missouri child welfare statute imposed criminal liability on anyone who knowingly aided such activity.
    \item The Model Penal Code in § 2.02(7) holds that ``knowledge'' of a particular element of a crime is established when the actor is aware of a ``high probability of its existence''---i.e., willful blindness towards a fact constitutes knowledge of that fact.
    \item The Missouri statute, however, did not adopt this definition of ``knowledge.'' The court thus found the defendant to be reckless, but not knowing, and held in favor of the defendant.
\end{enumerate}

\subsubsection{Strict Liability}

\begin{enumerate}
    \item Strict liability crimes assign guilt without requiring \emph{mens rea}.
\end{enumerate}

\paragraph{Public Welfare Offenses: \emph{United States v. Cordoba-Hincapie}}

\begin{enumerate}
    \item One category of strict liability crimes are ``public-welfare offenses''---e.g., liquor laws, anti-narcotics laws, motor vehicle regulations.
    \item Public-welfare laws are meant to regulate administrative offenses unrelated to questions of personal guilt.
    \item \emph{Mens rea} is probably required if the punishment of the wrongdoer far outweighs regulation of the social order.
    \item \emph{Mens rea} is probably not required if the punishment is light (e.g., small fine and no prison time).
    \item Even when a statute is silent on the \emph{mens rea} requirement, it can still sometimes be interpreted as requiring a minimal level of \emph{mens rea}. See \emph{Staples} below.
    \item With strict liability offenses, there is no basis for acquittal on the grounds of mistakes of fact or law. (It doesn't matter what you intended to do---for strict liability offenses, it only matters that you did it.)
\end{enumerate}

\paragraph{Inferring \emph{Mens Rea}\emph{Staples v. United States}}

\begin{enumerate}
    \item BATF agents found the defendant in possession of an unregistered semiautomatic AR-15 rifle that had been modified to shoot as an automatic weapon. Under the National Firearms Act, this gun was classified as a machine gun and was required to be registered. The defendant argued that he didn't know the gun had been modified, and therefore he should be shielded from criminal liability for failing to register it. The District Court and the Court of Appeals rejected his argument.
    \item Justice Thomas:
    \begin{enumerate}
        \item The relevant statute is silent concerning the \emph{mens rea} requirement.
        \item Common law holds that \emph{mens rea} should be required here, despite the statute's silence, unless it's clear that Congress intended to remove the \emph{mens rea} requirement.
        \item The prosecution argues that Congress intended this statute to address ``public welfare'' offense, and thus impose strict criminal liability.
        \item This interpretation of Congress's silence (says Thomas) has typically been applied to situations where the regulated offense poses a very clear threat to public safety (e.g., hand grenades in \emph{Freed} or narcotics in \emph{Balint}). In contrast, possession of items with no public safety threat has not been held to be a strict liability crime (e.g., food stamps in \emph{Liparota}.
        \item Gun ownership is generally an innocent activity (half of American households own a gun), unlike possessing a hand grenade or selling hard drugs.
        \item The severe penalty here (10 years) negates the public welfare rationale. Generally, \emph{mens rea} should be required as part of statutes defining felony offenses.
        \item If Congress wanted to impose severe criminal penalties on gun owners who unknowingly possessed certain offending weapons (like machine guns), it would have said so explicitly.
    \end{enumerate}
    \item Justice Stevens, dissenting:
    \begin{enumerate}
        \item This statute is not based on a common law crime. We cannot rely on common law to fill Congress's omissions. Rather, we should assume Congress's omissions are intentional.
        \item Machine guns are ``dangerous [and] deleterious devices.'' This is clearly a public welfare statute.
    \end{enumerate}
\end{enumerate}

\paragraph{Legislative Silence as Strict Liability: \emph{Garnett v. State}}

\begin{enumerate}
    \item The adult defendant had sex with someone he did not know was below the age of consent. The statutory rape language did not specify a \emph{mens rea} component. The trial court convicted him of statutory rape.
    \item The appellate court noted that in statutes that do not define a \emph{mens rea} component, the MPC generally recognizes strict liability only for offenses that do not give rise to any ``legal disability.''\footnote{Casebook p. 189.}
    \item In this case, however, the court pointed out that the legislature was explicit about \emph{mens rea} in the previous section, so its silence in the section at hand was likely deliberate. Therefore, it likely intended statutory rape to be a strict liability crime. The court upheld the conviction.
    \item The dissent argues that the legislative history and structure suggest that there is a \emph{mens rea} component.
\end{enumerate}

\subsubsection{Mistake and \emph{Mens Rea}}

\begin{enumerate}
    \item Good faith mistakes do not have to be reasonable to be valid defenses against crimes with \emph{mens rea} components.
    \item Why limit the application of the mistake doctrine?
    \begin{enumerate}
        \item Utilitarianism: we want people to know the law.
        \item Retributivism: hard to say; retributivists might actually want a broad application of the mistake doctrine, since unwitting offenders may not be morally culpable.
    \end{enumerate}
    \item \emph{Malum in se}: bad in itself---e.g., murder.
    \item \emph{Malum prohibitum}: bad because outlawed---e.g., driving without a license.
\end{enumerate}

\paragraph{Mistake of Fact: \emph{People v. Navarro}}

\begin{enumerate}
    \item The defendant stole four wooden beams from a construction site. He believed in good faith that the owner had abandoned the beams.
    \item The trial court instructed the jury that the defendant would not guilty of theft if he \emph{reasonably} believed in good faith that the beams had been abandoned or that he had permission to take them. The jury found the defendant guilty of theft.
    \item The appellate court reversed, reasoning that if the defendant believed in good faith that he was allowed to take the beams---regardless of whether that belief was reasonable---he lacked the intent necessary for theft. (A jury could infer that a defendant does not hold such a belief in good faith---but in this case, his belief was genuine.)
    \item \textbf{Honest mistake of fact is a defense when it negates a required mental element of the crime.}
    \begin{enumerate}
        \item For specific intent crimes, mistake of fact negates the \emph{mens rea} regardless of whether the mistake was reasonable.
        \item For general intent crimes, the mistake must have been reasonable to negate the \emph{mens rea}.
        \item (The MPC does not draw a distinction between specific and general intent crimes.)
    \end{enumerate}
\end{enumerate}

\paragraph{Mistake of Law: \emph{People v. Merrero}}

\begin{enumerate}
    \item A federal prison guard was charged with possessing an unlicensed loaded pistol at a club. He argued that he interpreted a state statute as exempting ``peace officers'' from the gun law---but in fact, the statute only exempted state penal corrections officers, not federal officers.
    \item The trial court rejected the defendant's argument that his misunderstanding of the law exempted him from criminal liability. He relied on a New York statute that relieves criminal liability if the defendant mistakenly relies on ``a statute or other enactment.'' The prosecution argued (and the court agreed) that misconstruing the meaning of a statute is not enough to establish a defense---drawing on MPC 2.04(3), it argued the statute must actually be ``determined to be invalid or erroneous.'' The appellate court reasoned that allowing defendants to simply interpret the law case-by-case would lead to chaos.
    \item The dissent argued that there is no retributivist or utilitarian justification for punishing the defendant in this case. It argued further that the defendant reasonably interpreted the statute exempting ``peace officers'' and that he had no way of knowing that the courts would later interpret the statute to exclude federal penal officers.
    \item According to the dissent, the majority opinion ruled out \emph{any} defense based on mistaken understandings of law. This is a misinterpretation of the reasons for the New York mistake-of-law statute---and the majority opinion's reliance on MPC 2.04(3) is puzzling since the New York legislature specifically rejected that part of the MPC. The dissent believed there should be room for ``good-faith mistaken belief founded on a well-grounded interpretation'' of official law.
    \item The MPC reflects the common law view that ignorance of the law is no excuse. But it includes a few exceptions---for instance, when the statute was enacted without fair notice, when a defendant relied on an official statement of law that turned out to be erroneous.\footnote{MPC § 2.04(3).}
    \item The three scenarios where mistake of law can serve as a valid defense are:
    \begin{enumerate}
        \item Reasonable reliance on an official source (\emph{People v. Merrero}).
        \item Lack of fair notice (\emph{Lambert v. California}\footnote{Casebook p. 207}).
        \item When it negates an element of the offense (\emph{Cheek v. United States}).
    \end{enumerate}
\end{enumerate}

\paragraph{Unreasonable Mistake of Law: \emph{Cheek v. United States}}

\begin{enumerate}
    \item The defendant stopped paying taxes in the early 1980s. He had been heavily involved in the anti-tax movement and genuinely believed that the income tax on wages is unconstitutional. Federal criminal tax offenses require specific intent to violate the law---they require \emph{willful} failure to file and pay taxes (otherwise, we'd all be criminals for making mistakes on our tax returns). Cheek argues that he did not \emph{willfully} fail to file or pay.
    \item The lower courts rejected Cheek's argument against the validity of jury instructions requiring an ``honest and reasonable'' belief that he was not required to pay income tax (as did the lower courts in \emph{Navarro}). In the court's opinion, Justice White identifies the precise issue as whether the defendant waw aware of his duty to pay taxes, ``which cannot be true if the jury credits a good-faith misunderstanding and belief submission.'' It does not matter if his believe was unreasonable (though, as in \emph{Navarro} the jury may infer that the belief was not in good faith). The Supreme Court reversed the Court of Appeals and held that Cheek could make his case to a jury.
    \item Justice Blackmun, dissenting, warns that this decision ``will encourage taxpayers to cling to frivolous views of the law.''
\end{enumerate}

\subsection{Causation}

\begin{enumerate}
    \item MPC § 2.03(1) finds causation when an act is (1) a but-for cause of the result and (2) the causal relationship satisfies any additional statutory requirements.% TODO: example?
    \item Why does causation matter in terms of justifications for punishment?
\end{enumerate}

\subsubsection{Actual Cause}

\begin{enumerate}
    \item ``Cause-in-fact'': a person's actions caused the outcome in question.
    \item \textbf{``But-for'' test}: a defendant's conduct is a cause-in-fact of the outcome in question if the outcome would not have occurred \emph{but for} the defendant's actions.
    \begin{enumerate}
        \item Establishing a but-for cause does not necessarily mean the defendant will be convicted. It's only a filter for identifying actors who \emph{may} be culpable.
    \end{enumerate}
    \item \textbf{``Substantial factor'' test}: if two independent defendants commit two separate acts, each of which could have caused the prohibited result, neither act is a ``but for'' cause. This test determines whether the action was nonetheless a ``substantial factor'' in bringing about the prohibited result. This test is used only in a minority of jurisdictions.
\end{enumerate}

\paragraph{Acceleration: \emph{Oxendine v. State}}

\begin{enumerate}
    \item The defendant's girlfriend pushed his six-year-old son into a bathtub, causing severe internal injury. Around 24 hours later, the defendant also hit his son repeatedly. His son died from his injuries soon after. Medical testimony was unable to isolate the mortal blow.
    \item The trial court found both defendants guilty of manslaughter. The Supreme Court of Delaware, however, found that the prosecution proved that the defendant had not \emph{accelerated} his son's death, but only aggravated it. The court found the defendant innocent of manslaughter but guilty of assault in the second degree.
\end{enumerate}

\subsubsection{Proximate Cause} 

\begin{enumerate}
    \item The doctrine of proximate cause determines whether an event that satisfies the but-for standard should be held accountable for the resulting harm.
    \item Proximate cause answers the question of who is most culpable for the harm.
\end{enumerate}

\paragraph{Chain of Causation: \emph{People v. Rideout}}

\begin{enumerate}
    \item The defendant was driving drunk and hit a car. The car's driver and passenger suffered no major injuries, but the car was damaged enough so that the headlights no longer worked. After moving safely to the side of the road, the car's passenger entered the road to inspect the car to see if they could turn on the hazard lights, where he was struck and killed by an oncoming car. The trial court found the defendant guilty for the passenger's death. The question is whether the defendant's drunk driving was the proximate cause of the passenger's death.
    \item The appellate court introduces the ideas of intervening and superseding causes. An intervening cause supersedes the original cause if the original actor could not reasonably foresee the second cause, i.e., if it breaks the chain of causation. Dressler divides the second cause into \emph{responsive intervening causes}, which arise directly from the original cause, and \emph{coincidental intervening causes}.
    \item The appellate court also introduces the \emph{apparent-safety doctrine}, which holds that the defendant's causation ceases when the victim has reached a place of apparent safety (e.g., far off on the side of the road).
    \item The appellate court also introduces the idea of \emph{voluntary human intervention}, which relieves the defendant's liability if the victim voluntarily enters into a dangerous situation (e.g., a road in the night without any lights).
    \item The appellate court overrules the trial court, finding that the prosecution failed to establish proximate cause. It remanded the case for a new trial.
    \item Later, the Michigan Supreme Court overturned the appellate court's assessment that a jury could not find proximate cause.
    \item When does an intervening cause break the causal chain? Murray:
    \begin{enumerate}
        \item \emph{De minimis} contribution to social harm---where the defendant's action was an insubstantial contribution to the harmful result, in comparison to the intervening event, the defendant is relieved of liability.
        \item Intended consequences doctrine---a voluntary act intended to bring about the harmful result will be considered a proximate cause of the harm, regardless of other intervening events.
        \item Omissions---an omission will rarely, if ever, supersede defendant's earlier, operative wrongful act.
        \item Foreseeability of the Intervening Cause:
        \begin{enumerate}
            \item Responsive (Dependent) Intervening Causes---defendant bears criminal responsibility for the harmful result to a victim who seeks to extricate himself or another from a dangerous situation created by defendant, even where the victim was contributorily negligent.
            \item Coincidental (Independent) Intervening Causes---an act that does not occur in response defendant's conduct may break the causal chain.
        \end{enumerate}
        \item Apparent-safety doctrine---once the victim has reached a place of apparent safety, defendant's prior wrongful act is no longer causally operative.
        \item Voluntary human intervention---victim's deliberate, informed intervention may break the causal chain.
    \end{enumerate}
\end{enumerate}

\paragraph{Superseding Intervening Cause: \emph{Velazquez v. State}}

\begin{enumerate}
    \item The defendant and the victim were drag racing. After the race, the victim spun around his car, raced back to the starting line, and careened over a guardrail, dying instantly.
    \item The trial court found that the defendant's participation in the drag race was a cause-in-fact of the victim's death. The appellate court found that the drag race had already ended when the victim decided to spin around and race back to the finish line---an act that superseded the defendant's cause-in-fact.
\end{enumerate}
