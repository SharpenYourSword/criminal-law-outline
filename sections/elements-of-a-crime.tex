\section{Elements of a Crime}

\begin{enumerate}
    \item Every crime has two elements: \textbf{actus reus} and \textbf{mens 
    rea}.
    \item Every crime also has attendant circumstances.
\end{enumerate}

\subsection{Actus Reus}

\begin{enumerate}
    \item Literally, ``guilty act.'' There is no universally accepted 
    definition. In murder, for instance, some would consider it to be the 
    pulling of the trigger. Others would consider it to be the death itself.  
    The most common definition would consider it to be both.
    \item Under both common law and the MPC, voluntary acts are the products 
    of free will. We do not not punish acts that are not the product of the 
    individual's free will.
    \item ``Omissions are not accidents.''---Marianne Moore.
    \item What constitutes an act? When does the act begin? See Model Penal 
    Code \S\ 2.01.
    \item If someone holds a gun to your head and tells you to act, your act 
    is voluntary. \textbf{An act is something you do willfully.}
    \item Thought crimes are not punishable (\emph{Minority Report}, 
    \emph{Firestarter}).
    \item The \textbf{harm principle}: we punish acts that are socially and 
    individually harmful. However, there are constitutional limits on what the 
    state can criminalize. The state may not criminalize behavior it believes 
    to be immoral or distasteful unless it causes actual harm. See 
    \emph{Lawrence} below.
    \item \textbf{Laws can only punish \emph{conduct}, not status.} For 
    instance, a city cannot make it a crime to be addicted to a substance.
    \item \textbf{Omissions and duty}: Criminal liability for omissions exists 
    only where there is a legal duty to act. Criminal law is reluctant to 
    create positive responsibilities, but there are a few common law 
    relationships where such responsibilities exist: e.g., parent-child, 
    spouse-spouse, master-servant.  Other circumstances giving rise to legal 
    duty include contracts, statutes, voluntarily assuming care and prevent 
    others from assisting, and creating harm to another. See \emph{Beardsley} 
    below.
    \item Under the MPC, omissions do not create criminal liability unless (1) 
    the law expressly makes the omission sufficient or (2) the duty is 
    otherwise imposed by law.\footnote{MPC \S\ 2.01(3).}
\end{enumerate}

\subsubsection{Voluntary vs. Involuntary Action: \emph{Martin v. State}}

We do not punish involuntary actions.

\begin{enumerate}
    \item Police officers took a drunk man from his home and onto a public 
    highway, where they then arrested him for public drunkenness. The court 
    held that public drunkenness cannot be established when the accused was 
    involuntarily carried to a public place.
\end{enumerate}

\subsubsection{Proving Involuntary Action: \emph{State v. Utter}}

``Act'' is synonymous with ``voluntary act.'' An involuntary or unconscious 
act cannot establish guilt.

\begin{enumerate}
    \item Defendant was drunk and stabbed his son. He had no memory of the 
    stabbing. He argued that his service in the army had caused him to develop 
    a ``conditioned response'' which makes him react violently and 
    involuntarily to people approaching unexpectedly from behind.
    \item The court reasoned that an ``act'' requires voluntary action---that 
    is, ``act'' is synonymous with ``voluntary act.'' An involuntary or 
    unconscious act cannot induce guilt---that is, it is not an ``act'' at 
    all.
    \item The court found that the defendant's theory of conditioned response 
    should have been presented to a jury \emph{if there was substantial 
    evidence to support it.} However, because the jury could not possibly know 
    or infer what had happened in the room at the time of the stabbing, the 
    question should not be sent to the jury.
\end{enumerate}

\subsubsection{Legal Duty to Act: \emph{People v. Beardsley}}

Criminal law is reluctant to create positive responsibilities, but there are a 
few common law relationships where such responsibilities exist: e.g., 
parent-child, spouse-spouse, master-servant. (See the beginning of this 
section for other circumstances giving rise to legal duty.)

\begin{enumerate}
    \item While his wife was away, the defendant was drinking heavily at his 
    house with another woman. The woman took several tablets of morphine and 
    became unresponsive. The defendant put her in a basement room in his house 
    (which another man was renting). The woman died that evening.
    \item The issue was whether the defendant had a legal duty to protect the 
    woman. If he had omitted to perform his duty, he would have been 
    criminally liable for manslaughter.
    \item The prosecution argued that the defendant was in the 
    role of the woman's guardian.
    \item The court reasoned that if the defendant had been drinking with a 
    man and that man attempted suicide, the defendant would not have had a 
    duty to protect him---so it should make no difference that he was with a 
    woman.
    \item The lower courts convicted the defendant of manslaughter, but the 
    Michigan Supreme Court here reversed.
\end{enumerate}

\subsubsection{Failure to Act: \emph{Barber v. Superior Court}}

Omissions do not establish culpability absent a legal duty to act.

\begin{enumerate}
    \item A patient suffered cardiac arrest after surgery. Doctors managed to 
    save him, but he suffered significant brain damage. He remained in a 
    vegetative state on life support with little chance of recovery. His 
    family decided to remove him from life support, and he died a few days 
    later
    \item The question before the court was whether his doctors had a duty to 
    keep him alive---since omitting to perform that duty would make them 
    liable for murder. \textbf{``There is no criminal liability for failure to 
    act unless there is a legal duty to act.''}
    \item The court reasoned that removing the man from life support 
    constituted an omission, not a positive act. The decision of whether to 
    continue treatment was left to the family. Therefore, the doctors did not 
    unlawfully fail to perform a legal duty.
\end{enumerate}

\subsubsection{Constitutional Limitations on Criminalizing Conduct: 
\emph{Lawrence v. Texas}}

There are constitutional limits on the behavior states can criminalize.

\begin{enumerate}
    \item The issue was whether a Texas law criminalizing sodomy violated the 
    Fourteenth Amendment's Due Process Clause and equal protection guarantee.
    \item Justice Kennedy:
    \begin{enumerate}
        \item The statute violated individuals' rights to privacy and liberty.
        \item The Supreme Court had previously ruled on a similar Georgia 
        statute outlawing sodomy in \emph{Bowers v. Hardwick}: ``\emph{Bowers} 
        was not correct when it was decided, and it is not correct today. It 
        ought not to remain binding precedent.''
        \item The equal protection guarantee ensures that homosexuals are 
        entitled to the same privacy protections as heterosexuals.
        \item The Constitution limits states' power to outlaw social harms.
    \end{enumerate}
\end{enumerate}

\subsection{Mens Rea}

\subsubsection{General Principles}

\begin{enumerate}
    \item ``Guilty mind.''
    \item \emph{Actus non facit reum nisi mens sit rea}: the act does not make 
    a person guilty unless the mind be also guilty.
    \item \emph{United States v. Cordoba-Hincapie}: a brief history of the 
    evolution from ancient English strict liability to the modern requirement 
    of a guilty state of mind.\footnote{Casebook p. 149.}
    \item There are two usages of mens rea:
    \begin{enumerate}
        \item \textbf{Culpability}: a morally culpable state of mind in 
        general.
        \item \textbf{Elemental}: the mental state specified in the definition 
        of the crime.
    \end{enumerate}
    \item The general ``culpability'' variant has given way to the 
    ``elemental'' variant. See \emph{Cunningham} below.
    \item The MPC follows the elemental variant, i.e., the actor must have the 
    specific state of mind required in the definition of the crime. See 
    notes on \S\ 2.02 below.
    \item Under \textbf{transferred intent}, we attribute liability to a 
    defendant who, intending to act against one person, accidentally acts 
    against another person instead. See \emph{Conley} below.
    \item \textbf{Specific and general intent}: see below.
\end{enumerate}

\paragraph{General Culpability: \emph{Regina v. Cunningham}}

Maliciousness requires intent or recklessness, not a generally culpable state 
of mind.

\begin{enumerate}
    \item The defendant stole a coin-operated gas meter from the basement of 
    his mother-in-law's house, causing noxious gas to escape and partially 
    asphyxiate his neighbor.
    \item The issue was whether his action was malicious. A lower court 
    convicted the defendant on the definition of malice as ``wickedness,'' 
    i.e., a generally culpable state of mind.
    \item The appellate court defined malice as (1) an \textbf{intention} to 
    do the specific harm, or (2) \textbf{recklessness} (i.e., he foresaw that 
    the harm might occur, but did it anyway). In this case, there was no 
    malice directed at Mrs. Wade. The court overturned the conviction.
\end{enumerate}

\paragraph{Transferred Intent: \emph{People v. Conley}}

\begin{enumerate}
    \item In a fight after a high school party, the defendant smashed a wine 
    bottle into the victim's face, causing permanent disability. He intended 
    to hit someone else (who ducked), but the court found that the defendant's 
    words and demeanor nonetheless intended his action to cause permanent 
    disability.
    \item The common law definition of intent includes both the actor's 
    conscious goal and the results that are ``virtually certain to 
    occur''\footnote{Casebook p. 155.}---similar to substantial certainty in 
    intentional torts.
    \item A person ``intends the natural and probable consequences of his 
    actions.'' The Fourteenth Amendment prevents courts from presuming this, 
    but juries can use common sense to recognize it.
    \item \textbf{Transferred intent}  allows transfer from one victim to 
    another. Transfer between different types of harms is less clear cut.  
    Courts often apply it, but not always.
\end{enumerate}

\paragraph{Specific and General Intent}

\begin{enumerate}
    \item There is dispute about the meaning of ``general intent'' and 
    ``specific intent.'' The most common version (and Murray's preferred 
    version):\footnote{Other versions of the specific/general intent 
    distinction:
    \begin{enumerate}
        \item General: the definition of the crime sets out no specific mental 
        state, so the prosecutor needs only to prove a generally culpable 
        state of mind. Specific: the definition of a crime explicitly sets out 
        a mental state.
        \item General: reserved for crimes that permit conviction on the basis 
        of a less culpable mental state (e.g., negligence or recklessness).  
        Specific: denotes an offense that includes a definition of intent or 
        knowledge.
    \end{enumerate}}
    \begin{enumerate}
        \item \textbf{General}: the actor only desired to commit the criminal 
        offense conduct--e.g., rape (unlawful carnal knowledge of a woman), 
        battery (harmful or offensive contact).
        \item \textbf{Specific}: the actor acted with an additional ``special 
        mental element''--e.g., murder (intentional killing \emph{with malice 
        aforethought}), larceny (taking away the property of another 
        \emph{with the intent to deprive him of it permanently}). 
        Subcategories:
        \begin{enumerate}
            \item Intent to commit a future act---e.g., possession \emph{with 
            intent to distribute}.
            \item Special motive---e.g., offensive contact \emph{intended to 
            cause humiliation} or intentional killing \emph{with malice 
            aforethought}.
            \item Knowledge of attendant circumstances---e.g., sale of obscene 
            material \emph{to a minor}.
        \end{enumerate}
    \end{enumerate}
\end{enumerate}

\paragraph{Common Law Specific Intent Crimes: BAFFLEPACK}
~\\\\
\begin{enumerate}
    \item At common law, there are ten specific intent crimes (BAFFLEPACK):
    \begin{enumerate}
        \item Burglary.
        \item Assault.
        \item False pretenses.
        \item Forgery.
        \item Larceny.
        \item Embezzlement.
        \item Premeditated murder.
        \item Attempt.
        \item Conspiracy/solicitation.
        \item Kidnapping for ransom.
    \end{enumerate}
\end{enumerate}

\paragraph{MPC \S\ 2.02: General Requirements of Culpability}
~\\\\
\begin{enumerate}
    \item \textbf{The MPC requires ``elemental'' culpability}---i.e., the 
    specific state of mind required in the definition of the crime, rather 
    than a generally morally culpable state of mind.
    \item The MPC abandons the elemental-culpable distinction. Most 
    jurisdictions have adopted the MPC's approach in whole or in part.
    \item There are four levels of culpability in the MPC:\footnote{MPC \S\ 
    2.02(2).}
    \begin{enumerate}
        \item \textbf{Purpose}: An actor intends to perform a specific action 
        or to cause a specific result.
        \item \textbf{Knowledge}: An actor is aware of factual circumstances 
        that establish criminal culpability, and if the element involves a 
        result of his conduct, he is practically certain that the result will 
        occur.
        \item \textbf{Recklessness}: An actor creates and recognizes a 
        substantial, unjustifiable risk and acts anyway. The jury should 
        decide whether the risk is substantial and unjustifiable and whether 
        \emph{disregard of the risk} deserves condemnation.
        \item \textbf{Negligence}: An actor inadvertently creates a 
        substantial, unjustifiable risk of which he should have been aware. 
        The jury should decide whether the risk is substantial and 
        unjustifiable and whether the defendant's \emph{failure to perceive 
        the risk} deserves condemnation.
    \end{enumerate}
    \item If a law does not specify a culpable state of mind (i.e., no mens 
    rea), culpability is established if the person acted purposefully, 
    knowingly, or recklessly. \textbf{Negligence is excluded unless the law 
    specifically prescribes it} (although many jurisdictions do not exclude 
    negligence). This tracks the common law approach. According to the ALI 
    commentary, ``since negligence is an exceptional basis of liability, it 
    should be excluded as a basis unless explicitly prescribed.''\footnote{MPC 
    \S\ 2.02(3). See also ALI commentary, casebook pp. 159--63, and for 
    commentary on \S\ 2.02(3) specifically, p. 162 n. 5.}
\end{enumerate}

\paragraph{Knowledge of Attendant Circumstances and Willful Blindness}
~\\\\
\begin{enumerate}
    \item \textbf{Willful blindness} means suspecting the truth but not 
    investigating it.
    \item MPC \S\ 2.02(7): where ``knowledge of the existence of a particular 
    fact is an element of the offense, such knowledge is established if a 
    person is aware of a high probability of its existence.''
\end{enumerate}

\paragraph{Defining Knowledge: \emph{State v. Nations}}
~\\\\
Mens rea requirements depend on statutory language.

\begin{enumerate}
    \item The defendant, Sandra Nations, operated a bar where a 
    sixteen-year-old girl was dancing for money.
    \item A Missouri child welfare statute imposed criminal liability on 
    anyone who knowingly aided such activity.
    \item The Model Penal Code in \S\ 2.02(7) holds that ``knowledge'' of a 
    particular element of a crime is established when the actor is aware of a 
    ``high probability of its existence''---i.e., willful blindness towards a 
    fact constitutes knowledge of that fact.
    \item The Missouri statute, however, did not adopt this definition of 
    ``knowledge.'' The court thus found the defendant to be reckless, but not 
    knowing, and held in favor of the defendant.
\end{enumerate}

\subsubsection{Strict Liability}

\begin{enumerate}
    \item Strict liability crimes assign guilt without requiring \emph{mens 
    rea}.
\end{enumerate}

\paragraph{Public Welfare Offenses: \emph{United States v. Cordoba-Hincapie}}
~\\\\
\begin{enumerate}
    \item One category of strict liability crimes are ``public-welfare 
    offenses''---e.g., liquor laws, anti-narcotics laws, motor vehicle 
    regulations.
    \item Public-welfare laws are meant to regulate administrative offenses 
    unrelated to questions of personal guilt.
    \item mens rea is probably required if the punishment of the wrongdoer far 
    outweighs regulation of the social order.
    \item mens rea is probably not required if the punishment is light (e.g., 
    small fine and no prison time).
    \item Even when a statute is silent on the mens rea requirement, it can 
    still sometimes be interpreted as requiring a minimal level of mens rea. 
    See \emph{Staples} below.
    \item With strict liability offenses, there is no basis for acquittal on 
    the grounds of mistakes of fact or law. \textbf{It doesn't matter what you 
    intended to do---it only matters that you did it.}
\end{enumerate}

\paragraph{Inferring Mens Rea: \emph{Staples v. United States}}
~\\\\
If a statute is silent on the mens rea, courts can look to the common law for 
guidance. Strict liability generally should only be imposed for public welfare 
offenses. The severity of the punishment can also impact the analysis.

\begin{enumerate}
    \item BATF agents found the defendant in possession of an unregistered 
    semiautomatic AR-15 rifle that had been modified to shoot as an automatic 
    weapon. Under the National Firearms Act, this gun was classified as a 
    machine gun and was required to be registered.
    \item The defendant argued that he didn't know the gun had been modified, 
    and therefore he should be shielded from criminal liability for failing to 
    register it. The District Court and the Court of Appeals rejected his 
    argument.
    \item Justice Thomas:
    \begin{enumerate}
        \item The relevant statute is silent concerning the mens rea 
        requirement.
        \item Common law holds that a mens rea should be required here, 
        despite the statute's silence, unless it's clear that Congress 
        intended to remove the mens rea requirement.
        \item The prosecution argued that Congress intended this statute to 
        address ``public welfare'' offense, and thus impose strict criminal 
        liability.
        \item This interpretation of Congress's silence (says Thomas) has 
        typically been applied to situations where the regulated offense poses 
        a very clear threat to public safety (e.g., hand grenades in 
        \emph{Freed} or narcotics in \emph{Balint}). In contrast, possession 
        of items with no public safety threat has not been held to be a strict 
        liability crime (e.g., food stamps in \emph{Liparota}).
        \item Gun ownership is generally an innocent activity (half of 
        American households own a gun), unlike possessing a hand grenade or 
        selling hard drugs.
        \item The severe penalty here (10 years) negates the public welfare 
        rationale. Generally, mens rea should be required as part of statutes 
        defining felony offenses.
        \item If Congress wanted to impose severe criminal penalties on gun 
        owners who unknowingly possessed certain offending weapons (like 
        machine guns), it would have said so explicitly.
    \end{enumerate}
    \item Justice Stevens, dissenting:
    \begin{enumerate}
        \item This statute is not based on a common law crime. We cannot rely 
        on common law to fill Congress's omissions. Rather, we should assume 
        Congress's omissions are intentional.
        \item Machine guns are ``dangerous [and] deleterious devices.'' This 
        is clearly a public welfare statute.
    \end{enumerate}
\end{enumerate}

\paragraph{Legislative Silence as Strict Liability: \emph{Garnett v. State}}
~\\\\
Legislative context can indicate that the omission of a mens rea is 
deliberate, i.e., that the legislature intended to define a strict liability 
crime.

\begin{enumerate}
    \item The adult defendant had sex with someone he did not know was below 
    the age of consent. The statutory rape language did not specify a mens rea 
    component. The trial court convicted him of statutory rape.
    \item The appellate court noted that in statutes that do not define a mens 
    rea component, the \textbf{MPC generally recognizes strict liability only for 
    offenses that do not give rise to any ``legal 
    disability.''}\footnote{Casebook p. 189.}
    \item In this case, however, the court pointed out that the legislature 
    was explicit about mens rea in the previous section, so its silence in the 
    section at hand was likely deliberate. Therefore, it likely intended 
    statutory rape to be a strict liability crime. The court upheld the 
    conviction.
    \item The dissent argued that the legislative history and structure 
    suggest that there is a mens rea component.
\end{enumerate}

\subsubsection{Mistake and Mens Rea}

\begin{enumerate}
    \item \textbf{Good faith mistakes do not have to be reasonable to be valid 
    defenses against crimes with mens rea components.}
    \item Why limit the application of the mistake doctrine?
    \begin{enumerate}
        \item Utilitarianism: we want people to know the law.
        \item Retributivism: hard to say; retributivists might actually want a 
        broad application of the mistake doctrine, since unwitting offenders 
        may not be morally culpable.
    \end{enumerate}
    \item \textbf{\emph{Malum in se}}: bad in itself---e.g., murder.
    \item \textbf{\emph{Malum prohibitum}}: bad because outlawed---e.g., 
    driving without a license.
\end{enumerate}

\paragraph{Common Law vs. MPC Mistake of Law and Fact}
~\\\\
\begin{tabular}{ | p{6cm} | p{6cm} |}
\hline
    \textbf{Common law mistake of fact:}
    \begin{enumerate}
        \item Specific intent crimes (i.e., BAFFLEPACK): mistake of fact, 
        whether reasonable or not, will negate the mens rea for a crime.
        \item General intent crimes (e.g., rape): mistake of fact 
        will\emph{not} negate the mens rea unless the mistake was reasonable.
    \end{enumerate} &
    \textbf{MPC mistake of fact:}
    \begin{enumerate}
        \item Mistake of fact is a defense when it ``negatives'' a ``material 
        element of the offense.''\footnote{MPC \S\ 2.04.}
    \end{enumerate} \\ \hline
    \textbf{Common law mistake of law}: ignorance of the law is no excuse, with a 
    few exceptions (which are the same as the MPC mistake of law elements 
    below). &
    \textbf{MPC mistake of law} defense is available when:
    \begin{enumerate}
        \item There is a lack of fair notice.
        \item The actor reasonably relied on an official statement of the law.
        \item The ignorance negatives a mental element of the offense (e.g.,.  
        in \emph{Cheek} below, if the defendant was unaware of the duty to 
        file and pay taxes, he could not be found guilty of \emph{willful} tax 
        evasion.)
    \end{enumerate} \\
\hline
\end{tabular}

\paragraph{Mistake of Fact: \emph{People v. Navarro}}

For specific intent crimes, it doesn't matter whether a mistake of fact is 
reasonable.

\begin{enumerate}
    \item The defendant stole four wooden beams from a construction site. He 
    believed in good faith that the owner had abandoned the beams.
    \item The trial court instructed the jury that the defendant would not 
    guilty of theft if he \emph{reasonably} believed in good faith that the 
    beams had been abandoned or that he had permission to take them. The jury 
    found the defendant guilty of theft.
    \item The appellate court reversed, reasoning that if the defendant 
    believed in good faith that he was allowed to take the beams---regardless 
    of whether that belief was reasonable---he lacked the intent necessary for 
    theft. (A jury could infer that a defendant does not hold such a belief in 
    good faith---but in this case, his belief was genuine.)
\end{enumerate}

\paragraph{Mistake of Law: \emph{People v. Merrero}}

Misreading a statute does not entitle a defendant to a mistake of law defense. 
However, reasonable reliance on an official source can make the defense 
available.

\begin{enumerate}
    \item A federal prison guard was charged with possessing an unlicensed 
    loaded pistol at a club. He argued that he interpreted a state statute as 
    exempting ``peace officers'' from the gun law---but in fact, the statute 
    only exempted state penal corrections officers, not federal officers.
    \item The trial court rejected the defendant's argument that his 
    misunderstanding of the law exempted him from criminal liability. He 
    relied on a New York statute that relieves criminal liability if the 
    defendant mistakenly relies on ``a statute or other enactment.'' The 
    prosecution argued (and the court agreed) that misconstruing the meaning 
    of a statute is not enough to establish a defense---drawing on MPC \S\
    2.04(3), it argued the statute must actually be ``determined to be invalid 
    or erroneous.'' The appellate court reasoned that allowing defendants to 
    simply interpret the law case-by-case would lead to chaos.
    \item The dissent argued that there is no retributivist or utilitarian 
    justification for punishing the defendant in this case. It argued further 
    that the defendant reasonably interpreted the statute exempting ``peace 
    officers'' and that he had no way of knowing that the courts would later 
    interpret the statute to exclude federal penal officers.
    \item According to the dissent, the majority opinion ruled out \emph{any} 
    defense based on mistaken understandings of law. This is a 
    misinterpretation of the reasons for the New York mistake-of-law 
    statute---and the majority opinion's reliance on MPC 2.04(3) is puzzling 
    since the New York legislature specifically rejected that part of the MPC.  
    The dissent believed there should be room for ``good-faith mistaken belief 
    founded on a well-grounded interpretation'' of official law.
\end{enumerate}

\paragraph{Unreasonable Mistake of Law: \emph{Cheek v. United States}}

For specific intent crimes, the defendant cannot be found guilty if he lacked 
the requisite intent because of a mistake of fact, even if that mistake was 
unreasonable.

\begin{enumerate}
    \item The defendant stopped paying taxes in the early 1980s. He had been 
    heavily involved in the anti-tax movement and genuinely believed that the 
    income tax on wages is unconstitutional. Federal criminal tax offenses 
    require specific intent to violate the law---they require \emph{willful} 
    failure to file and pay taxes (otherwise, we'd all be criminals for making 
    mistakes on our tax returns). Cheek argues that he did not 
    \emph{willfully} fail to file or pay.
    \item The lower courts rejected Cheek's argument against the validity of 
    jury instructions requiring an ``honest and reasonable'' belief that he 
    was not required to pay income tax (as did the lower courts in 
    \emph{Navarro}).
    \item Justice White: the precise issue as whether the defendant was aware 
    of his duty to pay taxes, ``which cannot be true if the jury credits a 
    good-faith misunderstanding and belief submission.''\footnote{Casebook p. 
    212.} It does not matter if his belief was unreasonable (though, as in 
    \emph{Navarro}, the jury may infer that the belief was not in good faith). 
    The Supreme Court reversed the Court of Appeals and held that Cheek could 
    make his case to a jury.
    \item Justice Blackmun, dissenting: this decision ``will encourage 
    taxpayers to cling to frivolous views of the law.''
\end{enumerate}

\subsection{Causation}

\begin{enumerate}
    \item Causation is implicit in the concept of actus reus. It's the link 
    between the prohibited act and the harmful result.
    \item Causation issues typically arise in the context of homicide.
    \item ``Cause-in-fact'': a person's actions caused the outcome in 
    question.
    \item \textbf{``But-for'' test}: a defendant's conduct is a cause-in-fact 
    of the outcome in question if the outcome would not have occurred 
    \emph{but for} the defendant's actions.
    \item \textbf{``Substantial factor'' test}: if two independent defendants 
    commit two separate acts, each of which could have caused the prohibited 
    result, neither act is a ``but for'' cause. This test determines whether 
    the action was nonetheless a ``substantial factor'' in bringing about the 
    prohibited result. This test is used only in a minority of jurisdictions.
    \begin{enumerate}
        \item The MPC does \emph{not} use the substantial factor test.
    \end{enumerate}
    \item Certain intervening acts can break the causal chain. Murray: 
    \begin{enumerate}
        \item \emph{De minimis} contribution to social harm---where the 
        defendant's action was an insubstantial contribution to the harmful 
        result, in comparison to the intervening event, the defendant is 
        relieved of liability.
        \item Intended consequences doctrine---a voluntary act intended to 
        bring about the harmful result will be considered a proximate cause of 
        the harm, regardless of other intervening events.
        \item Omissions---an omission will rarely, if ever, supersede 
        defendant's earlier, operative wrongful act.
        \item Foreseeability of the Intervening Cause:
        \begin{enumerate}
            \item Responsive (Dependent) Intervening Causes---defendant bears 
            criminal responsibility for the harmful result to a victim who 
            seeks to extricate himself or another from a dangerous situation 
            created by defendant, even where the victim was contributorily 
            negligent.
            \item Coincidental (Independent) Intervening Causes---an act that 
            does not occur in response defendant's conduct may break the 
            causal chain.
        \end{enumerate}
        \item Apparent-safety doctrine---once the victim has reached a place 
        of apparent safety, defendant's prior wrongful act is no longer 
        causally operative.
        \item Voluntary human intervention---victim's deliberate, informed 
        intervention may break the causal chain.
    \end{enumerate}
    \item Why does causation matter in terms of justifications for punishment?
\end{enumerate}

\subsubsection{MPC vs. Common Law Causation}

\begin{enumerate}
    \item The MPC and common law do not differ significantly in their approach 
    to causation. They use different terminology, though. The MPC determines 
    the but-for cause and then evaluates the defendant's culpability. Common 
    law determines the but-for cause and then determines whether that cause is 
    the proximate cause. Both consider intervening causes.
    \item MPC causation:
    \begin{enumerate}
        \item MPC \S\ 2.03(1) finds causation when an act is (1) a but-for 
        cause of the result and (2) the causal relationship satisfies any 
        additional statutory requirements.
        \item If there are multiple acts that result in the prohibited harm, 
        the MPC subjects each to the but-for test.
    \end{enumerate}
    \item Common law causation is a two-step process:
    \begin{enumerate}
        \item Determine whether an act is the but-for cause of a harm or (2) 
        if it accelerated the harm (\emph{Oxendine}).
        \item Determine which of the causes is the proximate cause.
    \end{enumerate}
\end{enumerate}

\subsubsection{Actual Cause}

\paragraph{Acceleration: \emph{Oxendine v. State}}

There is a distinction between accelerating and merely aggravating an injury.

\begin{enumerate}
    \item The defendant's girlfriend pushed his six-year-old son into a 
    bathtub, causing severe internal injury. Around 24 hours later, the 
    defendant also hit his son repeatedly. His son died from his injuries soon 
    after. Medical testimony was unable to isolate the mortal blow.
    \item The trial court found both defendants guilty of manslaughter. The 
    Supreme Court of Delaware, however, found that the prosecution proved that 
    the defendant had not \emph{accelerated} his son's death, but only 
    aggravated it. The court found the defendant innocent of manslaughter but 
    guilty of assault in the second degree.
\end{enumerate}

\subsubsection{Proximate Cause} 

\begin{enumerate}
    \item The doctrine of proximate cause determines whether an event that 
    satisfies the but-for standard should be held accountable for the 
    resulting harm.
    \item \textbf{Proximate cause answers the question of who is most culpable 
    for the harm.}
    \item \textbf{The MPC does not use the term ``proximate cause.''} Issues 
    related to proximate cause are treated as relating to the actor's 
    culpability. See MPC \S\ 2.03(2)(b) and (3)(b).
\end{enumerate}

\paragraph{Chain of Causation: \emph{People v. Rideout}}

Intervening causes that are not reasonably foreseeable break the chain of 
causation.

\begin{enumerate}
    \item The defendant was driving drunk and hit a car. The car's driver and 
    passenger suffered no major injuries, but the car was damaged enough so 
    that the headlights no longer worked. After moving safely to the side of 
    the road, the car's passenger entered the road to inspect the car to see 
    if they could turn on the hazard lights, where he was struck and killed by 
    an oncoming car. The trial court found the defendant guilty for the 
    passenger's death. The question is whether the defendant's drunk driving 
    was the proximate cause of the passenger's death.
    \item The appellate court introduced the ideas of \textbf{intervening and 
    superseding causes}. An intervening cause supersedes the original cause if 
    the original actor could not reasonably foresee the second cause, i.e., if 
    it breaks the chain of causation. Dressler divides the second cause into 
    \emph{responsive intervening causes}, which arise directly from the 
    original cause, and \emph{coincidental intervening causes}.
    \item The appellate court also introduced the \emph{apparent-safety 
    doctrine}, which holds that the defendant's causation ceases when the 
    victim has reached a place of apparent safety (e.g., far off on the side 
    of the road).
    \item The appellate court also introduced the idea of \emph{voluntary 
    human intervention}, which relieves the defendant's liability if the 
    victim voluntarily enters into a dangerous situation (e.g., a road in the 
    night without any lights).
    \item The appellate court overruled the trial court, finding that the 
    prosecution failed to establish proximate cause. It remanded the case for 
    a new trial.
    \item Later, the Michigan Supreme Court overturned the appellate court's 
    assessment that a jury could not find proximate cause.
\end{enumerate}

\paragraph{Superseding Intervening Cause: \emph{Velasquez v. State}}

A victim's own act can constitute a superseding intervening cause.

\begin{enumerate}
    \item The defendant and the victim were drag racing. After the race, the 
    victim spun around his car, raced back to the starting line, and careened 
    over a guardrail, dying instantly.
    \item The trial court found that the defendant's participation in the drag 
    race was a cause-in-fact of the victim's death. The appellate court found 
    that the drag race had already ended when the victim decided to spin 
    around and race back to the finish line---an act that superseded the 
    defendant's cause-in-fact.
\end{enumerate}
