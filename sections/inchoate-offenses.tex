\section{Inchoate Offenses}

\subsection{Overview}

\begin{enumerate}
    \item \textbf{Attempt}, \textbf{solicitation}, and \textbf{conspiracy} 
    consist of conduct meant to culminate in a substantive offense but has 
    failed or not yet culminated.
    % TODO: define all three
    \item Rationales for punishing inchoate crimes:
    \begin{enumerate}
        \item General deterrence.
        \item Specific deterrence: those engaged in inchoate criminal activity 
        have demonstrated their criminal proclivities.
        \item Prevention of the target offense.
        \item Restore balance to society.
        \item Because those who plan to engage in criminal activity are 
        morally blameworthy, even if they don't actually achieve their 
        criminal ends.
    \end{enumerate}
    \item Most jurisdictions treat inchoate offenses as distinct from the 
    crimes toward which they tend.
    \item The MPC aims to prohibit an act that is a ``substantial step'' 
    toward the offense.
    \item This category of offenses relies heavily on judicial and law 
    enforcement discretion.
    \item Inchoate offense law aims to protect innocent conduct while 
    preventing crimes in progress.
\end{enumerate}

\subsection{Attempt}

\begin{enumerate}
    \item Attempt occurs when a person, with the intent to commit an offense, 
    performs an act in furtherance of that offense.
    \item \textbf{Incomplete attempt}: an actor does some of the intended acts 
    and then stops or an extraneous factor stops her.
    \item \textbf{Complete attempt}: an actor does everything she planned but 
    is unsuccessful.
    \item \textbf{Merger}: the attempt merges into the completed offense. You 
    cannot be convicted of both the attempted crime and the completed crime.
    \item \textbf{Attempt is a specific intent crime}. You must have the 
    intent to complete the target offense. Therefore, \textbf{you cannot 
    attempt unintentional acts or conduct}---e.g., there can be no attempted 
    involuntary manslaughter or attempted reckless murder. Most jurisdictions 
    also \textbf{do not recognize attempted felony-murder} because attempt is 
    a specific intent crime and specific intent to kill is not an element of 
    felony murder.
    \item Common law: attempt is a lesser offense than the target crime and is 
    punished less harshly.
    \item MPC \S\ 5.05: inchoate offenses (including attempt) are punished at 
    the same level as the target offense, except for capital crimes and 
    first-degree felonies.
    \item The MPC and common law align on abandonment. 
    \item With \textbf{conduct crimes} (e.g., possession or drunk driving), a 
    person is not guilty of attempt unless he acts with specific intent to 
    cause the unlawful result.
    \item With \textbf{results crimes}, a person can be held liable if 
    intentionally committed the actus reus of the target crime with the mens 
    rea for the target crime. 
    \item MPC \S\ 5.01 (Murray's summary): a person is guilty of attempt if, 
    acting with the culpability required for the target offense, he:
    \begin{itemize}
        \item (a) Deals with completed attempts of conduct offenses.
        \item (b) Deals with completed attempts of results offenses.
        \item (c) Deals with incomplete attempts.
    \end{itemize}
    \item Common law attempt tests (from Murray):
    \begin{enumerate}
        \item \textbf{First step test}: Once it is clear that the defendant 
        has the purpose to commit a crime, anything the defendant does that 
        could lead to the completion of the crime would be a sufficient actus 
        reus to make the defendant guilty of attempt.
        \item \textbf{Last step test}: Where the actor has done all that is in 
        his power to do before he is prevented from committing the act by some 
        intervention. Reflects the common law view that defendant was not 
        guilty unless he had done all he could to commit a crime, but only 
        failed because of bad luck.
        \item \textbf{Physical proximity test}: Defendant is responsible for 
        the attempt if he is in a position of physical proximity that would 
        enable him to complete the target offense.
        \item \textbf{Dangerous proximity test}: An attempt occurs when the 
        defendant's conduct is in ``dangerous proximity to success,'' or when 
        an act ``is so near to the result'' that the danger of success is very 
        great.
        \begin{enumerate}
            \item How many steps towards the commission of the crime the 
            defendant has taken?
            \item How much more action is required for completion?
            \item Why was the crime not completed?
            \item The amount of harm likely to result from crime.
            \item Seriousness of prospective harm; and 
            \item Appropriateness of law enforcement's interference with defendant's acts.
        \end{enumerate}
        \item \textbf{Indispensable element test}: Considers whether the actor 
        has performed an act---or obtained control over something---that is 
        indispensable to the commission of the target offense. 
        \item \textbf{Probable desistance test}: When, in the ordinary course 
        of events, without interruption from an external source, the actor 
        reached a point where it was unlikely that he would have voluntarily 
        desisted from his effort to commit the crime. The test is not 
        concerned with how much needs to be done to commit the crime, but 
        rather, how much has already been done. Requires the jury to think 
        like a criminal and determine when an ordinary person in the 
        defendant's shoes would have done enough that he would not be able to 
        desist.
        \item \textbf{Abnormal step test}: An attempt is a step towards crime 
        that goes beyond the point where the normal citizen would think better 
        of his conduct and desist.  Requires the jury to think about what the 
        normal citizen (i.e. the reasonable person, or the ordinary person) 
        would in the defendant's circumstances.
        \item \textbf{\emph{Res ipsa loquitur}/unequivocality test}:  An 
        attempt occurs when a person's conduct, standing alone, unambiguously 
        manifests his criminal intent.  Considers whether the defendant's 
        actions, viewed in the abstract, demonstrate an unequivocal intent to 
        commit a crime.
    \end{enumerate}
    \item MPC vs. common law:
    \begin{enumerate}
        \item Common law focuses on what is left to be completed.
        \item MPC focuses on the actor's criminal disposition and considers 
        what has already been done in furtherance of the crime.
        \item The MPC imposes liability when there is firm evidence of 
        criminal intent, i.e., when the defendant has taken a substantial step 
        towards the completion of the offense. \S\S\ 5.01(2)(a)--(g) identify 
        scenarios that are strongly corroborative of criminal purpose.
    \end{enumerate}
    \item \textbf{Legal and factual impossibility}: see \emph{Thousand} below.
    \item In jurisdictions that recognize it, the defendant can claim the 
    \textbf{defense of abandonment} if he voluntarily and completely renounced 
    his criminal purpose. See \emph{McCloskey} below. The MPC recognizes 
    abandonment (aka renunciation) in \S\ 5.01(4).
\end{enumerate}

\subsubsection{General principles}

\paragraph{Robbins, ``Double Inchoate Crimes''}

\begin{enumerate}
    \item Many jurisdictions have a few specific attempt rules alongside a 
    general attempt statute.
    \item Purpose is not deterrence but rather to give law enforcement a basis 
    for intervention.
    \item There are two varieties of criminal attempt:
    \begin{enumerate}
        \item \textbf{Incomplete}: Actor is interrupted.
        \item \textbf{Complete}: Actor does every act planned but fails to 
        cause the intended result (e.g., shoots and misses).
    \end{enumerate}
    \item If the basis for attempt law is to support law enforcement 
    intervention, what is the basis for punishing completed attempts?
    \item Do attempts cause social harm?
\end{enumerate}

\paragraph{Ashworth, ``Criminal Attempts and the Role of Resulting Harm under 
the Code, and in the Common Law''}

\begin{enumerate}
    \item Prevention is the main reason for punishing preliminary steps on the 
    way to causing harm.
    \item Justifications for punishing attempt:
    \begin{enumerate}
        \item Retributivism: it tends to ``restore an order of 
        fairness.''\footnote{Casebook p. 734.} \textbf{Harm-based} 
        retributivism is inapplicable unless the definition of harm is 
        broadened to include a presumed apprehension of fear of attempters.  
        \textbf{Intent-based} retributivism holds individuals liable for their 
        intentions, and \textbf{belief-based} retributivism for their belief 
        that what they were doing was wrong.
        \item Utilitarianism: incapacitation, specific deterrence, and 
        sometimes rehabilitation.
    \end{enumerate}
    \item Justifications for punishing the perpetrator of the completed 
    offense:
    \begin{enumerate}
        \item Harm-based retributivism: the apprehension of fear can provide a 
        basis for punishment.
        \item Intent-based retributivism: ``no relevant moral difference'' 
        between a completed attempt and a successful crime.\footnote{Casebook 
        p.  735.}
        \item Consequentialism: the effect of punishment must outweigh its 
        consequences. Complete attempters show clear propensities for causing 
        harm, so punishment is called for.
    \end{enumerate}
    \item George Fletcher's ``two patterns of criminality'':\footnote{Casebook 
    p. 736.}
    \begin{enumerate}
        \item ``Objectivist'': an act is criminal if a neutral third party 
        could recognize the criminality of the actor's conduct.
        \item ``Subjectivist'': the actor's intentions create criminality.
    \end{enumerate}
\end{enumerate}

\paragraph{ALI Comment to MPC \S\ 5.05}

\begin{enumerate}
    \item Should attempts be treated as lesser offenses than successfully 
    completed crimes?
    \item One common law formula fixed the punishment for attempt at half of 
    the maximum for the completed crime, or 10--50 years for crimes punishable 
    by death or life in prison.
    \item Traditionally, criminal attempts were punished less severely than 
    completed offenses, even if the only difference was bad luck. ``...the 
    reward for failing, no matter how hard you try to succeed or how close you 
    come, is lesser punishment.''\footnote{Sanford H. Kadish, Casebook p.  
    737.}
    \item MPC: punishment for attempt, solicitation, and conspiracy is 
    determined ``by the gravity of the most serious offense that is its 
    object.'' The completion or failure of the plan shouldn't matter because 
    there is little deterrent force.
    \item Should an attempt be treated as a less serious offense than the 
    target crime?
\end{enumerate}

\subsubsection{Mens Rea}

\paragraph{Attempted Murder and Intent: \emph{People v. Gentry}}
~\\\\
Since murder is a specific intent crime that requires intent to kill, 
attempted murder also requires intent.

\begin{enumerate}
    \item Gentry had spilled or poured gasoline on his girlfriend which then 
    accidentally ignited. The jury convicted him of attempted murder.
    \item The jury instructions included all four culpable mental states as 
    possible components of murder. Gentry argued on appeal that murder 
    requires specific intent.
    \item The appellate court agreed, holding that both attempted murder and 
    murder require specific intent to kill. Knowledge is insufficient.  
    Reversed.
    \item Criminal law involves two ``intents'': intentional conduct and 
    intent to commit the completed offense. They often merge, but they would 
    be separate if, for instance, an actor shot a gun merely to scare the 
    victim but accidentally killed him. Both intents must be 
    proven.\footnote{Casebook p. 740.}
    \item Dressler on applying MPC \S\ 5.01:\footnote{Casebook p. 741 n. 4.}
    \begin{enumerate}
        \item Was it a complete or incomplete attempt?
        \item Was the target crime an offense (e.g., murder) or conduct (e.g., 
        drunk driving) crime?
        \item 1(a) and 1(b) apply to complete attempts. 1(c) applies to 
        complete attempts.
    \end{enumerate}
\end{enumerate}

\paragraph{Attempted Felony Murder: \emph{Bruce v. State}}
~\\\\
Felony murder requires no specific intent to kill. Thus, there can be no 
attempted felony murder. Unless you live in Florida.

\begin{enumerate}
    \item Bruce entered the victim's shoe store with a loaded gun and demanded 
    money from the cash register. The victim ducked out of the way and Bruce 
    shot him, causing injury but not death.
    \item The trial court convicted Bruce of attempted first degree felony 
    murder.
    \item The appellate court held that criminal attempt requires specific 
    intent to commit a particular offense. Felony murder, however, requires no 
    specific intent to kill. Thus, there can be no attempted felony murder.  
    Reversed.
    \item (Most states agree, but Florida does not.\footnote{Casebook p.  
    743.})
    \item Can you be guilty of attempted statutory rape? Under the MPC the 
    answer is yes. \textbf{To be guilty of attempt, the actor must have acted 
    with the mental state required for the target offense.} Statutory rape is 
    a strict liability offense, i.e., no particular mental state is required.  
    Since mistake of age is irrelevant for the target offense, it is likewise 
    irrelevant for the attempt.\footnote{Casebook p. 745.}
\end{enumerate}

\subsubsection{Actus Reus}

\paragraph{\emph{United States v. Mandujano}}
~\\\\
Attempting to define attempt. See also the list of common law attempt tests at 
the top of this section.

\begin{enumerate}
    \item \emph{United States v. Noreikis}: the distinction between 
    preparation and attempt ``is one incapable of being formulated into a hard 
    and fast rule.''\footnote{Casebook p. 746.}
    \item \emph{United States v. Coplon}: attempt is when ``he has done all 
    that it is within his power to do, but has been prevented by intervention 
    from outside'' (Learned Hand).
    \begin{enumerate}
        \item \emph{Locus poenitentiae}: ``place of repentance.''
    \end{enumerate}
    \item \emph{Mims v. United States}, relying on a test from \emph{People v.  
    Buffum}: an ``appreciable fragment'' must have been committed, it must be 
    in progress such that it will be completed unless interrupted, and it must 
    not be equivocal.
    \item Others (from the case notes):
    \begin{enumerate}
        \item \emph{United States v. Oviedo}: attempt exist if the objective 
        acts, ``without any reliance on the \emph{mens rea}, mark the 
        defendant's conduct as criminal in nature.''\footnote{Casebook p.  
        747.}
        \item \emph{Stokes v. State}: if the ``design of a person to commit a 
        crime is clearly shown, slight acts done in furtherance of this design 
        will constitute an attempt.''
        \item \emph{People v. Luna}: if intent is ``clearly shown,'' any act 
        toward commission constitutes attempt.
        \item Sayre, ``Criminal Attempts'': the more serious the crime, the 
        ``further back'' in the series of preliminary acts should the law look 
        for acts constituting attempt.
        \item Enker, ``Impossibility in Criminal Attempts---Legality and the 
        Legal Process'': courts must weigh several factors, including (1) 
        whether the act is ``sufficiently close to the substantive crime,'' 
        (2) whether the actor's conduct makes one ``reasonably certain that he 
        is firmly committed to a specific illegal venture,'' and (3) whether 
        ``the act is sufficiently unambiguous to demonstrate the actor's 
        illegal intent.''\footnote{Casebook p. 748.}
    \end{enumerate}
\end{enumerate}

\paragraph{Locus Poenitentiae: \emph{Commonwealth v. Peaslee}}
~\\\\
Preparation becomes attempt if it ``comes very near'' to the completed act.

\begin{enumerate}
    \item The defendant had prepared to burn down a building. He asked one of 
    employees to start the fire, and the employee refused. Later, the two of 
    them drove toward the building to be burned, but turned back a quarter of 
    a mile away.
    \item The question was whether the defendant's actions ``near enough to 
    the accomplishment of the substantive offense to be 
    punishable.''\footnote{Casebook p. 750.}
    \item If preparation ``comes very near'' to the completed act, it can be 
    punished as attempt. But in this case, preparation to set the fire without 
    any intent to actually light it is ``too remote.''
\end{enumerate}

\paragraph{No Chance of Success: \emph{People v. Rizzo}}
~\\\\
There can be no attempt if there is no chance of success.

\begin{enumerate}
    \item The defendant and three others drove around looking for a man they 
    intended to rob. When the defendant jumped out of the car to look for the 
    man, all four were arrested. It turned out that the person they intended 
    to rob was nowhere nearby.
    \item The court held that there cannot be an attempt if there is no chance 
    of success. ``...these defendants had planned to commit a crime and were 
    looking around the city for an opportunity to commit it, but the 
    opportunity fortunately never came.''\footnote{Casebook p. 754.}
\end{enumerate}

\paragraph{\emph{People v. Miller}}
~\\\\
With clear intent, any slight act done in furtherance of the target offense 
constitutes attempt.

\begin{enumerate}
    \item The defendant had earlier threatened to kill Albert Jeans. Later 
    that day, he went, carrying a loaded rifle, to a field where Jeans and the 
    constable were planting hops. He surrendered his gun to the constable.
    \item The court cited \emph{Stokes}, which held that with clear intent, 
    any slight act done in furtherance constitutes attempt. But the 
    \emph{Stokes} test, the court held, ``still presupposes some direct act or 
    movement in the execution of the design.''\footnote{Casebook p. 756.} As 
    long as the actor remains equivocal, there can be no attempt.
\end{enumerate}

\paragraph{Preparation and Substantial Steps: \emph{State v. Reeves}}
~\\\\
An actor is guilty of attempt if she takes substantial steps towards the 
completed offense.

\begin{enumerate}
    \item Two twelve-year-old girls decided to kill their teacher with rat 
    poison. The teacher saw the girls standing over her desk and giggling. The 
    school found rat poison in one of their purses.
    \item The trial court found them guilty of attempted second-degree murder.
    \item The current Tennessee statute, based on the MPC, finds attempt if 
    the ``conduct constitutes a substantial step toward the commission of the 
    offense.''\footnote{Casebook p. 760.}
    \item The defendant argued that the state deliberately omitted the 
    examples in MPC \S\ 510.2 of conduct that are strongly corroborative of 
    the actor's criminal intent. However, the court was not convinced that 
    this omission meant that the legislature intended to retain the sharp 
    distinction between ``mere preparation'' and the ``act itself.'' It held 
    that the jury is free to apply the MPC's ``substantial step'' rule.
\end{enumerate}

\subsubsection{Special defenses}

\subsubsection{Impossibility: \emph{People v. Thousand}}

\begin{enumerate}
    \item Thousand sent lewd pictures to someone he thought was an underage 
    girl but who turned out to be an undercover cop. He was charged with 
    attempted distribution of obscene material to a minor.
    \item Thousand argued that the existence of a child was a required element 
    of the offense and he moved for dismissal. The trial court granted the 
    motion and the appellate court affirmed.
    \item \textbf{Factual impossibility}: the defendant intends to commit a 
    crime but fails because of a factual circumstance unknown to her or beyond 
    her control---e.g., trying to kill someone by pulling the trigger of an 
    unloaded gun. Never recognized as a defense.
    \item \textbf{Pure legal impossibility}: criminal law does not prohibit the 
    actor's conduct or intended result---e.g., a man has sex with a 
    fifteen-year-old believing that the statutory rape law sets the minimum 
    age at sixteen, but in fact sets it at fifteen. He would not be found 
    guilty.
    \item \textbf{Hybrid legal impossibility}: the actor's goal was illegal but 
    he could not complete it because of a factual mistake regarding the legal 
    status of a relevant factor---e.g., shooting a corpse believing it is 
    alive.
    \item Any instance of hybrid legal impossibility can be redrawn as factual 
    impossibility and is therefore not available as a defense in most 
    jurisdictions.
    \item The court here declined to accept factual impossibility or hybrid 
    legal impossibility as defenses. Reversed.
\end{enumerate}

\subsubsection{Abandonment: \emph{Commonwealth v. McCloskey}}

Abandonment is a defense to attempt. It is distinct from failing to complete 
the attempt.

\begin{enumerate}
    \item The defendant prepared to escape from prison. He began to escape, 
    including cutting barbed wire, but changed his mind.
    \item The trial court found him guilty of attempted prison breach.
    \item The Supreme Court of Pennsylvania reversed on the grounds that the 
    defendant had ``not yet attempted the act.''\footnote{Casebook p. 787.}
    \item Judge Cercone, concurring, agreed with the outcome, but argued that 
    the basis should be the defense of \textbf{abandonment}. Otherwise, the 
    prison guards would not have been able to stop the defendant's escape 
    until he was scaling the prison walls.
    \begin{enumerate}
        \item The PA legislature substantially adopted the MPC, including \S\ 
        5.01. He argued that the court had long ago adopted abandonment as an 
        affirmative defense. And if it hadn't, it should 
        have.\footnote{Casebook p. 788.}
    \end{enumerate}
\end{enumerate}

\subsection{Assault}

\begin{enumerate}
    \item Under common law, mayhem consisted of injury ``impairing the 
    victim's ability to defend himself or to annoy his 
    adversary.''\footnote{Casebook p. 790.}
    \item Battery: any offensive and unlawful contact.
    \item Assault was \textbf{originally just the attempt to commit battery}. 
    It required stricter proximity than ordinary attempt. It evolved to 
    include menacing and actual attempts, as well as conditional assaults 
    (i.e., threats).
    \item The MPC removed the common law categories and implemented a single 
    definition under \S\ 211.1. It consolidates assault and battery and 
    removes the increased proximity requirement.
\end{enumerate}

\subsection{Solicitation}

\begin{enumerate}
    \item Solicitation is the \textbf{asking, enticing, inducing, or 
    counseling of another to commit a crime}. Under the common law approach, 
    the {target offense must have been a felony}.
    \item MPC \S\ 5.02: ``A person is guilty of solicitation to commit a crime 
    if with the purpose of promoting or facilitating its commission he 
    commands, encourages or requests another person to engage in specific 
    conduct that would constitute such crime or an attempt to commit such 
    crime or which would establish his complicity in its commission or 
    attempted commission.''
    \item \textbf{Solicitation merges if the offense is attempted or 
    completed.}
    \item Abandonment and renunciation are defenses to solicitation where the 
    defendant \textbf{prevents the solicitee from committing the target 
    crime}.
\end{enumerate}

\subsubsection{Defining Solicitation: \emph{State v. Mann}}

\begin{enumerate}
    \item ``Solicitation involves the asking, enticing, inducing, or 
    counselling of another to commit a crime.''\footnote{Casebook p. 792.}
    \item ``...the solicitor is morally more culpable than a conspirator...''
    \item \emph{Merger}: ''The offense of solicitation merges into the crime 
    solicited if the latter attempt is committed or attempted by the solicited 
    party.''\footnote{Casebook p. 793.}
\end{enumerate}

\subsubsection{Completed Communication: \emph{State v. Cotton}}

The MPC criminalizes solicitations that fail to reach the intended recipient, 
but New Mexico does not recognize this provision.

\begin{enumerate}
    \item While in prison, the defendant wrote letters to his wife asking her 
    to help prevent his step-daughter from testifying against him. His 
    cellmate covertly removed the letters from their envelopes and turned them 
    over to the authorities.
    \item The defendant was convicted of two counts of criminal solicitation.
    \item The appellate court noted that the New Mexico state legislature 
    explicitly omitted MPC \S\ 5.02(2), which criminalizes solicitations that 
    fail to reach the intended recipient. The court reasoned that this 
    omission indicates the legislature's intent to require actual 
    communication for solicitation to be accomplished.
    \item Reversed.
\end{enumerate}

\subsection{Conspiracy}

\begin{enumerate}
    \item At common law, a conspiracy was a mutual agreement or understanding, 
    express or implied, between two or more persons to commit a criminal act 
    or to accomplish a legal act by unlawful means.  Modernly, jurisdictions 
    define a conspiracy as a \textbf{(1) mutual agreement or understanding, 
    express or implied, (2) between two or more persons (3) to commit a 
    criminal act.} The charge also requires \textbf{proof of an overt act by 
    one of the parties to the agreement in furtherance thereof}.
    \item Agreement must be proven by inference and circumstantial evidence.
    \item Conspiracy involves (1) intent to form an agreement and (2) intent 
    to commit the elements of the target offense.
    \item Under the common law, conspiracy \textbf{does not merge} into the 
    completed offense. Under the MPC, conspiracy \textbf{\emph{does} merge}.
    \item \textbf{Pinkerton liability} holds each conspirator liable for the 
    criminal acts of any co-conspirator. There is \textbf{no Pinkerton 
    liability in the MPC}, thought most jurisdictions where the MPC has 
    influence have retained Pinkerton liability. \footnote{``Law would lose all sense of just proportion if 
    simply because of the conspiracy itself each [conspirator] were held 
    accountable for thousands of additional offenses of which he was 
    completely unaware and which he did not influence at all'' ALI 
    commentary to MPC \S\ 2.06.}).
    \item Conspiracy is a \textbf{specific intent crime}. The prosecution must 
    prove intent to join the agreement---e.g., you cannot accidentally join a 
    conspiracy.
    \item Circumstances can, however, are sometimes sufficient to prove that 
    the defendant's knowledge of the criminal activity demonstrates his intent 
    to participate (see \emph{Lauria} below):
    \begin{enumerate}
        \item When the purveyor of legal goods for illegal uses has acquired a 
        stake in the criminal venture.
        \item When no legitimate use for the goods or services exists.
        \item When the volume of business with the buyer is grossly 
        disproportionate to any legitimate demand.
        \item Under the MPC, there is no conspiracy where a provider of goods 
        or services is aware of the criminal activity but does not share the 
        criminal purpose.
    \end{enumerate}
    \item The MPC requires an \textbf{overt act} to prove 
    conspiracy.\footnote{MPC \S\ 5.03(5).} See \emph{Sconce} below.
    \item Common law factors from which intent can be inferred: association, 
    knowledge, presence, participation.\footnote{Casebook p. 818.}
    \item Most jurisdictions follow the common law for conspiracy, not the 
    MPC.
    \item MPC vs. common law:
    \begin{enumerate}
        \item The MPC allows unilateral conspiracies (``conspiracies of 
        one'').\footnote{MPC \S\ 5.03(2).}
        \item \textbf{Vicarious liability} (Murray): Under the MPC, If a person guilty of 
        conspiracy under \S\ 5.03(1) knows that a person with whom he conspires to 
        commit a crime has conspired with another person or persons to commit the 
        same crime, he is guilty of conspiring with such other person or persons, 
        whether or not he knows their identity, to commit such crime.
    \end{enumerate}
    \item Why prosecutors love conspiracies (Murray):
    \begin{enumerate}
        \item Separate crime with its own penalties
        \item Allows apprehension of the defendant at an earlier stage than 
        attempt.
        \item As we will discuss, members of a conspiracy are vicariously 
        liable for the acts of their co-conspirators in furtherance of the 
        conspiracy.
        \item Allows for the apprehension and prosecution of large groups of 
        individuals \item Conspiracy is a continuing offense, which gives 
        prosecutors a long time in which to file charges.
        \item Prosecutors may file charges for conspiracy in any venue where 
        an act of the conspiracy occurred.
        \item Evidentiary exceptions permit the admissibility of 
        co-conspirator's statements.
        \item Under federal law, conspiracy aggravates the degree of the 
        target offense.
    \end{enumerate}
    \item \textbf{Multiple crimes} under the MPC (Murray):If a person 
    conspires to commit a number of crimes, he is guilty of only one 
    conspiracy so long as such multiple crimes are the object of the same 
    agreement or continuous conspiratorial relationship.
    \item \textbf{Wheel conspiracy}: the spokes do not know each other, but 
    they are all connected to a central hub. A shared interest connects them 
    through the ``rim.'' A wheel conspiracy is not complete unless there is 
    the rim. See \emph{Kilgore} below, and \emph{Kotteakos}, where the Supreme 
    Court held that small conspiracies could not be shown to be a larger 
    conspiracy unless the spokes shared a common interest---e.g., if the 
    individual borrowers used part of the proceeds obtained by the others' 
    loans as the down payments for their loans.
    \item \textbf{Chain conspiracy}: several layers of personnel. Their 
    individual success depends on the success of the entire chain.
    \item \textbf{Withdrawal}: see \emph{Sconce} below.
    \begin{enumerate}
        \item Common law: once the agreement is made, there is often no 
        defense to conspiracy, though abandonment can serve as a defense to 
        future conspiratorial acts. In many jurisdictions, abandonment 
        requires communication of withdrawal to co-conspirators, and many 
        require the abandoner to dissuade the other conspirators to abandon 
        also, or to otherwise thwart the goal.
        \item MPC \S\ 5.03(6): ``It is an affirmative defense that the actor, 
        after conspiring to commit a crime, \textbf{thwarted} the success of 
        the conspiracy under circumstances manifesting a complete and 
        voluntary \textbf{renunciation} of the criminal purpose.''
        \item The MPC \textbf{does} allow abandonment as a defense to 
        conspiracy.
    \end{enumerate}
\end{enumerate}

\begin{enumerate}
    \item Justice Holmes: conspiracy is ``a partnership in criminal 
    purposes.''\footnote{\emph{United States v. Kissel}, 218 U.S. 601 (1910).} 
    Partnership requires an agreement prior to the criminal activity.
\end{enumerate}

\subsubsection{General Principles}

\paragraph{Drug Smuggling: Kerman, \emph{Orange is the New Black: My Year in a 
Women's Prison}}

\begin{enumerate}
    \item Kerman was indicted on conspiracy charges for drug smuggling and 
    money laundering. She pleaded guilty and received a 30-month sentence.
    \item She was liable for the entire scope of the conspiracy.
\end{enumerate}

\paragraph{\emph{People v. Carter}}

\begin{enumerate}
    \item ```The gist of the offense of criminal conspiracy lies in the 
    unlawful agreement.' The crime is complete upon the formation of the 
    agreement...''\footnote{Casebook p. 797.}
    \item Some jurisdictions require an overt act, but the threshold is low.
    \item \textbf{The specific intent is twofold: (1) intent to combine with 
    others and (2) intent to accomplish the illegal objective.}
    \item Conspiracy fills the gap created when attempt is too narrowly 
    conceived.
    \item Often used to fight organized crime.
    \item ``...collective criminal agreement---partnership in crime---presents 
    a greater potential threat to the public than individual 
    delicts.''\footnote{Casebook p. 798.}
    \item The \emph{Callahan} court argued that conspiratorial groups are 
    dangerous because they can lead to criminal activity beyond the original 
    intended crime. Others have argued that conspiracies are less dangerous 
    because plans are more likely to leak or that some conspirators' doubts 
    will influence other participants.
    \item Katyal: groups polarize toward extremes and are thus more 
    dangerous.\footnote{Casebook p. 799.}
    \item Acts that are immoral but not criminal can be punished with 
    conspiracy charges.\footnote{Casebook p. 799--800.}
    \item Spectrum of inchoate offenses: solicitation \textless conspiracy 
    \textgreater attempt.
    \item Is ``attempted conspiracy'' cognizable or is it the same as 
    solicitation?
\end{enumerate}

\paragraph{Pinkerton Liability: \emph{Pinkerton v. United States}}
~\\\\
Pinkerton liability holds each conspirator liable for the criminal acts of any 
co-conspirator.

\begin{enumerate}
    \item Two brothers were found guilty of conspiracy and multiple 
    substantive counts of violating the Internal Revenue Code. Daniel did not 
    participate in the substantive crimes. The question is whether 
    participation in the conspiracy is enough to find the conspirator guilty 
    of the substantive crimes.
    \item Daniel argued that direct participation was necessary to find guilt 
    on the substantive counts.
    \item The Supreme Court disagreed, holding that ``an overt act of one 
    partner may be the act of all without any new agreement specifically 
    directed to that act.''\footnote{Casebook p. 802.}
    \item Judge Rutledge, dissenting: such vicarious liability is appropriate 
    in civil but not criminal contexts. He finds it ``dangerous'' but does not 
    explain why.
\end{enumerate}

\subsubsection{\emph{Mens rea}}

\paragraph{Conspiracy to Commit Implied Malice Murder: \emph{People v. Swain}}
~\\\\
It is not possible to conspire to commit a crime that does not require 
specific intent.
% TODO: Can you conspire to commit rape, or another general intent crime?

\begin{enumerate}
    \item Swain and Chatman were both convicted of conspiracy to commit murder 
    and other crimes ``stemming from the drive-by shooting death of a 
    15-year-old boy.''\footnote{Casebook p. 806.}
    \item In jail, Swain boasted about being a good shot. But at trial, he 
    testified that he was not in the van at the time of the shooting.
    \item Chatman claimed the original plan was to steal the decedent's car.  
    He claimed he fired shots in self defense.
    \item The jury found Chatman guilty of second-degree murder and 
    conspiracy. It found Swain guilty of conspiracy but not murder.
    \item The question on appeal was whether conspiracy to commit murder 
    requires express malice or whether it is possible to conspire to commit 
    implied malice murder. Implied malice murder does not require an intent to 
    kill.
    \item California recognizes three kinds of second-degree 
    murder:\footnote{Casebook p. 808.}
    \begin{enumerate}
        \item Unpremeditated with express malice.
        \item Implied malice (i.e., depraved heart).
        \item Felony murder.
    \end{enumerate}
    \item The charge here was implied malice murder, which does not require a 
    specific intent to kill.
    \item Conspiracy requires specific intent to commit the elements of the 
    target crime, it is not possible to conspire to commit a crime that does 
    not require specific intent.
    \item Reversed.
\end{enumerate}

\paragraph{Suppliers as Conspirators: \emph{People v. Lauria}}
~\\\\
A suppliers of goods can be held liable as a conspirator (1) when he overtly 
intends to participate in criminal activity, or (2) when intent can be 
inferred on (a) his interest in the activity or (b) the seriousness of the 
crime.

\begin{enumerate}
    \item Lauria operated an answering service. He knew many of his customers 
    used it for prostitution.
    \item Lauria was indicted for conspiracy to commit prostitution. The trial 
    court set aside the indictment for lack of ``reasonable or probable 
    cause.''\footnote{Casebook p. 810.}
    \item The appellate court asked, ``[u]nder what circumstances does a 
    supplier become part of a conspiracy to further an illegal enterprise by 
    furnishing goods or services which he knows are to be used by the buyer 
    for criminal purposes?''\footnote{Casebook p. 810.}
    \item Two Supreme Court cases addressed this question:
    \begin{enumerate}
        \item \emph{Falcone}: a seller of sugar, yeast, and cans was not 
        guilty in a moonshining conspiracy, even though it knew of the 
        criminal activity.
        \item \emph{Direct Sales}: a morphine distributor was found guilty of 
        conspiracy for selling 300 times the normal requirement of the drug to 
        a physician.  \item The rule is that intent to ``further, promote, and 
        cooperate'' in the criminal activity must be present for the actor to 
        be guilty of conspiracy.
    \end{enumerate}
    \item The appellate court identified three areas where intent can be 
    inferred:
    \begin{enumerate}
        \item When the seller of goods has a stake in the criminal activity 
        (e.g., renting a room at inflated prices to a prostitute).
        \item When there is no legitimate use for the goods or services (e.g., 
        supplying horse-racing information by wire).
        \item When the volume of business with the buyer is ``grossly 
        disproportionate to any legitimate demand.''
    \end{enumerate}
    \item The court developed a two-part rule for establishing the intent of a 
    supplier: (1) when he overtly intends to participate in criminal activity, 
    or (2) when intent can be inferred on (a) his interest in the activity or 
    (b) the seriousness of the crime.\footnote{Casebook p. 813.}
    \item Inferences of intent do not apply to misdemeanors.
    \item The conspiracy charges against Lauria do not stand because he was 
    only charged with furthering a misdemeanor.
\end{enumerate}

\subsubsection{Actus Reus}

\paragraph{Goldstein, ``Conspiracy to Defraud the United States''}

\begin{enumerate}
    \item The conspiratorial agreement is a ``theoretical construct.'' By 
    calling it an act, ``courts foster the already elaborate illusion that 
    conspiracy reaches actual, not potential, harm.''\footnote{Casebook p.  
    816.}
    \item Juries can find conspiracy ``on less evidence than might otherwise 
    be required.''
\end{enumerate}

\paragraph{Inferring Conspiracy: \emph{Commonwealth v. Azim}}
~\\\\
Conspiracy can be established by inference.

\begin{enumerate}
    \item Azim was carrying two men in his car. He stopped the car for them to 
    assault and rob another man. He was convicted of assault, robbery, and 
    conspiracy.
    \item On appeal, Azim argued that ``because his conspiracy conviction was 
    not supported by sufficient evidence against him, the charges of assault 
    and robbery must also fail.''\footnote{Casebook p. 818.}
    \item In \emph{Volk}, the court held that conspiracy could ``be 
    inferentially established''---i.e., the agreement need not be explicit.
    \item The court held that there was sufficient evidence of conspiracy, and 
    because conspiracy was established, Azim was also guilty of the criminal
    acts of his co-conspirators.
\end{enumerate}

\paragraph{\emph{Commonwealth v. Cook}}
~\\\\
Conspiracy can also be difficult to establish by inference. Cf. \emph{Azim}, 
above, which is difficult to reconcile with this case.

\begin{enumerate}
    \item Cook was convicted of conspiracy to commit rape. On appeal, he 
    argued that there was insufficient evidence to establish a conspiracy.
    \item The victim visited a housing project to visit her boyfriend, who 
    turned out not to be home. The defendant and his brother invited her to 
    hang out on their porch. After forty-five minutes, Cook's brother 
    suggested that the three of them go to a convenience store to buy 
    cigarettes. On a path along the way, the victim slipped, and Cook's 
    brother raped her.
    \item The Court found no conspiracy because there was no evidence of a 
    preconceived plan, the conduct up to the crime was in the open, and the 
    criminal action was spontaneous.
    \item The Court also rejected the Commonwealth's argument that Cook's 
    complicity as an accomplice made him a co-conspirator. There was no prior 
    agreement to commit a criminal act.
    \item Reversed.
\end{enumerate}

\subsubsection{Scope of the Agreement}

\paragraph{ALI Commentary to MPC \S\ 5.03}

\begin{enumerate}
    \item \textbf{``The scope problem''}: ``Has a retailer conspired with the 
    smugglers to import the narcotics? Has a prostitute conspired with the 
    leaders of the vice ring to commit the acts of prostitution of each other 
    prostitute who is controlled by the ring?''
\end{enumerate}

\paragraph{Wheel and Chain Conspiracies: \emph{Kilgore v. State}}
~\\\\
A wheel conspiracy requires multiple ``spokes'' from a ``hub'' with a ``rim'' 
connecting the spokes.

\begin{enumerate}
    \item Facts:
        \item February 6, 1981: David Oldaker and Greg Benton attempted to 
        kill Roger Norman at the request of Tom Harden.
        \item June 8, 1981: Kilgore (the defendant) and Lee Berry allegedly 
        shot Norman in the back while he was driving but did not kill him.
        \item July 8, 1981: Kilgore and Bob Price allegedly successfully 
        killed Norman. Tom Carden allegedly had given them money.
    \item Kilgore was convicted of murder.
    \item At trial, Oldaker testified that Benton told him that the man who 
    wanted Norman killed was Tom Carden. The Supreme Court of Georgia found 
    that this testimony was admissible only if Oldaker, Benton, and Kilgore 
    were co-conspirators. This was a ``wheel'' conspiracy in which Tom Carden 
    communicated with each of the ``spokes'' individually---however, there was 
    no evidence suggesting that Oldaker and Benton had any contact with 
    Kilgore. Thus, the three were not co-conspirators and Oldaker's testimony 
    was inadmissible.
    \item (It's not clear whether the court upheld the conviction after it 
    determined that the testimony was inadmissible hearsay.)
\end{enumerate}

\paragraph{Single Agreement, Multiple Offenses: \emph{Braverman v. United 
States}}
~\\\\
A single agreement can only be punished as a single conspiracy, even if it 
aimed at several criminal offenses.

\begin{enumerate}
    \item Defendant moonshiners were indicted on seven counts of conspiracy to 
    violate U.S. internal revenue laws. The defendants moved to require the 
    government to choose one of the seven counts, arguing that the evidence 
    could not prove seven different conspiracies. The government argued that 
    the indictment was proper because there were seven different criminal 
    goals of a single ongoing conspiracy.
    \item The jury returned a guilty verdict on each of the seven counts. The 
    Sixth Circuit affirmed.
    \item The government conceded that there was a single agreement among the 
    conspirators. The Supreme Court held that a single agreement can only be 
    punished as a single conspiracy, even if it aimed at several criminal 
    offenses. Reversed.
\end{enumerate}

\subsubsection{Defenses}

\paragraph{Wharton's Rule: \emph{Ianelli v. United States}}
~\\\\
Wharton's Rule prevents conspiracy charges for crimes requiring multiple 
participants. Wharton's rule presumes merger.

\begin{enumerate}
    \item The eight petitioners were convicted of conspiring to violate and 
    violating a federal gambling statute which criminalized gambling 
    businesses involving at least five people.\footnote{18 U.S.C. \S\ 1955.}
    \item The Third Circuit affirmed.
    \item ``Wharton's Rule'' prevents conspiracy charges for crimes requiring 
    multiple participants.\footnote{Casebook p. 839.} The classic ``Wharton's 
    Rule'' offenses are adultery, incest, bigamy, and 
    duelling.\footnote{Casebook p. 841.}
    \item Courts were divided on whether the rule requires the conspiracy 
    charge to be dismissed before trial or whether a prosecutor can charge 
    both and instruct the jury that conviction for the offense precludes 
    conviction for the conspiracy.
    \item Courts were also divided on whether Wharton's Rule applies when 
    there are more people involved than are necessary to commit the offense.
    \item \textbf{The MPC does not recognize Wharton's Rule.}
    \item One exception exists for when the two conspirators are not the two 
    who will commit the crime (e.g., ``go commit adultery with my wife''). In 
    those cases, Wharton's Rule does not apply.
    \item Wharton's Rule was intended to apply to offenses that ``require 
    concerted criminal activity... absent legislative intent to the contrary, 
    the Rule supports a presumption that the two merge when the substantive 
    offense is proved.''\footnote{Casebook p. 842.} Here, however, the 
    legislative history of the gambling statute shows a clear intent to target 
    organized crime, so it makes sense not to merge the conspiracy and the 
    substantive offense.
\end{enumerate}

\paragraph{\emph{Gebardi v. United States}}
~\\\\
Under the common law, acquiescence to activity in which acquiescence does not 
estasblish guilt is insufficient to establish conspiracy for either 
conspirator. Under the MPC, conspiracy can be unilateral.

\begin{enumerate}
    \item Gebardi was convicted of conspiracy to violate the Mann Act 
    (transporting ``any woman or girl for the purpose of prostitution or 
    debauchery, or for any other immoral purpose''\footnote{Casebook p. 844}).
    \item The Mann Act requires more than the woman's ``mere acquiescence.'' 
    She must ``aid or assist'' her own transport. The \textbf{``legislative 
    exemption defense''} holds that a victim cannot be guilty of contributing 
    to her own victimization. \textbf{Victims cannot be co-conspirators.}
    \item ``As there is no proof that the man conspired with anyone else to 
    bring about the transportation, the convictions of both petitioners [for 
    conspiracy] must be reversed.''\footnote{Casebook p. 845.}
    \item (Under MPC \S\ 5.04(1), the court would likely find the man guilty 
    of conspiracy.)
\end{enumerate}

\paragraph{Withdrawal from Conspiracy: \emph{People v. Sconce}}

\begin{enumerate}
    \item Sconce offered Garcia \$10,000 to kill Estephan. Garcia offered 
    \$5,000 to Dutton to perform the killing. Three weeks later, Sconce told 
    Garcia to call it off. Meanwhile, Dutton had been arrested on a parole 
    violation.
    \item The trial court set aside the information relating to conspiracy 
    because Sconce had withdrawn.
    \item The appellate court held that withdrawal is a complete defense to 
    conspiracy only if it occurs before an overt act. Here, Sconce had already 
    completed the conspiracy, and the overt act was giving the money, so his 
    withdrawal only precluded liability for future acts of his 
    co-conspirators.
    \item ``...withdrawal from the conspiracy is not a defense to the 
    completed crime of conspiracy...''\footnote{Casebook p. 847.}
    \item Reversed.
\end{enumerate}

