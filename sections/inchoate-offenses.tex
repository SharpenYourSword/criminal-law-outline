\section{Inchoate Offenses}

\subsection{Overview}

\subsubsection{ALI Commentary to MPC \S\ 5}

\begin{enumerate}
    \item \textbf{Attempt}, \textbf{solicitation}, and \textbf{conspiracy} 
    consist of conduct meant to culminate in a substantive offense but has 
    failed or not yet culminated.
    \item Deterrence is a minor goal, but others include:
    \begin{enumerate}
        \item Provide a legal basis for law enforcement intervention.
        \item Deal with people who are disposed towards criminal activity.
        \item If the actor's failure is purely accidental, exculpation ``would 
        shock the common sense of justice.''\footnote{Casebook p. 730.}
    \end{enumerate}
    \item Most jurisdictions treat inchoate offenses as distinct from the 
    crimes toward which they tend.
    \item The MPC aims to prohibit an act that is a ``substantial step'' 
    toward the offense.
    \item This category of offenses relies heavily on judicial and law 
    enforcement discretion.
    \item Law aims to protect innocent conduct while preventing crimes in 
    progress.
\end{enumerate}

\subsection{Attempt}

\subsubsection{General principles}

\paragraph{Robbins, ``Double Inchoate Crimes''}

\begin{enumerate}
    \item Many jurisdictions have a few specific attempt rules alongside a 
    general attempt statute.
    \item Purpose is not deterrence but rather to give law enforcement a basis 
    for intervention.
    \item There are two varieties of criminal attempt:
    \begin{enumerate}
        \item \emph{Incomplete}: Actor is interrupted.
        \item \emph{Complete}: Actor does every act planned but fails to cause 
        the intended result (e.g., shoots and misses).
    \end{enumerate}
    \item If the basis for attempt law is to support law enforcement 
    intervention, what is the basis for punishing completed attempts?
    \item Do attempts cause social harm?
\end{enumerate}

\paragraph{Ashworth, ``Criminal Attempts and the Role of Resulting Harm under 
the Code, and in the Common Law}

\begin{enumerate}
    \item Prevention is the main reason for punishing preliminary steps on the 
    way to causing harm.
    \item What has the incomplete attempter done that is wrong?
    \begin{enumerate}
        \item Retributivists: it tends to ``restore an order of 
        fairness.''\footnote{Casebook p. 734.} \textbf{Harm-based} 
        retributivism is inapplicable unless the definition of harm is 
        broadened to include a presumed apprehension of fear of attempters. 
        \textbf{Intent-based} retributivism holds individuals liable for their 
        intentions, and \textbf{belief-based} retributivism for their belief 
        that what they were doing was wrong.
    \end{enumerate}
    \item What about the complete offender?
    \begin{enumerate}
        \item Harm-based retributivism: the apprehension of fear can provide a 
        basis for punishment.
        \item Intent-based retributivism: ``no relevant moral difference'' 
        between a completed attempt and a successful crime.\footnote{Casebook 
        p.  735.}
        \item Consequentialism: the effect of punishment must outweigh its 
        consequences. Complete attempters show clear propensities for causing 
        harm, so punishment is called for.
    \end{enumerate}
    \item George Fletcher's ``two patterns of criminality'':\footnote{Casebook 
    p. 736.}
    \begin{enumerate}
        \item ``Objectivist'': an act is criminal if a neutral third party 
        could recognize the criminality of the actor's conduct.
        \item ``Subjectivist'': the actor's intentions create criminality.
    \end{enumerate}
\end{enumerate}

\paragraph{ALI Comment to MPC 5.05}

\begin{enumerate}
    \item Should attempts be treated as lesser offenses than successfully 
    completed crimes?
    \item One common law formula fixed the punishment for attempt at half of 
    the maximum for the completed crime, or 10--50 years for crimes punishable 
    by death or life in prison.
    \item Traditionally, criminal attempts were punished less severely than 
    completed offenses, even if the only difference was bad luck. ``...the 
    reward for failing, no matter how hard you try to succeed or how close you 
    come, is lesser punishment.''\footnote{Sanford H. Kadish, Casebook p. 
    737.}
    \item MPC: punishment for attempt, solicitation, and conspiracy is 
    determined ``by the gravity of the most serious offense that is its 
    object.'' The completion or failure of the plan shouldn't matter because 
    there is little deterrent force.
    \item Should an attempt be treated as a less serious offense than the 
    target crime?
\end{enumerate}

\subsubsection{\emph{Mens Rea}}

\paragraph{Attempted murder and intent: \emph{People v. Gentry}}

\begin{enumerate}
    \item Gentry had spilled or poured gasoline on his girlfriend which then 
    accidentally ignited. The jury convicted him of attempted murder.
    \item The jury instructions included all four culpable mental states as 
    possible components of murder. Gentry argued on appeal that murder 
    requires specific intent.
    \item The appellate court agreed, holding that both attempted murder and 
    murder require specific intent to kill. Knowledge is insufficient. 
    Reversed.
    \item Criminal law involves two ``intents'': intentional conduct and 
    intent to commit the completed offense. They often merge, but they would 
    be separate if, for instance, an actor shot a gun merely to scare the 
    victim but accidentally killed him. Both intents must be 
    proven.\footnote{Casebook p. 740.} [Would transferred intent apply?]
    \item Dressler on applying MPC \S\ 5.01:\footnote{Casebook p. 741 n. 4.}
    \begin{enumerate}
        \item Was it a complete or incomplete attempt?
        \item Was the target crime an offense (e.g., murder) or conduct (e.g., 
        drunk driving) crime?
        \item 1(a) and 1(b) apply to complete attempts. 1(c) applies to 
        complete attempts.
    \end{enumerate}
\end{enumerate}

\paragraph{Attempted felony murder: \emph{Bruce v. State}}

\begin{enumerate}
    \item Can there be attempted felony murder?
    \item Bruce entered the victim's shoe store with a loaded gun and demanded 
    money from the cash register. The victim ducked out of the way and Bruce 
    shot him, causing injury but not death.
    \item The trial court convicted Bruce of attempted first degree felony 
    murder.
    \item The appellate court held that criminal attempt requires specific 
    intent to commit a particular offense. Felony murder, however, requires no 
    specific intent to kill. Thus, there can be no attempted felony murder.  
    Reversed.
    \item (Most states agree, but Florida does not.\footnote{Casebook p. 
    743.})
    \item Can you have attempted statutory rape? Under the MPC the answer is 
    yes. \textbf{To be guilty of attempt, the actor must have acted with the 
    mental state required for the target offense.} Statutory rape is a strict 
    liability offense, i.e., no particular mental state is required. Since 
    mistake of age is irrelevant for the target offense, it is likewise 
    irrelevant for the attempt.\footnote{Casebook p. 745.}
\end{enumerate}

\subsubsection{\emph{Actus Reus}}

\paragraph{\emph{United States v. Mandujano}}

\begin{enumerate}
    \item A series of attempts to define attempt.
    \item \emph{United States v. Noreikis}: the distinction between 
    preparation and attempt ``is one incapable of being formulated into a hard 
    and fast rule.''\footnote{Casebook p. 746.}
    \item \emph{United States v. Coplon}: attempt is when ``he has done all 
    that it is within his power to do, but has been prevented by intervention 
    from outside'' (Learned Hand).
    \begin{enumerate}
        \item \emph{Locus poenitentiae}: ``place of repentance.''
    \end{enumerate}
    \item \emph{Mims v. United States}, relying on a test from \emph{People v.  
    Buffum}: an ``appreciable fragment'' must have been committed, it must be 
    in progress such that it will be completed unless interrupted, and it must 
    not be equivocal.
    \item Others (from the case notes):
    \begin{enumerate}
        \item \emph{United States v. Oviedo}: attempt exist if the objective 
        acts, ``without any reliance on the \emph{mens rea}, mark the 
        defendant's conduct as criminal in nature.''\footnote{Casebook p. 
        747.}
        \item \emph{Stokes v. State}: if the ``design of a person to commit a 
        crime is clearly shown, slight acts done in furtherance of this design 
        will constitute an attempt.''
        \item \emph{People v. Luna}: if intent is ``clearly shown,'' any act 
        toward commission constitutes attempt.
        \item Sayre, ``Criminal Attempts'': the more serious the crime, the 
        ``further back'' in the series of preliminary acts should the law look 
        for acts constituting attempt.
        \item Enker, ``Impossibility in Criminal Attempts---Legality and the 
        Legal Process'': courts must weigh several factors, including (1) 
        whether the act is ``sufficiently close to the substantive crime,'' 
        (2) whether the actor's conduct makes one ``reasonably certain that he 
        is firmly committed to a specific illegal venture,'' and (3) whether 
        ``the act is sufficiently unambiguous to demonstrate the actor's 
        illegal intent.''\footnote{Casebook p. 748.}
    \end{enumerate}
\end{enumerate}

\paragraph{Locus poenitentiae: \emph{Commonwealth v. Peaslee}}

\begin{enumerate}
    \item The defendant had prepared to burn down a building. He asked one of 
    employees to start the fire, and the employee refused. Later, the two of 
    them drove toward the building to be burned, but turned back a quarter of 
    a mile away.
    \item Were the defendant's actions ``near enough to the accomplishment of 
    the substantive offense to be punishable''?\footnote{Casebook p. 750}
    \item If preparation ``comes very near'' to the completed act, it can be 
    punished as attempt. But in this case, preparation to set the fire without 
    any intent to actually light it is ``too remote.''
\end{enumerate}

\paragraph{\emph{People v. Rizzo}}

\begin{enumerate}
    \item Defendant and three others drove around looking for a man they 
    intended to rob. When the defendant jumped out of the car to look for the 
    man, all four were arrested. It turned out that the person they intended 
    to rob was nowhere nearby.
    \item The court held that there cannot be an attempt if there is no chance 
    of success. ``...these defendants had planned to commit a crime and were 
    looking around the city for an opportunity to commit it, but the 
    opportunity fortunately never came.''\footnote{Casebook p. 754.}
\end{enumerate}

\paragraph{\emph{People v. Miller}}

\begin{enumerate}
    \item The defendant had earlier threatened to kill Albert Jeans. Later 
    that day, he went, carrying a loaded rifle, to a field where Jeans and the 
    constable were planting hops. He surrendered his gun to the constable.
    \item The court cites \emph{Stokes} (with clear intent, any slight act 
    done in furtherance constitutes attempt). But the \emph{Stokes} test, the 
    court held, ``still presupposes some direct act or movement in the 
    execution of the design.''\footnote{Casebook p. 756.} As long as the actor 
    remains equivocal, there can be no attempt.
\end{enumerate}

% \paragraph{\emph{State v. Reeves}}
% 
% \begin{enumerate}
%     \item % TODO
% \end{enumerate}
% 
% \subsubsection{Special defenses}
% 
% \subsubsection{\emph{People v. Thousand}}
% 
% \begin{enumerate}
%     \item % TODO
% \end{enumerate}
% 
\subsubsection{Abandonment: \emph{Commonwealth v. McCloskey}}

\begin{enumerate}
    \item Defendant prepared to escape from prison. He began to escape, 
    including cutting barbed wire, but changed his mind.
    \item The trial court found him guilty of attempted prison breach.
    \item The Supreme Court of PA reversed on the grounds that the defendant 
    had ``not yet attempted the act.''\footnote{Casebook p. 787.}
    \item Judge Cercone, concurring, agreed with the outcome, but argued that 
    the basis should be the defense of \textbf{abandonment}. Otherwise, the 
    prison guards would not have been able to stop the defendant's escape 
    until he was scaling the prison walls.
    \begin{enumerate}
        \item The PA legislature substantially adoped the MPC, including \S\ 
        5.01. He argued that the court had long ago adoped abandonment as an 
        affirmative defense. And if it hadn't, it should 
        have.\footnote{Casebook p. 788.}
    \end{enumerate}
\end{enumerate}

\subsection{Assault}

\begin{enumerate}
    \item Mayhem: under common law, it consisted of injury ``impairing the 
    victim's ability to defend himself or to annoy his 
    adversary.''\footnote{Casebook p. 790.}
    \item Battery: any offensive and unlawful contact.
    \item Assault was originally just the attempt to commit battery. It 
    required stricted proximity than ordinary attempt. It evolved to include 
    menacing and actual attempts, as well as conditional assaults (i.e., 
    threats).
    \item The MPC removed the common law categories and implemented a single 
    definition under \S\ 211.1. It consolidates assault and battery and 
    removes the increased proximity requirement.
\end{enumerate}

\subsection{Solicitation}

\subsubsection{Defining solicitation: \emph{State v. Mann}}

\begin{enumerate}
    \item ``Solicitation involves the asking, enticing, inducing, or 
    counselling of another to commit a crime.''\footnote{Casebook p. 792.}
    \item ``...the solicitor is morally more culpable than a conspirator...''
    \item \emph{Merger}: ''The offense of solicitation merges into the crime 
    solicited if the latter attempt is committed or attempted by the solicited 
    party.''\footnote{Casebook p. 793.}
\end{enumerate}

\subsubsection{\emph{State v. Cotton}}

\begin{enumerate}
    \item While in prison, the defendant wrote letters to his wife asking her 
    to help prevent his step-daughter from testifying against him. His 
    cellmate covertly removed the letters from their envelopes and turned them 
    over to the authorities.
    \item The defendant was convicted of two counts of criminal solicitation.
    \item The appellate court noted that the New Mexico state legislature 
    explicitly omitted MPC \S\ 5.02(2), which criminalizes solicitations that 
    fail to reach the intended recipient. The court reasoned that this 
    omission indicates the legislature's intent to require actual 
    communication for solicitation to be accomplished.
    \item Reversed.
\end{enumerate}

\subsection{Conspiracy}

\begin{enumerate}
    \item Justice Holmes: conspiracy is ``a partnership in criminal 
    purposes.''\footnote{\emph{United States v. Kissel}, 218 U.S. 601 (1910).} 
    Partnership requires an agreement prior to the criminal activity.
\end{enumerate}

\subsubsection{General principles}

\paragraph{Kerman, \emph{Orange is the New Black: My Year in a Women's 
Prison}}

\begin{enumerate}
    \item Kerman was indicted on conspiracy charges for drug smuggling and 
    money laundering. She pleaded guilty and received a 30-month sentence.
\end{enumerate}

\paragraph{\emph{People v. Carter}}

\begin{enumerate}
    \item ```The gist of the offense of criminal conspiracy lies in the 
    unlawful agreement.' The crime is complete upon the formation of the 
    agreement...''\footnote{Casebook p. 797.}
    \item Some jurisdictions require an overt act, but the threshold is low.
    \item The specific intent is twofold: (1) intent to combine with others 
    and (2) intent to accomplish the illegal objective.
    \item Conspiracy fills the gap created when attempt is too narrowly 
    conceived.
    \item Often used to fight organized crime.
    \item ``...collective criminal agreement---partnership in crime---presents 
    a greater potential threat to the public than individual 
    delicts.''\footnote{Casebook p. 798.}
    \item The \emph{Callahan} court argued that conspiratorial groups are 
    dangerous because they can lead to criminal activity beyond the original 
    intended crime. Others have argued that conspiracies are less dangerous 
    because plans are more likely to leak or that some conspirators' doubts 
    will influence other participants.
    \item Katyal: groups polarize toward extremes and are thus more 
    dangerous.\footnote{Casebook p. 799.}
    \item Acts that are immoral but not criminal can be punished with 
    conspiracy charges.\footnote{Casebook p. 799--800.}
    \item Spectrum of inchoate offenses: solicitation < conspiracy < attempt.
    \item Is ``attempted conspiracy'' cognizable or is it the same as 
    solicitation?
\end{enumerate}

\paragraph{\emph{Pinkerton v. United States}}

\begin{enumerate}
    \item Two brothers were found guilty of conspiracy and multiple 
    substantive counts of violating the Internal Revenue Code. Daniel did not 
    participate in the substantive crimes. The question is whether 
    participation in the conspiracy is enough to find the conspirator guilty 
    of the substantive crimes.
    \item Daniel argued that direct participation was necessary to find guilt 
    on the substantive counts.
    \item The Supreme Court disagreed, holding that ``an overt act of one 
    partner may be the act of all without any new agreement specifically 
    directed to that act.''\footnote{Casebook p. 802.}
    \item Judge Rutledge, dissenting: such vicarious liability is appropriate 
    in civil but not criminal contexts. He finds it ``dangerous'' but does not 
    explain why.
    \item Accomplice liability ≠ conspiracy liability.
\end{enumerate}

\subsubsection{\emph{Mens rea}}

\paragraph{\emph{People v. Swain}}

\begin{enumerate}
    \item Swain and Chatman were both convicted of conspiracy to commit murder 
    and other crimes ``stemming from the drive-by shooting death of a 
    15-year-old boy.''\footnote{Casebook p. 806.}
    \item In jail, Swain boasted about being a good shot. But at trial, he 
    testified that he was not in the van at the time of the shooting.
    \item Chatman claimed the original plan was to steal the decedent's car. 
    He claimed he fired shots in self defense.
    \item The jury found Chatman guilty of second-degree murder and 
    conspiracy.  It found Swain guilty of conspiracy but not murder.
    \item The question on appeal was whether conspiracy to commit murder 
    requires express malice or whether it is possible to conspire to commit 
    implied malice murder.
    \item Implied malice murder does not require an intent to kill.
    \item California recognizes three kinds of second-degree 
    murder:\footnote{Casebook p. 808.}
    \begin{enumerate}
        \item Unpremeditated with express malice.
        \item Implied malice (i.e., depraved heart).
        \item Felony murder.
    \end{enumerate}
    \item The charge here was implied malice murder, which does not require a 
    specific intent to kill.
    \item Conspiracy requires specific intent to commit the elements of the 
    target crime, it is not possible to conspire to commit a crime that does 
    not require specific intent.
    \item Reversed.
\end{enumerate}

\paragraph{\emph{People v. Lauria}}

\begin{enumerate}
    \item Lauria operated an answering service. He knew many of his customers 
    used it for prostitution.
    \item Lauria was indicted for conspiracy to commit prostitution. The trial 
    court set aside the indictment for lack of reasonable or probable cause.
    \item The appellate court asked, ``[u]nder what circumstances does a 
    supplier become part of a conspiracy to further an illegal enterprise by 
    furnishing goods or services which he knows are to be used by the buyer 
    for criminal purposes?''\footnote{Casebook p. 810.}
    \item Two Supreme Court cases addressed this question:
    \begin{enumerate}
        \item \emph{Falcone}: a seller of sugar, yeast, and cans was not 
        guilty in a moonshining conspiracy, even though it knew of the 
        criminal activity.
        \item \emph{Direct Sales}: a morphine distributor was found guilty of 
        conspiracy for selling 300 times the normal requirement of the drug to 
        a physician.  \item The rule is that intent to ``further, promote, and 
        cooperate'' in the criminal activity must be present for the actor to 
        be guilty of conspiracy.
    \end{enumerate}
    \item The appellate court identified three areas where intent can be 
    inferred:
    \begin{enumerate}
        \item When the seller of goods has a stake in the criminal activity 
        (e.g., renting a room at inflated prices to a prostitute).
        \item When there is no legitimate use for the goods or services (e.g., 
        supplying horse-racing information by wire).
        \item When the volume of business with the buyer is ``grossly 
        disproportionate to any legitimate demand.''
    \end{enumerate}
    \item The court developed a two-part rule for establishing the intent of a 
    supplier: (1) when he overtly intends to participate in criminal activity, 
    or (2) when intent can be inferred on (a) his interest in the activity or 
    (b) the seriousness of the crime.\footnote{Casebook p. 813.}
    \item Inferences of intent do not apply to misdemeanors.
    \item The conspiracy charges against Lauria do not stand because he was 
    only charged with furthering a misdemeanor.
\end{enumerate}

\subsubsection{\emph{Actus Reus}}

\paragraph{Goldstein, ``Conspiracy to Defraud the United States''}

\begin{enumerate}
    \item The conspiratorial agreement is a ``theoretical construct.'' By 
    calling it an act, ``courts foster the already elaborate illusion that 
    conspiracy reaches actual, not potential, harm.''\footnote{Casebook p. 
    816.}
    \item Juries can find conspiracy ``on less evidence than might otherwise 
    be required.''
\end{enumerate}

\paragraph{Inferring Conspiracy: \emph{Commonwealth v. Azim}}

\begin{enumerate}
    \item Azim was carrying two men in his car. He stopped the car for them to 
    assault and rob another man. He was convicted of assault, robbery, and 
    conspiracy.
    \item On appeal, Azim argued that ``because his conspiracy conviction was 
    not suported by sufficient evidence against him, the charges of assault 
    and robbery must also fail.''\footnote{Casebook p. 818.}
    \item In \emph{Volk}, the court held that conspiracy could ``be 
    inferentially established''---i.e., the agreement need not be explicit.
    \item The court held that there was sufficient evidence of conspiracy, and 
    because conspiracy was established, Azim was also guilty of the criminal
    acts of his co-conspirators.
\end{enumerate}

\paragraph{\emph{Commonwealth v. Cook}}

\begin{enumerate}
    \item Cook was convicted of conspiracy to commit rape. On appeal, he 
    argued that there was insufficient evidence to establish a conspiracy.
    \item The victim visited a housing project to visit her boyfriend, who 
    turned out not to be home. The defendant and his brother invited her to 
    hang out on their porch. After forty-five minutes, Cook's brother 
    suggested that the three of them go to a convenience store to buy 
    cigarettes. On a path along the way, the victim slipped, and Cook's 
    brother raped her.
    \item The Court found no conspiracy because there was no evidence of a 
    preconceived plan, the conduct up to the crime was in the open, and the 
    criminal action was spontaneous.
    \item The Court also rejected the Commonwealth's argument that Cook's 
    complicity as an accomplice made him a co-conspirator. There was no prior 
    agreement to commit a criminal act.
    \item Reversed.
\end{enumerate}

\subsubsection{Scope of the Agreement}

\paragraph{ALI Commentary to MPC \S\ 5.03}

\begin{enumerate}
    \item ``The scope problem'': ``Has a retailer conspired with the smugglers 
    to import the narcotics? Has a prostitute conspired with the leaders of 
    the vice ring to commit the acts of prostitution of each other prostitute 
    who is controlled by the ring?
\end{enumerate}

\paragraph{\emph{Kilgore v. State}}

\begin{enumerate}
    \item Facts:
        \item February 6, 1981: David Oldaker and Greg Benton attempted to kill 
        Roger Norman at the request of Tom Harden.
        \item June 8, 1981: Kilgore (the defendant) and Lee Berry allegedly shot 
        Norman in the back while he was driving but did not kill him.
        \item July 8, 1981: Kilgore and Bob Price allegedly successfully killed 
        Norman. Tom Carden allegedly had given them money.
    \item Kilgore was convicted of murder.
    \item At trial, Oldaker testified that Benton told him that the man who 
    wanted Norman killed was Tom Carden. The Supreme Court of Georgia found 
    that this testimony was admissible only if Oldaker, Benton, and Kilgore 
    were co-conspirators. This was a ``wheel'' conspiracy in which Tom Carden 
    communicated with each of the ``spokes'' individually---however, there was 
    no evidence suggesting that Oldaker and Benton had any contact with 
    Kilgore. Thus, the three were not co-conspirators and Oldaker's testimony 
    was inadmissible.
    \item (It's not clear whether the court upheld the conviction after it 
    determined that the testimony was inadmissible hearsay.)
\end{enumerate}

\paragraph{\emph{Braverman v. United States}}

\begin{enumerate}
    \item Defendant moonshiners were indicted on seven counts of conspiracy to 
    violate U.S. internal revenue laws. The defendants moved to require the 
    government to choose one of the seven counts, arguing that the evidence 
    could not prove seven different conspiracies. The government argued that 
    the indictment was proper because there were seven different criminal 
    goals of a single ongoing conspiracy.
    \item The jury returned a guilty verdict on each of the seven counts. The 
    Sixth Circuit affirmed.
    \item The government conceded that there was a single agreement among the 
    conspirators. The Supreme Court held that a single agreement can only be 
    punished as a single conspiracy, even if it aimed at several criminal 
    offenses. Reversed.
\end{enumerate}

\subsubsection{Defenses}

\paragraph{Wharton's Rule: \emph{Ianelli v. United States}}

\begin{enumerate}
    \item The eight petitioners were convicted of conspiring to violate and 
    violating a federal gambling statute which criminalized gambling 
    businesses involving at least five people.
    \item The Third Circuit affirmed.
    \item ``Wharton's Rule'' prevents conspiracy charges for crimes requiring 
    multiple participants.\footnote{Casebook p. 839.} The classic ``Wharton's 
    Rule'' offenses are adultery, incest, bigamy, and 
    duelling.\footnote{Casebook p. 841.}
    \item Courts were divided on whether the rule requires the conspiracy 
    charge to be dismissed before trial or whether a prosecutor can charge 
    both and instruct the jury that conviction for the offense precludes 
    conviction for the conspiracy.
    \item Courts were also divided on whether Wharton's Rule applies when 
    there are more people involved than are necessary to commit the offense.
    \item Wharton's Rule was intended to apply to offenses that ``require 
    concerted criminal activity.... absent legislative intent to the contrary, 
    the Rule supports a presumption that the two merge when the substantive 
    offense is proved.''\footnote{Casebook p. 842.} Here, however, the legislative 
    history of the gambling statute shows a clear intent to target organized crime, so it makes sense 
    not to merge the conspiracy and the substantive offense.
\end{enumerate}

\paragraph{\emph{Gebardi v. United States}}

Under the common law, acquiescence to activity in which acquiescence does not 
estasblish guilt is insufficient to establish conspiracy for either 
conspirator. Under the MPC, one party can be guilty of conspiracy if he 
believes the other is able to form a conspiratorial relationship.

\begin{enumerate}
    \item Gebardi was convicted of conspiracy to violate the Mann Act 
    (transporting ``any woman or girl for the purpose of prostitution or 
    debauchery, or for any other immoral purpose''\footnote{Casebook p. 844}).
    \item The Mann Act requires more than the woman's ``mere acquiescence.'' 
    She must ``aid or assist'' her own transport.
    \item ``As there is no proof that the man conspired with anyone else to 
    bring about the transportation, the convictions of both petitioners [for 
    conspiracy] must be reversed.''\footnote{Casebook p. 845.}
    \item (Under MPC \S\ 5.04(1), the court would likely find the man 
    guilty of conspiracy.)
\end{enumerate}

\paragraph{Withdrawal from Conspiracy: \emph{People v. Sconce}}

\begin{enumerate}
    \item Sconce offered Garcia \$10,000 to kill Estephan. Garcia offered 
    \$5,000 to Dutton to perform the killing. Three weeks later, Sconce told 
    Garcia to call it off. Meanwhile, Dutton had been arrested on a parole 
    violation.
    \item The trial court set aside the information relating to conspiracy 
    because Sconce had withdrawn.
    \item The appellate court held that withdrawal is a complete defense to 
    conspiracy only if it occurs before an overt act. Here, Sconce had already 
    completed the conspiracy, so his withdrawal only precluded liability for 
    future acts of his co-conspirators.
    \item ``...withdrawal from the conspiracy is not a defense to the 
    completed crime of conspiracy...''\footnote{Casebook p. 847.}
    \item Reversed.
\end{enumerate}

