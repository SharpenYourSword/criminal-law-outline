\section{Homicide}

\begin{enumerate}
    \item Homicide is a neutral term. It is not necessarily a crime.
\end{enumerate}

\subsection{Intentional Killing}

\subsubsection{Murder}

\begin{enumerate}
    \item Premiditation is usually the distinction between first- and second-degree murder.
    \item The MPC does not distinguish between first- and second-degree. Culpability is evaluated at sentencing.
    \item There are four common law definitions of murder:\footnote{Casebook p. 236.}
    \begin{enumerate}
        \item Intent to kill.
        \item Intent to cause grevious bodily harm.
        \item ``Depraved-heart murder'' (i.e., extreme recklessness regarding homicidal risk).
        \item Intent to commit a felony.
    \end{enumerate}
    \item In 1794, Pennsylvania introduced the idea of degrees.
    \item The MPC recognizes three kinds of criminal homicide: murder, manslaughter, and negligent homicide.
\end{enumerate}

\paragraph{Premeditation and deliberation: \emph{State v. Guthrie}}

\begin{enumerate}
    \item What facts establish premeditation and deliberation?
    \item The defendant stabbed and killed a coworker after the coworker taunted him and snapped him in the nose with a towel. The trial court found him guilty of first degree murder. The defendant argues that the trial court's instructions to the jury were improper because ``the terms wilful, deliberate, and premeditated were equated with a mere intent to kill.'' The appellate court agreed that ``premeditation'' cannot be synonymous with intent---rather, it must be long enough for the defendant to be ``fully conscious of what he intended.'' Reversed and remanded for a new trial.
\end{enumerate}

\paragraph{\emph{Midgett v. State}}

\begin{enumerate}
    \item The defendant repeatedly abused his young son, who died from the injuries. The trial court found him guilty of first degree murder, which required premeditation and deliberation. The defendant argues that there was no premeditation, and the Supreme Court of Arkansas agreed. The dissent argued that symptoms of malnourishment indicated starvation, but the majority argued that the evidence did not prove starvation.
    \item Shortly after this case, the Arkansas legislature amended its criminal code to broaden first degree murder to include ``extreme indifference to the value of human life'' of people fourteen years old or younger.
\end{enumerate}

\paragraph{\emph{State v. Forrest}}

\begin{enumerate}
    \item The defendant shot and killed his terminally ill father in the hospital. The trial court convicted him of first degree murder. The defendant argued that there was no premeditation or deliberation, and therefore no evidence to prove first-degree murder. The appellate court upheld the conviction, noting that premeditation must be proved by circumstantial evidence, including provocation from the victim, the defendant's conduct and statements, ill will between the parties, lethal blows after the victim was rendered helpless, and evidence of an especially brutal killing. In this case, the court found that the victim was laying helpless and did nothing to provoke the defendant, and that the defendant had earlier made statements about ``putting his father out of his misery.''
    \item Premeditation/deliberation: in cold blood.
    \item Provocation: in hot blood.
\end{enumerate}

\subsubsection{Manslaughter}

\begin{enumerate}
    \item Provocation can mitigate murder to manslaughter. Common law and the MPC diverge on what constitutes provocation.
    \item Common law rule of provocation:\footnote{Casebook p. 267.}
    \begin{enumerate}
        \item There must have been adequate provocation.
        \item The killing must have been in the heat of passion.
        \item There must not have been a cooling off period.
        \item There must have been a causal connection between the provocation, the passion, and the fatal act.
        \item (The sorts of provocations that courts have allowed as defenses at common law are the sort of actions that have offended traditional notions of a man's honor---e.g., catching a wife in the act with another man.)
    \end{enumerate}
    \item The MPC replaces the provocation defense with an ``emotional disturbance'' test.\footnote{MPC § 210.3(1)(b) at Casebook p. 1000.} Under the MPC, homicide constitutes manslaughter when:
    \begin{enumerate}
        \item The homicide is committed ``under the influence of extreme mental or emotional disturbance.''
        \item There is a reasonable explanation or excuse for the mental or emotional disturbance under the circumstances as the defendant believed them to be. In other words, the excuse is reasonable if a reasonable person in the defendant's situation would have been disturbed.
        \item (The cooling off period is not an issue under the MPC definition.)
    \end{enumerate}
    \item For a critique of the MPC's rules of provocation in terms of gender discrimination, see Victoria Nourse, "Passion's Progress."
    \item Words are usually not enough to constitute provocation. Words can be sufficient if they accompany a threat of intent and ability to cause bodily harm.\footnote{Casebook pp. 267--68.} Some jurisdictions allow ``informative words'' (e.g, ``your husband is having an affair with ...'') to constitute provocation.
\end{enumerate}

\paragraph{Verbal Provocation: \emph{Girouard v. State}}

\begin{enumerate}
    \item Are words enough to satisfy the provocation requirement for reducing murder to manslaughter?
    \item The defendant stabbed and killed his wife after she taunted him relentlessly. The trial court, in a bench trial, convicted him of second degree murder. He argued on appeal that the rule of provocation should be expanded to include verbal provocation. The appellate court relied on the rule that for provocation to mitigate a charge of murder, it must be ``calculated to inflame the passion of a reasonable man and tend to cause him to act for the moment from passion to reason.'' The standard is objective. The court found that words can constitute adequate provocation if they accompany intent and ability to cause bodily harm. That was not the case in this scenario, however. The court upheld the second degree murder conviction.
\end{enumerate}

\paragraph{\emph{People v. Casassa}}

\begin{enumerate}
    \item The defendant stabbed and killed his neighbor out of jealousy. The trial court found him guilty of second degree murder. The defendant argued he was acting under ``extreme emotional disturbance,'' which would reduce the charge to manslaughter. The appellate court reasoned that the emotional disturbance must meet an objectively reasonable standard. In this case, the disturbance was a result of the defendant's unique mental state---i.e., a reasonable person would not have been so emotionally disturbed under the circumstances.
\end{enumerate}

\subsection{Unintentional Killing}

\paragraph{\emph{People v. Knoller}}

\paragraph{\emph{State v. Hernandez}}

\paragraph{\emph{State v. Williams}}

\subsection{Felony Murder}

\paragraph{\emph{People v. Fuller}}

\paragraph{\emph{People v. Howard}}

\paragraph{\emph{People v. Smith}}

\paragraph{\emph{State v. Sophophone}}

\paragraph{Misdemeanor Manslaughter}

