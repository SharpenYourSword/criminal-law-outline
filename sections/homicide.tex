\section{Homicide}

\begin{enumerate}
    \item Homicide is a neutral term. It is not necessarily a crime.
\end{enumerate}

\subsection{Common Law vs. Model Penal Code}

\subsubsection{Common Law}

\begin{enumerate}
    \item Murder
    \begin{enumerate}
        \item First-degree: premeditated/deliberated.
        \item Second-degree:
        \begin{enumerate}
            \item Intent to kill without premeditation.\footnote{E.g., the case where the victim snapped the defendant in the nose with a towel.}
            \item ``Depraved heart'' killing/implied malice (unintentional--e.g., gross recklessness).
        \end{enumerate}
        \item Both degrees require malice aforethought:
        \begin{enumerate}
            \item Intent to kill.
            \item Intent to cause grievous bodily injury.
            \item Depraved or abandoned heart.
            \item Intent to commit a felony.
        \end{enumerate}
    \end{enumerate}
    \item Manslaughter
    \begin{enumerate}
        \item Voluntary: requires provocation.
        \item Involuntary: includes negligence and recklessness (unintentional).
    \end{enumerate}
\end{enumerate}

\subsubsection{Model Penal Code}

\begin{enumerate}
    \item Murder
    \begin{enumerate}
        \item With purpose.
        \item With knowledge.
        \item With recklessness (unintentional).
    \end{enumerate}
    \item Manslaughter
    \begin{enumerate}
        \item With recklessness (unintentional).
        \item Under extreme mental or emotional distress.
    \end{enumerate}
    \item Negligent homicide.
\end{enumerate}

\subsection{Intentional Killing}

\subsubsection{Murder}

\begin{enumerate}
    \item Premiditation is usually the distinction between first- and second-degree murder.
    \item The MPC does not distinguish between first- and second-degree. Culpability is evaluated at sentencing.
    \item There are four common law definitions of murder:\footnote{Casebook p. 236.}
    \begin{enumerate}
        \item Intent to kill.
        \item Intent to cause grevious bodily harm.
        \item ``Depraved-heart murder'' (i.e., extreme recklessness regarding homicidal risk).
        \item Intent to commit a felony.
    \end{enumerate}
    \item In 1794, Pennsylvania introduced the idea of degrees.
    \item The MPC recognizes three kinds of criminal homicide: murder, manslaughter, and negligent homicide.
\end{enumerate}

\paragraph{Premeditation and deliberation: \emph{State v. Guthrie}}

\begin{enumerate}
    \item What facts establish premeditation and deliberation?
    \item The defendant stabbed and killed a coworker after the coworker taunted him and snapped him in the nose with a towel. The trial court found him guilty of first degree murder. The defendant argues that the trial court's instructions to the jury were improper because ``the terms wilful, deliberate, and premeditated were equated with a mere intent to kill.'' The appellate court agreed that ``premeditation'' cannot be synonymous with intent---rather, it must be long enough for the defendant to be ``fully conscious of what he intended.'' Reversed and remanded for a new trial.
\end{enumerate}

\paragraph{\emph{Midgett v. State}}

\begin{enumerate}
    \item The defendant repeatedly abused his young son, who died from the injuries. The trial court found him guilty of first degree murder, which required premeditation and deliberation. The defendant argues that there was no premeditation, and the Supreme Court of Arkansas agreed. The dissent argued that symptoms of malnourishment indicated starvation, but the majority argued that the evidence did not prove starvation.
    \item Shortly after this case, the Arkansas legislature amended its criminal code to broaden first degree murder to include ``extreme indifference to the value of human life'' of people fourteen years old or younger.
\end{enumerate}

\paragraph{\emph{State v. Forrest}}

\begin{enumerate}
    \item The defendant shot and killed his terminally ill father in the hospital. The trial court convicted him of first degree murder. The defendant argued that there was no premeditation or deliberation, and therefore no evidence to prove first-degree murder. The appellate court upheld the conviction, noting that premeditation must be proved by circumstantial evidence, including provocation from the victim, the defendant's conduct and statements, ill will between the parties, lethal blows after the victim was rendered helpless, and evidence of an especially brutal killing. In this case, the court found that the victim was laying helpless and did nothing to provoke the defendant, and that the defendant had earlier made statements about ``putting his father out of his misery.''
    \item Premeditation/deliberation: in cold blood.
    \item Provocation: in hot blood.
\end{enumerate}

\subsubsection{Manslaughter}

\begin{enumerate}
    \item Provocation can mitigate murder to manslaughter. Common law and the MPC diverge on what constitutes provocation.
    \item Common law rule of provocation:\footnote{Casebook p. 267.}
    \begin{enumerate}
        \item There must have been adequate provocation.
        \item The killing must have been in the heat of passion.
        \item There must not have been a cooling off period.
        \item There must have been a causal connection between the provocation, the passion, and the fatal act.
        \item (The sorts of provocations that courts have allowed as defenses at common law are the sort of actions that have offended traditional notions of a man's honor---e.g., catching a wife in the act with another man.)
    \end{enumerate}
    \item The MPC replaces the provocation defense with an ``emotional disturbance'' test.\footnote{MPC § 210.3(1)(b) at Casebook p. 1000.} Under the MPC, homicide constitutes manslaughter when:
    \begin{enumerate}
        \item The homicide is committed ``under the influence of extreme mental or emotional disturbance.''
        \item There is a reasonable explanation or excuse for the mental or emotional disturbance under the circumstances as the defendant believed them to be. In other words, the excuse is reasonable if a reasonable person in the defendant's situation would have been disturbed.
        \item (The cooling off period is not an issue under the MPC definition.)
    \end{enumerate}
    \item For a critique of the MPC's rules of provocation in terms of gender discrimination, see Victoria Nourse, "Passion's Progress."
    \item Words are usually not enough to constitute provocation. Words can be sufficient if they accompany a threat of intent and ability to cause bodily harm.\footnote{Casebook pp. 267--68.} Some jurisdictions allow ``informative words'' (e.g, ``your husband is having an affair with ...'') to constitute provocation.
\end{enumerate}

\paragraph{Verbal Provocation: \emph{Girouard v. State}}

\begin{enumerate}
    \item Are words enough to satisfy the provocation requirement for reducing murder to manslaughter?
    \item The defendant stabbed and killed his wife after she taunted him relentlessly. The trial court, in a bench trial, convicted him of second degree murder. He argued on appeal that the rule of provocation should be expanded to include verbal provocation. The appellate court relied on the rule that for provocation to mitigate a charge of murder, it must be ``calculated to inflame the passion of a reasonable man and tend to cause him to act for the moment from passion to reason.'' The standard is objective. The court found that words can constitute adequate provocation if they accompany intent and ability to cause bodily harm. That was not the case in this scenario, however. The court upheld the second degree murder conviction.
\end{enumerate}

\paragraph{\emph{People v. Casassa}}

\begin{enumerate}
    \item The defendant stabbed and killed his neighbor out of jealousy. The trial court found him guilty of second degree murder. The defendant argued he was acting under ``extreme emotional disturbance,'' which would reduce the charge to manslaughter. The appellate court reasoned that the emotional disturbance must meet an objectively reasonable standard. In this case, the disturbance was a result of the defendant's unique mental state---i.e., a reasonable person would not have been so emotionally disturbed under the circumstances.
\end{enumerate}

\subsubsection{Unintentional Killing}

\begin{enumerate}
    \item Implied malice is required to prove murder from an unintentional killing.
    \item At common law, implied malice requires a ``depraved heart''---i.e., it involves acting with a conscious disregard for human life and conduct involving a high probability of death (see \emph{Knoller} below).
    \item The MPC does not use the depraved heart standard. Under the MPC, ordinary recklessness proves manslaughter. It proves murder when the actor's ``conscious disregard for the risk, under the circumstances, manifests extreme indifference to the value of human life.''\footnote{Casebook p. 303.} 
    \item At common law, intent to cause grievous bodily injury is sufficient to establish ``malice aforethought.'' The MPC does not adopt this approach---instead, it handles such cases under the standard of extreme recklessness.\footnote{Casebook p. 304.}
\end{enumerate}

\paragraph{\emph{Implied Malice: People v. Knoller}}

\begin{enumerate}
    \item Defendants came into possession of two large, aggressive Presa Canario dogs. They'd been warned repeatedly about the dogs' dangerously aggressive behavior. The dogs killed a woman in the hallway of the defendants' apartment building.
    \item The trial court took the position that a murder charge required conduct involving ``a high probability of resulting in the death of another.'' The jury found the defendants guilty. The court granted defendants' motion for a new trial on the grounds that Knoller did not know that her conduct involved a high probability of death. The appellate court reversed the order for a new trial, holding that the standard for second-degree murder should be ``conscious disregard of the risk of serious bodily injury to another,'' rather than a high probability of death.''
    \item The Supreme Court of California focused on the issue of implied malice as an element of murder. It uses two definitions of implied malice: (1) the \emph{Thomas} test: ``wanton disregard for human life, and (2) the \emph{Phillips} test: ``conscious disregard for human life.'' The tests articulate the same standard, but the court prefers the second for clarity.
    \item The Supreme Court reversed the appellate court, holding that implied malice requires awareness of a risk of death, not just serious bodily harm. It also held that the trial court erred in its interpretation of the test. The trial court held the awareness of a high probability of death to be a subjective perception, but it's actually an objective standard. The subjective component is that the defendant must have acted with ``conscious disregard for human life.'' (Not all jurisdictions require conscious disregard to establish implied malice.)
\end{enumerate}

\paragraph{\emph{State v. Hernandez}}

\begin{enumerate}
    \item The defendant killed a woman while he was driving drunk. The trial court convicted him of involuntary manslaughter. The issue on appeal is whether stickers and pins inside the car with ``drinking slogans''---``The more I drink the better I look,'' etc.---were admissible evidence to help establish the elements of involuntary manslaughter, which the relevant Missouri statute (deriving from the MPC) defines as (1) criminal negligence and (2) resulting death. Criminal negligence is the culpable failure to perceive a substantial and unjustifiable risk.
    \item The government introduced the slogans as evidence that the defendant was aware of the risk of driving drunk. But this was a prosecutorial error, because the statute only criminalizes negligence, and if the defendant was aware of the risk, he could not have acted negligently (though he may have acted recklessly). Therefore, the evidence was inadmissible in th state's attempt to prove the elements of involuntary manslaughter. (The prosecution probably could have proved second-degree murder based on recklessness.)
    \item The appellate court held that the slogans served only to illustrate the defendant's character. Reputation and character testimony were inadmissible here, and so the appellate court reversed the conviction.
    \item The dissent argued that at least three of the slogans indicated that alcohol can impair perception, and that the defendant therefore should have been aware of the substantial risks involved with drinking and driving. It also suggested that the defendant's intoxication might have prevented him from perceiving the risk.
\end{enumerate}

\paragraph{\emph{Omission: State v. Williams}}

\begin{enumerate}
    \item The defendants' infant child died when they failed to seek medical attention for a tooth infection that became a gangrenous abscess.
    \item At common law, involuntary manslaughter required gross negligence, not just ordinary negligence. Washington State law, however, only requires ordinary negligence.
    \item The trial court found the defendants guilty of involuntary manslaughter. The appellate court affirmed.
    \item At common law, the defendants likely would not have been convicted of manslaughter.
    \item The appellate court found negligence.
    \item Washington redrafted its code in 1975. Today, ``[c]riminal homicide convictions on the basis of ordinary negligence are nearly non-existent.''
    \item Is it appropriate to punish negligent homicide?
    \begin{enumerate}
        \item The utilitarian argument against punishment is that negligent actors cannot be deterred. The MPC drafters rejected this argument, arguing that the threat of deterrence encourages people to act with greater care.
        \item The retributivist argument is that an actor cannot be morally culpable for actions that he does not know he is taking. Stephen Garvey argues that culpability exists in the failure to exercise self-control ``over desires that influence the formation and awareness of one's beliefs.'' Jerome Hall argues that blame is sometimes appropriate in response to negligence, but punishment is not.\footnote{Casebook pp. 312--313.}
    \end{enumerate}
\end{enumerate}

\subsubsection{Felony Murder}

\begin{enumerate}
    \item At common law, any homicide committed while committing a felony was considered murder. The homicide was a strict liability offense.
    \item The proliferation of statutory (\emph{malum prohibitem} felonies has led the MPC to enumerate the specific felonies that can constitute the bases for felony murder.
\end{enumerate}

\paragraph{\emph{People v. Fuller}} 

\begin{enumerate}
    \item The defendant accidentally killed a driver while involved in a high speed car chase. He had been breaking into cars in a parking lot when the police noticed him, and a chase ensued. The Court of Appeal ruled that the trial court had erred in striking the first-degree murder count. The appellate court allowed the prosecution for first degree murder under the felony murder rule---but it notes that if it were ``starting from a clean slate,'' it would not allow the prosecution because the original felony, burglary, was not dangerous to human life.
\end{enumerate}

\paragraph{Roth and Sundby, ``The Felony-Murder Rule''}

\begin{enumerate}
    \item The US is the only western country that recognizes the felony murder rule.
    \item The rule is meant to (1) deter accidental killings during felonies as well as (2) the felonies themselves. On (1), how can you deter an unintentional act? On (2), there is doubt that stricter punishments deter serious crimes, and it makes more sense to punish the intended conduct (e.g., carrying a deadly weapon) rather than the unintendend killing.
    \item Transferred intent is not a valid justification for the felony murder rule because of the differences in \emph{mens rea} for the felony and for murder.
    \item Justifying a murder charge on a retributivist justification is a regression to the primitive ``evil mind'' theory of common law.
\end{enumerate}

\paragraph{Crump and Crump, ``In Defense of the Felony Murder Doctrine''}

\begin{enumerate}
    \item ``Felony murder reflects a social judgment'' that felonies involving killing are more serious than non-lethal felonies.
    \item The rule distinguishes crimes that cause death, thereby ``reinforcing the reverence for human life.''
    \item Core disagreement with Roth and Sundby: punishing negligent killings \emph{can} deter. Also, felons who killed intentionally might testify that the killings were accidental; the felony murder rule denies them this defense.
    \item A clear felony murder rule is less confusing to juries, so it leads to more consistent results. Also, by simplifying the questions involved, it makes administration more efficient.
\end{enumerate}

\paragraph{Tomkovicz, ``The Endurance of the Felony Murder Rule''}

\begin{enumerate}
    \item Restricting the felony murder rule to certain types of felonies enhances its fairness, helping the doctrine survive.
\end{enumerate}

\paragraph{\emph{People v. Howard}}

\begin{enumerate}
    \item The defendant was driving a stolen car without a rear license place. A chase ensued when police tried to pull him over. During the chase, the defendant hit and killed another driver. The trial jury convicted the defendant of murder. The appellate court affirmed, rejecting the defendant's claim that he could not be charged with second-degree murder because of California precedent rejecting the felony-murder rule for felonies that are not inherently dangerous.
    \item The California Supreme Court looked at the statute defining high-speed chases.\footnote{Section 2800.2; Casebook p. 329.} It noted that in 1996, the legislature significantly broadened the statutory definition of ``willful or wanton disregard for the safety of persons or property'' to include any flight from a police officer involving three traffic violations. It concluded that a violation of this statute ``is not, in the abstract, inherently dangerous to human life.'' Therefore, the prosecution cannot rely on the felony murder rule. Reversed.
    \item Brown, concurring and dissenting: this interpretation of the statute defies common sense. The conviction should be overturned, but only because the felony murder rule should be removed entirely.
    \item Baxter, dissenting: ``there is no doubt that the defendant committed exactly the reckless endangerment of human life forbidden by the statute.''
\end{enumerate}

\paragraph{\emph{People v. Smith}}

\begin{enumerate}
    \item Defendant was abusing her child, who accidentally fell, hit her head, and died of respiratory arrest. The trial court applied the felony-murder rule to convict her of second-degree murder.
    \item The purpose of the felony-murder rule is to deter negligent and accidental killings in the commission of felonies. If the felony is an integral part of the homicide, the felony-murder rule can serve no additional deterring function, and it prevents the consideration of malice aforethought.
    \item \emph{People v. Ireland}: ``We therefore hold that a second degree felony-murder instruction may not properly be given when it is based upon a felony which is an integral part of the homicide and which the evidence produced by the prosecution shows to be an offense included in fact within the offense charged.''\footnote{Casebook p. 335.}
    \item \emph{People v. Wilson}: The felony-murder rule cannot be applied to cases where the action would not be felonious but for the assault, and the assault is an integral part of the homicide.
    \item \emph{People v. Sears}: If assault is intended against one person but results in the accidental killing of another, the felony-murder rule should not apply (because it would carry harsher punishments than if the intended victim was killed, in which case the felony-murder rule would not not apply).
    \item \emph{People v. Burton}: The felony-murder rule can apply if the underlying violent action was committed with an ``independent felonious purpose.'' For instance, in an armed robbery case where an accidental killing results, the rule applies because the underlying purpose was to rob, not to assault.
    \item The felony murder rule does not apply in this case. Reversed.
\end{enumerate}

\paragraph{\emph{State v. Sophophone}}

\begin{enumerate}
    \item The defendant and three accomplices broke into a house. The police arrived and shot one of the accomplices. The defendant was charged, among other things, with felony murder. The trial court convicted him on all counts.
    \item Under Kansas state law, aggravated burglary counts as one of the inherently dangerous felonies that triggers the felony murder rule.
    \item The defendant argues that he was in custody at the time of his accomplice's death and cannot therefore be held liable.
    \item The question is ``whether the felony-murder rule should apply when the fatal act is performed by a non-felon.'' There are two approaches:
    \begin{enumerate}
        \item \textbf{Agency approach} (the majority rule): The rule does not apply when the person who causes the death is a non-felon. The killing was the result of actions contrary to the intentions of the felon.
        \item \textbf{Proximate causation approach}: The rule applies. A felon is responsible for the consequences of the actions he sets in motion.
    \end{enumerate}
    \item The court holds that the rule does not apply. ``...we believe that making one criminally responsible for the lawful acts of a law enforcement officer is not the intent of the felony-murder statute as it is currently written.''\footnote{Casebook p. 340.}
    \item Dissent: nothing in the statute requires the court to adopt the ``agency'' approach. ``This set of events could have very easily resulted in the death of a law enforcement officer, and in my opinion this is exactly the type of case the legislature had in mind when it adopted the felony-murder rule.''\footnote{Casebook p. 341.}
    \item \emph{Res gestae}: The felony-murder rule applies when a killing occurs during the commission (or attempted commission) of a felony. Most courts also apply it in the aftermath, e.g., during a getaway.
    \item There must also be causal relationship between the death and the felony. The cause must be proximate.
\end{enumerate}

\paragraph{Misdemeanor Manslaughter}

\begin{enumerate}
    \item ``An unintended homicide that occurs during the commission of an unlawful act not amounting to a felony constitutes common law involuntary manslaughter.''\footnote{Casebook p. 343}.
    \item Manslaughter convictions have been upheld in cases where the act is morally culpable but not technically criminal---e.g., someone attempted to commit suicide with a gun, someone attempted to intervene, and the one who intervened was accidentally shot and killed.
\end{enumerate}
