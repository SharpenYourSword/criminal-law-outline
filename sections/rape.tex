\section{Rape}

\subsection{Overview}

\begin{enumerate}
    \item 32\% of rapes were reported to law enforcement in 1994 and 1995.\footnote{Casebook p. 385.}
    \item 91\% of victims were female.\footnote{Casebook p. 386.} (MPC § 213.1 defines rape only as male-against-female.)
    \item 8\% of forcible rapes reported in 1995 turned out to be unfounded.\footnote{Casebook p. 387.}
    \item Susan Estrich: rape law exposes the sexism of the law.
    \item Two frameworks for understanding the crime of rape: (1) a crime of violence and (2) a crime against sexual autonomy.\footnote{Casebook p. 391.}
\end{enumerate}

\subsubsection{Annette Gordon Reed, \emph{Celia's Case}}

\begin{enumerate}
    \item A white male landowner bought Celia, a female slave, and treated her as a concubine. She eventually decided to stand up to his advances and ended up killing him by hitting him on the head. She stood trial and was put to death.
\end{enumerate}

\subsubsection{C. Vann Woodward, Review of \emph{Scottsboro: A Tragedy of the American South}}

\begin{enumerate}
    \item Two white women falsely accused the nine Scottsboro Boys of rape. All were convicted and sentenced. Eventually, all were exonerated.
\end{enumerate}

\subsubsection{NY Times Obituary for Ruth Schut}

\begin{enumerate}
    \item A note on the death of one of the two accusers of the Scottsboro Boys. She later recanted her accusations.
\end{enumerate}

\subsection{Forcible Rape}

\subsubsection{The Force Requirement: \emph{State v. Alston}}

\begin{enumerate}
    \item Alston and Brown were in a semi-abusive relationship. She broke it off. Then, one day, he coerced her to come to a friend's house. She did not forcibly resist his sexual advances. Later that day, she filed a police complaint.
    \item The trial court convicted the defendant of second degree rape.
    \item The Supreme Court of North Carolina here reasoned that the statutory definition of rape required intercourse to be (1) by force and (2) against the victim's will. The court found no evidence that the victim forcibly resisted. It overturned the conviction.
    \item Susan Estrich: The victim was not forced to engage in sex, but did so against her will. ``To say that there is no `force' in such a situation is to create a gulf between power and force, and to define the latter solely in schoolboy terms.''\footnote{Casebook p. 409}
    \item Vivian Berger: It's not clear that this was actually a rape. Overprotecting women ``risks enfeebling instead of empowering.''\footnote{Casebook p. 410.}
\end{enumerate}

\subsubsection{\emph{Rusk v. State}}

\begin{enumerate}
    \item The victim met the defendant at bar. She drove him home. She did not want to go into his apartment, but he took her car keys out of her ignition, and she followed him out of fear. They had sex and she did not resist.
    \item The trial court apparently found that there was insufficient evidence for a trier of fact to prove rape. The Court of Special Appeals of Maryland here affirmed.
    \item Judge Wilner, dissenting:
    \begin{enumerate}
        \item The court inappropriately substituted its judgement for the jury's.
        \item The court's reasoning requires the victim to either (1) resist, risking physical harm or death, or (2) ``be termed a willing partner.''\footnote{Casebook p. 414.}
        \item The defendant's actions demonstrate the requisite threat of force to prove robbery. Why doesn't it prove rape?
        \item Rape victims who resist are more likely to be injured than those who don't.\footnote{Casebook pp. 415--416.}
    \end{enumerate}
\end{enumerate}

\subsubsection{\emph{State v. Rusk}}

\begin{enumerate}
    \item The Court of Appeals of Maryland agreed with Judge Wilner's dissent (above). The victim's apprehension of fear ``was plainly a question of fact for the jury.'' Remanded for a new trial.
    \item Judge Cole, dissenting: words expressing fear ``do not transform a seducer into a rapist.'' Rape is a crime of violence. The victim ``must resist unless the defendant has objectively manifested his intent to use physical force to accomplish is purpose.''
\end{enumerate}

% TODO:
% \subsubsection{Susan Anger, \emph{The Incident}}
% 
% \begin{enumerate}
%     \item 
% \end{enumerate}
% 
% \subsubsection{\emph{State of New Jersey in the Interest of M.T.S.}}
% 
% \begin{enumerate}
%     \item 
% \end{enumerate}
% 
% \subsubsection{\emph{People v. Liberta}}
% 
% \begin{enumerate}
%     \item 
% \end{enumerate}
% 
% \subsubsection{Margaret Mitchell, \emph{Gone With the Wind}}
% 
% \begin{enumerate}
%     \item 
% \end{enumerate}
% 
% \subsubsection{\emph{State v. Gounagias}}
% 
% \begin{enumerate}
%     \item 
% \end{enumerate}

\subsection{Mens Rea}

\subsubsection{Mistake of Fact: \emph{People v. Williams}}

\begin{enumerate}
    \item Defendant and victim went to a hotel room. They had sex, but their factual accounts differ significantly. The defendant claimed that it was consensual, but that afterwards the victim said she would claim rape unless he gave her \$50. The victim claimed it was forcible rape.
    \item The trial court did not instruct the jury on a mistaken belief as to consent. It convicted the defendant of forcible rape and false imprisonment.
    \item The Court of Appeal reversed.
    \item Here, the Supreme Court of California based its understanding of mistake of fact in rape cases on \emph{People v. Mayberry}. A successful \emph{Mayberry} defense requires (1) a good faith belief in the mistaken fact and that (2) the mistake was reasonable. The jury can only receive instruction on the defense when there is ``substantial evidence'' to support it.
    \item The court here found that (1) there was no evidence of equivocal conduct on the part of the victim and (2) that the defendant's argument seeks to prove actual consent, not a reasonable mistake of consent. It reversed the appellate court.
    \item Mosk, concurring: first, the majority's interpretation of the \emph{Mayberry} rule is illogical. It would require the defendant to take the position that he was mistaken about consent, and therefore that there was no consent. Second, ``equivocal conduct'' from the victim is not necessary. Requiring it means that in cases where the facts are in dispute, such as this one, the jury must completely credit the victim's account and discredit the defendant's.
    \item Kennard, concurring: there are three fact patterns where force \emph{and} consent are present: (1) where the amount of force is ``slight,'' (2) where the victim consents to the use of force, and (3) where enough time has passed between the threat of force and the act of intercourse so that the defendant could reasonably believed that the victim's participation was not coerced.\footnote{Casebook p. 460}. The \emph{Mayberry} defense should only be available in these three cases.
\end{enumerate}

\subsection{Statutory Rape}

\subsubsection{\emph{State v. Garnett}}

\begin{enumerate}
    \item See p. 18.
    \item Statutory rape law as oppressive: young females are presumed ``too innocent and naive to understand the implications and nature of her act.'' The male is presumed criminally responsible---even in cases where he himself might be young and naive.\footnote{Casebook p. 478--479.}
    \item Early reforms: Victorian feminists urged statutory rape laws to curb the spread of venereal disease and protect young females from sexual abuse.\footnote{Casebook p. 477.}
    \item 1970s reforms: statutory rape laws seen as restrictions on sexual autonomy. Most jurisdictions made statutory rape gender neutral. Many have advocated for abolishing it entirely.\footnote{Casebook p. 478.}
    \item \emph{Michael M. v. Superior Court of Sonoma County}: US Supreme Court upheld gender-specific statutory rape laws on the basis of deterring teenage pregnancy.
\end{enumerate}

\subsubsection{\emph{State v. Limon}}

\begin{enumerate}
    \item Defendant had turned 18 one week before having homosexual intercourse with a 14-year-old. The age gap was less than four years. He was convicted of criminal sodomy and sentenced to 206 months in prison. Under the recent Kansas ``Romeo and Juliet'' statute, he would have been sentenced to only 13--15 months in prison if the contact had been heterosexual.\footnote{p. 1 (WestLaw).} He argued that the Romeo and Juliet statute violates equal protection because of harsher punishment for homosexuals.
    \item The trial court convicted him of criminal sodomy and the appellate court confirmed.
    \item The Supreme Court of Kansas considered in detail the possible rationales for the statute. It concluded that there is no rational basis for the law.
    \item The court held that the statute violated the equal protection clauses in both the state and federal constitutions. It reversed the conviction and struck the words ``and are members of the opposite sex'' from the statute.
\end{enumerate}

% \subsection{Defenses}
% 
% \subsubsection{Justification: Self Defense}
% 
% \paragraph{\emph{United States v. Peterson}}
% 
% \begin{enumerate}
%     \item todo
% \end{enumerate}
% 
% \paragraph{\emph{People v. Goetz}}
% 
% \begin{enumerate}
%     \item todo
% \end{enumerate}
% 
% \paragraph{\emph{State v. Wanrow}}
% 
% \begin{enumerate}
%     \item todo
% \end{enumerate}
% 
% \paragraph{\emph{State v. Norman}}
% 
% \begin{enumerate}
% %     \item todo
% \end{enumerate}
