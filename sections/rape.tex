\section{Rape}

\subsection{Overview}

\begin{enumerate}
    \item At common law, rape was defined as ``unlawful carnal knowledge of a 
    woman [not the defendant's wife] \textbf{forcibly} and \textbf{against her 
    will}.''\footnote{Blackstone, 4 Commentaries on the Laws of England 210.}
    \item 32\% of rapes were reported to law enforcement in 1994 and 
    1995.\footnote{Casebook p. 385.}
    \item 91\% of victims were female.\footnote{Casebook p. 386.} (MPC \S\ 
    213.1 defines rape only as male-against-female.)
    \item 8\% of forcible rapes reported in 1995 turned out to be 
    unfounded.\footnote{Casebook p. 387.}
    \item Susan Estrich: rape law exposes the sexism of the law.
    \item Two frameworks for understanding the crime of rape: (1) a crime of 
    violence and (2) a crime against sexual autonomy.\footnote{Casebook p.  
    391.}
    \item Rape is widely believed to be underreported because of (1) the 
    sensitivity of the issue, which makes it hard to bring up, and (2) the 
    difficulty of proof in many cases.
    \item The force requirement may have developed to guard against false 
    claims of rape as defenses to fornication and adultery.
    \item \textbf{Competing definitions of force}: (1) force used to overcome 
    lack of consent or (2) any amount of sexual touching brought about 
    involuntarily.\footnote{Casebook p. 437.} The threat of serious physical 
    harm can satisfy the force requirement.
    \item Mens rea is more difficult to establish without the force 
    requirement.
    \item The MPC abolished criminal sexual offenses between consenting adults 
    (e.g., sodomy).
    \item Rape law has been widely reformed to (1) eliminate the resistance 
    requirement, (2) allow penetration to suffice for the force requirement, 
    and (3) eliminate the spouse rape exemption.
    \item Can you be raped, but not by a rapist?
\end{enumerate}

\subsubsection{Annette Gordon Reed, \emph{Celia's Case}}

\begin{enumerate}
    \item A white male landowner bought Celia, a female slave, and treated her 
    as a concubine. She eventually decided to stand up to his advances and 
    ended up killing him by hitting him on the head. She stood trial and was 
    put to death.
\end{enumerate}

\subsubsection{C. Vann Woodward, Review of \emph{Scottsboro: A Tragedy of the 
American South}}

\begin{enumerate}
    \item Two white women falsely accused the nine Scottsboro Boys of rape.  
    All were convicted and sentenced. Eventually, all were exonerated.
\end{enumerate}

\subsubsection{NY Times Obituary for Ruth Schut}

\begin{enumerate}
    \item A note on the death of one of the two accusers of the Scottsboro 
    Boys. She later recanted her accusations.
\end{enumerate}

\subsection{Forcible Rape}

\begin{enumerate}
    \item Some jurisdictions require (or required) the victim to forcibly 
    resist for the defendant to be convicted of rape. See \emph{Alston} below.
    \item The forcible resistance requirement has been seriously questioned.  
    See \emph{Rusk} below. Critiques:
    \begin{enumerate}
        \item It conflates the force and non-consent elements.
        \item It emphasizes the victim's conduct and deemphasizes the 
        defendant's conduct.
        \item The victim's resistance could further compromise the victim's 
        safety.
        \item Resistance is not always the immediate response to unwanted 
        advances.
    \end{enumerate}
    \item \textbf{Threats of force can be sufficient} to meet the force 
    requirement---e.g., brandishing a knife. The victim need not forcibly 
    resist.
    \item Failure to obtain \textbf{affirmative assent} can be enough to 
    convict for rape in some jurisdictions. Others require the defendant to 
    show reasonable belief in consent.
    \item In some jurisdictions, the force needed for penetration can meet the 
    force requirement. See \emph{M.T.S.} below.
    \item The \textbf{spouse rape exemption} has been largely dismissed. See 
    \emph{Liberta} below. Its rationales included:
    \begin{enumerate}
        \item Promotion of marital harmony.
        \item Preservation of marital privacy.
        \item Encourage reconciliation between spouses.
        \item Limit opportunities to fabricate rape charges for divorce 
        proceedings.
    \end{enumerate}
\end{enumerate}

\subsubsection{The Force Requirement: \emph{State v. Alston}}

Some jurisdiction require (or required) the victim to forcibly resist for the 
defendant to be convicted of rape.

\begin{enumerate}
    \item Alston and Brown were in a semi-abusive relationship. She broke it 
    off. Then, one day, he coerced her to come to a friend's house. She did 
    not forcibly resist his sexual advances. Later that day, she filed a 
    police complaint.
    \item The trial court convicted the defendant of second-degree rape.
    \item The Supreme Court of North Carolina reasoned that the statutory 
    definition of rape required intercourse to be (1) by force and (2) against 
    the victim's will. The court found no evidence that the victim forcibly 
    resisted. It overturned the conviction.
    \item Susan Estrich: The victim was not forced to engage in sex, but did 
    so against her will. ``To say that there is no `force' in such a situation 
    is to create a gulf between power and force, and to define the latter 
    solely in schoolboy terms.''\footnote{Casebook p. 409}
    \item Vivian Berger: It's not clear that this was actually a rape.  
    Overprotecting women ``risks enfeebling instead of 
    empowering.''\footnote{Casebook p. 410.}
\end{enumerate}

\subsubsection{Questioning the Force Requirement: \emph{Rusk v. State}}

Judge Wilner's dissent rails against the force requirement.

\begin{enumerate}
    \item The victim met the defendant at bar. She drove him home. She did not 
    want to go into his apartment, but he took her car keys out of her 
    ignition, and she followed him out of fear. They had sex and she did not 
    resist.
    \item The trial court convicted him of rape. He argued on appeal that 
    there was insufficient evidence for conviction. The Court of Special 
    Appeals of Maryland here reversed.
    \item Judge Wilner, dissenting, argues that lack of resistance is not 
    consent:
    \begin{enumerate}
        \item The court inappropriately substituted its judgement for the 
        jury's.
        \item The court's reasoning requires the victim to either (1) resist, 
        risking physical harm or death, or (2) ``be termed a willing 
        partner.''\footnote{Casebook p. 414.}
        \item The defendant's actions demonstrate the requisite threat of 
        force to prove robbery. Why doesn't it prove rape?
        \item Rape victims who resist are more likely to be injured than those 
        who don't.\footnote{Casebook pp. 415--416.}
        \item Courts can look to factors other than physical resistance, e.g., 
        reasonable expressions of fear.
    \end{enumerate}
\end{enumerate}

\subsubsection{Removing the Force Requirement: \emph{State v. Rusk}}

The court here abandoned the requirement. The dissent's fear might be based on 
the racial subtext---e.g., in Scottsboro the accused were wrongly convicted, 
and a force requirement might have given some protection.

\begin{enumerate}
    \item The Court of Appeals of Maryland agreed with Judge Wilner's dissent 
    (above). The victim's apprehension of fear ``was plainly a question of 
    fact for the jury.'' Remanded for a new trial.
    \item Judge Cole, dissenting: words expressing fear ``do not transform a 
    seducer into a rapist.'' Rape is a crime of violence. The victim ``must 
    resist unless the defendant has objectively manifested his intent to use 
    physical force to accomplish is purpose.''
\end{enumerate}

\subsubsection{Susan Anger, \emph{The Incident}}

\begin{enumerate}
    \item Anger writes about her rape and why she did not forcibly resist.
\end{enumerate}

\subsubsection{Acquaintance Rape: \emph{State of New Jersey in the Interest of 
M.T.S.}}

The prosecution only needs to show force required for penetration, not 
additional physical force. Rape is proven if there was force and lack of 
consent.

\begin{enumerate}
    \item Fifteen-year-old C.G. lived in a house with nine other people, 
    including seventeen-year-old M.T.S. C.G. testified that she woke up to 
    find M.T.S. on top of her and sexually penetrating her. She slapped him 
    and ``he jumped right off.''\footnote{Casebook p. 436.} According to 
    M.T.S., they briefly had consensual sex before she pushed him off and ``he 
    hopped off right away.'' ``The court did not fully credit either 
    teenager's testimony.''\footnote{Casebook p. 435.}
    \item The trial court found him guilty of sexual assault. The appellate 
    court reversed, holding that rape required ``some level of force more than 
    that necessary to accomplish the penetration.''\footnote{Casebook p. 435.}
    \item The Supreme Court of New Jersey held that the force requirement did 
    not call for physical force beyond the force needed for penetration. Rape 
    is proven if there is (1) force needed for penetration and (2) lack of 
    consent. Both elements were present here. Reversed.
    \item New Jersey reframed rape as a crime of violence, focusing on the 
    defendant's actions, not the victim's. This aligns with the law's analysis 
    of other violent crimes---e.g., you don't require a victim's resistance in 
    robbery.
\end{enumerate}

\subsubsection{Ending Wife Rape: \emph{People v. Liberta}}

\begin{enumerate}
    \item The New York Supreme Court squarely rejected the spouse exemption 
    for rape.
    ``Among the recent decisions in this country addressing the marital 
    exemption, only one court has concluded that there is a rational basis for 
    it.... We agree with the other courts which have analyzed the exemption, 
    which have been unable to find any present justification for it.... [T]he 
    marital exemption...lacks a rational basis, and therefore violates the 
    equal protection clauses of both the Federal and State Constitutions.
\end{enumerate}

\subsubsection{Margaret Mitchell, \emph{Gone With the Wind}}

\begin{enumerate}
    \item An excerpt from the controversial ``rape of Scarlett'' where 
    forcible sex becomes consensual.
\end{enumerate}

\subsubsection{Male Rape: \emph{State v. Gounagias}}

\begin{enumerate}
    \item An instance of a male rape victim.
\end{enumerate}

\subsection{Mens Rea}

\subsubsection{Mistake of Fact: \emph{People v. Williams}}

Mistake of fact regarding consent is available as a defense when (1) the 
defendant believed in good faith that there was consent and (2) the mistake 
was reasonable.

\begin{enumerate}
    \item The defendant and the victim went to a hotel room to watch TV, 
    according to the victim. They had sex, but their factual accounts differ 
    significantly. The defendant claimed that it was consensual, but that 
    afterwards the victim said she would claim rape unless he gave her \$50. 
    The victim claimed it was forcible non-consensual sex.
    \item The trial court did not instruct the jury on a mistaken belief of 
    consent. It convicted the defendant of forcible rape and false 
    imprisonment.
    \item The Court of Appeal reversed.
    \item Here, the Supreme Court of California based its understanding of 
    mistake of fact in rape cases on \emph{People v. Mayberry}. A successful 
    \emph{Mayberry} defense requires (1) a good faith belief in the mistaken 
    fact and (2) that the mistake was reasonable. The jury can only receive 
    instruction on the defense when there is ``substantial evidence'' to 
    support it.
    \item For the defense to be available, there must have been substantial 
    evidence of equivocal conduct. The court here found that (1) there was no 
    evidence of equivocal conduct on the part of the victim and (2) that the 
    defendant's argument sought to prove actual consent, not a reasonable 
    mistake of consent. It reversed the appellate court.
    \item Mosk, concurring: first, the majority's interpretation of the 
    \emph{Mayberry} rule is illogical. It would require the defendant to take 
    the position that he was mistaken about consent, and therefore that there 
    was no consent. Second, ``equivocal conduct'' from the victim is not 
    necessary. Requiring it means that in cases where the facts are in 
    dispute, such as this one, the jury must completely credit the victim's 
    account and discredit the defendant's.
    \item Kennard, concurring: there are three fact patterns where force 
    \emph{and} consent are present: (1) where the amount of force is 
    ``slight,'' (2) where the victim consents to the use of force, and (3) 
    where enough time has passed between the threat of force and the act of 
    intercourse so that the defendant could reasonably believed that the 
    victim's participation was not coerced.\footnote{Casebook p. 460}. The 
    \emph{Mayberry} defense should only be available in these three cases.
\end{enumerate}

\subsection{Statutory Rape}

\begin{enumerate}
    \item Policy goals of statutory rape laws:
    \begin{enumerate}
        \item Protect children.
        \item Provide recourse for rape victims who could not prove force or 
        resistance.
    \end{enumerate}
    \item Criticisms:
    \begin{enumerate}
        \item Criminalizes consensual activity.
        \item Limits sexual autonomy.
        \item May be used in a discriminatory way to prosecute teenagers.
    \end{enumerate}
    \item Reforms:
    \begin{enumerate}
        \item Limit liability for defendants who are close in age to victims.
        \item Recognizing mistake of fact regarding the victim's age as a 
        defense.
        \item Selective prosecution.
    \end{enumerate}
    \item Some jurisdictions define statutory rape as a strict liability 
    offense. In those cases, mistake of fact is not a defense.
\end{enumerate}

\subsubsection{\emph{State v. Garnett}}

A quick history of statutory rape law.

\begin{enumerate}
    \item Statutory rape law as oppressive: young females are presumed ``too 
    innocent and naive to understand the implications and nature of her act.'' 
    The male is presumed criminally responsible---even in cases where he 
    himself might be young and naive.\footnote{Casebook p. 478--479.}
    \item Early reforms: Victorian feminists urged statutory rape laws to curb 
    the spread of venereal disease and protect young females from sexual 
    abuse.\footnote{Casebook p. 477.}
    \item 1970s reforms: statutory rape laws seen as restrictions on sexual 
    autonomy. Most jurisdictions made statutory rape gender neutral. Many have 
    advocated for abolishing it entirely.\footnote{Casebook p. 478.}
    \item \emph{Michael M. v. Superior Court of Sonoma County}: US Supreme 
    Court upheld gender-specific statutory rape laws on the basis of deterring 
    teenage pregnancy.
\end{enumerate}

\subsubsection{Equal Protection: \emph{State v. Limon}}

Statutes defining different sex crime standards for homosexual and 
heterosexual acts violate equal protection.

\begin{enumerate}
    \item The defendant had turned 18 one week before having homosexual 
    intercourse with a 14-year-old. The age gap was less than four years. He 
    was convicted of criminal sodomy and sentenced to 206 months in prison.  
    Under the recent Kansas ``Romeo and Juliet'' statute, he would have been 
    sentenced to only 13--15 months in prison if the contact had been 
    heterosexual.\footnote{p. 1 (WestLaw).} He argued that the Romeo and 
    Juliet statute violated equal protection because of harsher punishment for 
    homosexuals.
    \item The trial court convicted him of criminal sodomy and the appellate 
    court confirmed.
    \item The Supreme Court of Kansas considered in detail the possible 
    rationales for the statute. It concluded that there was no rational basis 
    for the law.
    \item The court held that the statute violated the equal protection 
    clauses in both the state and federal constitutions. It reversed the 
    conviction and struck the words ``and are members of the opposite sex'' 
    from the statute.
\end{enumerate}
