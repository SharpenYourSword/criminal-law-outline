\section{Defenses}

\subsection{Overview}

\begin{enumerate}
    \item A defense is ``any set of identifiable conditions or circumstances 
    which may prevent a conviction for an offense.''\footnote{Casebook p. 480.}
    \item \textbf{Failure of proof defense}: negate a required element of the 
    offense---e.g., mistake of fact in a strict liability crime.
    \item \textbf{Offense modification}: ``while the actor has apparently 
    satisfied all the elements of the offense charged, he has not in fact 
    caused the harm or evil sought to be prevented by the statute defining the 
    offense''---e.g., a parent of a kidnapped child pays ransom to the 
    kidnapper, which satisfies the elements of complicity in kidnapping but 
    confers a defense to criminal charges.\footnote{Casebook p. 482.}
    \item \textbf{Justification}: society tolerates or even encourages the 
    conduct---e.g., burning a field to save a town of 10,000. 
    \item \textbf{Excuse}: the deed may be wrong, but the conditions are such 
    that the actor is not responsible---e.g., the ``actor knocks the mailman 
    over the head with a baseball bat because she believes he is coming to 
    surgically implant a radio receiver which will take control of her 
    body.''\footnote{Casebook p. 482.}
    \item \textbf{Nonexculpatory public policy defense}: the act is wrong, but 
    policy reasons dictate a defense---e.g., the statute of limitations.
    \item Defenses in the MPC are in \S\ 3.
    \item \textbf{Justification of the \emph{act}, excuse of the \emph{actor}.}
\end{enumerate}

\subsection{Justification}

\begin{enumerate}
    \item Justification focuses on the \textbf{act, not the actor}.
    \item Justified acts are socially redeemable.
    \item Structure of justification defenses: ``triggering conditions permit 
    a necessary and proportional response.''\footnote{Casebook p. 497.}
    \begin{enumerate}
        \item \emph{Triggering conditions}: circumstances that must exist 
        before the actor can act with justification---e.g., an aggressor 
        threatens unjustified harm against a protected interest.
        \item \emph{Necessary}: the response must be necessary to protect the 
        interest at stake.
        \item \emph{Proportional}: the act must cause harm in reasonable 
        relation to the harm threatened.
    \end{enumerate}
    \item Where an actor has no option but deadly force to prevent a thief 
    from stealing apples from her orchard, she must sacrifice her apples out 
    of regard to the thief's life.\footnote{Casebook p. 497.}
\end{enumerate}

\subsubsection{Self-Defense}

\begin{enumerate}
    \item Self-defense is the most common justification defense.
    \item Common law elements of self-defense (all are necessary):
    \begin{enumerate}
        \item An honest and reasonable fear of death or great bodily harm.
        \item An imminent and unlawful threat.
        \item A proportional response to the threat.
        \item The defendant must not have been the initial aggressor.
        \item (Some jurisdictions impose a duty on the defendant to retreat, 
        except in his own home.)
    \end{enumerate}
    \item MPC:
    \begin{enumerate}
        \item \emph{Subjectivity}: the actor must believe the use of force is 
        necessary. Reasonableness is not required.\footnote{MPC \S\ 3.09(2).}
        \item \emph{``Force is immediately necessary'' replaces 
        ``imminent threat''}: shifts from threat to necessity of force (e.g., 
        allows a battered spouse to shoot the abuser while the abuser 
        sleeps---see \emph{Norman} below).
        \item \emph{Expands castle doctrine} to include place of work.
        \item As at common law, the MPC does not allow the defense if the 
        defendant provoked the use of force against himself by using serious 
        force, or if he can safely retreat (except at home or at work).
    \end{enumerate}
    \item The two key differences between common law and the MPC are 
    \textbf{objectively vs. subjectively reasonable} requirement and the 
    \textbf{immediately necessary vs. imminent harm} requirement.
    \item Critiques of self-defense (from Murray):
    \begin{enumerate}
        \item In recent years, critics have argued that the law of 
        self-defense is insufficiently attentive to the physical (and other) 
        differences between men and women.
        \item Some have argued that the emphasis on the reasonable man's 
        response makes the male experience the legal default.
        \item In this vein, the objective standard is said to preclude 
        defensive actions that are objectively unreasonable, but that are 
        completely reasonable in light of the defendant's idiosyncratic 
        circumstances.
        \item Further, many have argued that the doctrine's imminence 
        requirement precludes the use of the defense in domestic violence 
        situations.
        \item The MPC and its more subjective standards might appear to be 
        more attuned to these sorts of criticisms than the common law 
        approach. 
    \end{enumerate}
    \item \textbf{Imperfect self-defense} mitigates to manslaughter. Two versions:
    \begin{enumerate}
        \item First: a nondeadly aggressor who is the victim of a deadly 
        atttack must retreat to a place of complete safety before using deadly 
        force. Otherwise, his defense is considered imperfect.
        \item Second: unreasonable belief that the killing in self-defense was 
        justified.\footnote{Understanding Criminal Law pp. 232--33.}
        \item MPC: the defendant is justified in using deadly force if he 
        subjectively believed that it was necessary. However, if he was 
        reckless or negligent regarding the facts relating to his conduct, he 
        cannot use the defense for crimes for which recklessness or negligence 
        establish culpability (e.g., he cannot raise self-defense against the 
        charge of negligent homicide).\footnote{MPC \S\ 3.09(2) and 
        Understanding Criminal Law p. 253.}
    \end{enumerate}
\end{enumerate}

\paragraph{Self-Defense and Provocation: \emph{United States v. Peterson}}
~\\\\
Self-defense is usually not available to a defendant to provokes or is the 
aggressor in it. The castle doctrine is only available to a defendant who is 
without fault in bringing on the conflict.

\begin{enumerate}
    \item The victim was stealing the windshield wipers from the defendant's 
    car. The defendant got a pistol and said, ``If you move, I will shoot.'' 
    The victim picked up a wrench and walked toward the defendant with the 
    wrench raised. The defendant told him to ``not take another step.'' The 
    victim kept walking and defendant shot him in the face, killing him 
    instantly.
    \item The trial court indicted him for second-degree murder and the jury 
    convicted him of manslaughter.
    \item The defendant argued that two jury instructions were erroneous: (1) 
    on whether he was the aggressor in the altercation before the homicide, and 
    (2) on whether he could have retreated without jeopardizing his safety.
    \item The defendant argued that the killing was excusable.
    \item At common law, there would be no excuse defense.
    \item On the first point, the court found that ``the evidence plainly 
    presented an issue of fact as to whether Peterson's conduct was an 
    invitation to and provocation of the encounter which ended in the fatal 
    shot.''\footnote{p. 504.}
    \item On the second point, the court found that the instruction on retreat 
    was proper. The defendant countered that the ``castle'' doctrine does not 
    require a defendant to retreat from his own house. The court rejected this 
    argument on the grounds that the doctrine ``can be invoked only by one who 
    is without fault in bringing the conflict on.''\footnote{Casebook p. 505.}
    \item Affirmed.
\end{enumerate}

\paragraph{Self-Defense and Reasonable Belief: \emph{People v. Goetz}}
~\\\\
In New York, to successfully claim self-defense a defendant must have acted as 
a reasonable person would have acted.

\begin{enumerate}
    \item Goetz shot several youths on a New York City subway after they 
    approached him and asked for money.
    \item The trial court indicted Goetz for attempted murder, assault, and 
    weapons possession. It dismissed the case on erroneous jury instructions 
    on the defense of justification. It found that the prosecutor erred in 
    instructing jurors to consider how a reasonable man would have acted under 
    the circumstances. The statutory test for justification should be ``wholly 
    subjective.''\footnote{Casebook p. 513.}
    \item Appellate court: affirmed dismissal.\footnote{Casebook p. 513.}
    \item The prosecutor, the trial court, and a plurality of the appellate 
    court read the New York justification statute to mean that the defendant's 
    belief must have been ``reasonable to him.'' This is consistent with the 
    MPC, on which the NY criminal statutory reforms of 1961 were based.
    \item However, the court here argued that the NY statute does not follow 
    the MPC. The legislature intended to keep the objective reasonableness 
    requirement.
    \item Court of Appeals reversed and reinstated the indictments.
\end{enumerate}

\paragraph{Abused Spouse Syndrome: \emph{State v. Norman}}
~\\\\
Self-defense for battered spouses presents a jury question.

\begin{enumerate}
    \item J.T. Norman was found dead from two gunshot wounds to the head. The 
    defendant, his wife, told police that he had been beating her all day and 
    that she shot him while he slept. The trial court refused to grant an 
    acquittal on self-defense.
    \item Multiple experts testified that Mrs. Norman suffered from abused 
    spouse syndrome and did not leave because she thought escape was 
    completely impossible. The experts testified that she believed killing 
    her husband while he slept was necessary.
    \item The court held that she ``believed killing the victim was necessary 
    to avoid being killed.''\footnote{Casebook p. 536.} The evidence was 
    sufficient to send the question of self-defense to the jury. Reversed.
\end{enumerate}

\paragraph{Imminence: \emph{State v. Norman}}
~\\\\
Self-defense requires an imminent, not potential, threat of harm.

\begin{enumerate}
     \item On appeal from the appellate court case (above), the North Carolina 
     Supreme Court reversed, holding that self-defense requires a threat of 
     \emph{imminent} death or great bodily harm. The defendant ``had ample 
     time and opportunity to resort to other means of prevent further abuse of 
     her husband.''\footnote{Casebook p. 538.}
     \item If the appellate court's approach were allowed, ``[h]omicidal 
     self-help would then become a lawful solution'' for battered 
     spouses.\footnote{Casebook p. 539.}
\end{enumerate}

\subsubsection{Necessity}

\begin{enumerate}
    \item Recognized in around twenty jurisdictions.
    \item Elements of the common law necessity defense (compare to MPC \S\ 
    3.02, below):
    \begin{enumerate}
        \item Clear and imminent danger.
        \item Actor expects, as a reasonable person, that his action will 
        abate the danger.
        \item No legal way to avert the danger.
        \item The harm caused by violating the law is less than the harm to be 
        avoided.
        \item The defense is not prohibited by statute.
        \item ``Clean hands'': defendant must not have substantially 
        contributed to the emergency.
    \end{enumerate}
    \item The defendant's actions are evaluated in terms of the harm that was 
    reasonably foreseeable at the time, rather than the harm that actually 
    occurred.
    \item Common (but not universal) limitations on the necessity defense:
    \begin{enumerate}
        \item Preclusion by legislature.
        \item ``Clean hands'' requirement.
        \item Danger must have been created by a force of nature.
        \item Must be used to protect persons or property (and not, say, 
        economic interests).
    \end{enumerate}
    \item Why have the necessity defense?
    \begin{enumerate}
        \item Utilitarianism: maximize social welfare.
        \item Retributivism: an actor cannot be culpable if the goal is to 
        avoid harm.
    \end{enumerate}
\end{enumerate}

\paragraph{MPC \S\ 3.02 (``Choice of Evils'') and ALI Commentary}

\begin{itemize}
    \item (1) Conduct to avoid harm is justified if:
    \begin{itemize}
        \item (a) The harm avoided is greater than the harm caused.
        \item (b) There are no explicit exceptions for this particular 
        situation.
        \item (c) There is no legislative purpose to exclude the claimed 
        justification.
    \end{itemize}
    \item (2) If the actor recklessly or negligently caused the situation, the 
    justification offense cannot be used to defend against an offense for 
    which recklessness or negligence establishes culpability.
    \item \textbf{MPC differences from common law necessity}:
    \begin{enumerate}
        \item Harm need not be imminent.
        \item No ``clean hands'' requirement, i.e., the defendant can have contributed to the emergency.
        \item No mention of homicide.
        \item No mention of economic interests vs. personal interests. 
        Arguably it would allow the necessity defense for protection of 
        economic interests.
        \item Does not permit the defense where (1) the defendant's belief in the 
        danger was negligent or reckless and (2) the mental state required for 
        the offense was recklessness or negligence.\footnote{MPC \S\ 3.02(2).}
    \end{enumerate}
\end{itemize}

\paragraph{Emergencies: \emph{Nelson v. State}}
~\\\\
You can appropriate property in emergencies, but only if a reasonable person 
would believe the situation constituted an emergency.

\begin{enumerate}
    \item Nelson's truck became bogged down. He and two companions took a dump 
    truck from the Highway Department Yard to help pull out the truck, and it 
    also got stuck. A man named Curly helped them take a front-end loader from 
    the yard, which also got stuck.
    \item The trial court and appellate court affirmed convictions of reckless 
    destruction of personal property and joyriding.
    \item Nelson argued that the jury instruction on the necessity defense 
    should have been based on a subjective interpretation of the emergency 
    situation. The court here agreed, but held that there was not a reasonable 
    apprehension of emergency (since the truck remained stable for 12 hours 
    and one of the actors slept in it). Moreover, there were lawful 
    alternatives. Affirmed.
\end{enumerate}

\paragraph{Civil Disobedience: \emph{United States v. Schoon}}
~\\\\
Laws and policies do not constitute the types of harms that allow for the 
necessity defense.

\begin{enumerate}
    \item The three defendants disrupted an IRS office out of protest of US 
    policies in El Salvador. The district court denied them the necessity 
    defense.
    \item There are four elements of the necessity defense:
    \begin{enumerate}
        \item Actors chose the lesser evil.
        \item They acted to prevent imminent harm.
        \item They ``reasonably anticipated a direct causal relationship 
        between their conduct and the harm to be averted.''\footnote{Casebook 
        p. 567.}
        \item They had no legal alternatives.
    \end{enumerate}
    \item The necessity defense is not available if any of the four elements 
    are lacking.
    \item The court distinguished between indirect and direct civil 
    disobedience. This case involved indirect civil disobedience.
    \item Necessity is a utilitarian defense because it aims to maximize 
    social welfare.
    \item The court reasoned:
    \begin{enumerate}
        \item Balance of harms: law or policy ``cannot constitute a legally 
        cognizable harm.''
        \item Causal relationship: indirect civil disobedience is ``unlikely 
        to abate the evil.''
        \item Legal alternatives: ``legal alternatives will never be deemed 
        exhausted when the harm can be mitigated by congressional 
        action.''\footnote{Casebook p. 570.}
    \end{enumerate}
    \item Affirmed.
    \item Judge Fernandez, concurring: ``I am not so sure that this defense of 
    justification should be grounded on utilitarian theory alone.''
\end{enumerate}

\paragraph{\emph{The Queen v. Dudley and Stephens}}

\begin{enumerate}
    \item Facts: see p. 6.
    \item Was the killing of Richard Parker necessary and justified, or was it 
    murder?
    \item Bracton and Hale: the only justified killing is in self-defense.
    \item \emph{United States v. Holmes}: If killing some would save others, 
    and there is no imminent danger, those to die must be chosen by lot.
    \item Bacon: contradicting Hale---if you're on a lifeboat, and another 
    passenger would sink it, you can justifiably push him away.
    \item The court here found murder. ``...a man has no right to declare 
    temptation to be an excuse, though he might himself have yielded to 
    it...''\footnote{Casebook p. 575.}
\end{enumerate}

\subsection{Excuse}

\begin{enumerate}
    \item Focuses on the \textbf{actor, not the act}.
    \item Theories of excuse:
    \begin{enumerate}
        \item \textbf{Utilitarian}: Bentham:``identify situations in which 
        conduct is nondeterrable, so that punishment would be so much 
        unnecessary evil.'' Hart: excuses maximize the effect of a person's 
        choices within the framework of coercive law.
        \item \textbf{Causation}: ``a person should not be blamed for her 
        conduct if it was caused by factors outside her 
        control.''\footnote{Casebook p. 581.}
        \item \textbf{Character}: punishment should be proportional to a 
        wrongdoer's moral desert, and moral desert should be measured by the 
        actor's overall character. Or: a person should not be blamed for her 
        conduct in circumstances where bad character cannot be inferred from 
        the conduct.
        \item \textbf{Free Choice}: people should only be punished for actions 
        they freely committed. An actor is free if he has the capacity and 
        opportunity to (1) understand the facts, (2) appreciate that the 
        conduct violates society's mores, and (3) conform her conduct to the 
        dictates of the law.\footnote{Casebook p. 582.}
    \end{enumerate}
\end{enumerate}

\subsubsection{Duress}

\begin{enumerate}
    \item \textbf{Elements of the duress defense}:
    \begin{enumerate}
        \item Another person threatened to kill or seriously injure the 
        defendant or a third party (usually a close relative) unless the 
        defendant commit the offense.
        \item The defendant reasonably believed the threat was genuine.
        \item The threat was ``present, imminent, and impending'' at the time 
        of the criminal act.
        \item There was no reasonable escape from the threat except through 
        compliance.
        \item The defendant was not at fault in exposing himself to the 
        threat.
        \item The defense is not available in homicide cases.
    \end{enumerate}
    \item MPC duress:\footnote{MPC \S\ 2.09(1)}: ``It is an affirmative 
    defense that the actor was engaged in the conduct charged to constitute an 
    offense because he was coerced to do so by the use of, or a threat to use, 
    unlawful force against his person or the person of another, that a person 
    of reasonable firmness in his situation would have been unable to 
    resist.''
    \item Common law vs. MPC:
    \begin{enumerate}
        \item Abandons the deadly force and imminent threat requirements. 
        Instead, it takes the threat and its imminence as factors in 
        evaluating whether person of reasonable firmness under the 
        circumstances would have committed the offense.
        \item The threat can be to any person, regardless of the defendant's 
        relationship.
        \item The reasonableness requirement takes into account the 
        defendant's full situation (e.g., prior experiences, emotional 
        condition).
        \item The MPC \emph{does} allow the duress defense in homicide cases.
    \end{enumerate}
\end{enumerate}

\paragraph{Coercion to Smuggle Drugs: \emph{United States v. Contento-Pachon}}
~\\\\
The duress defense can be available to someone coerced into smuggling drugs.

\begin{enumerate}
    \item The defendant was coerced into smuggling cocaine into the US. 
    \item The trial court excluded evidence of duress.
    \item The Appellate court here held that there was sufficient evidence of 
    duress to send the factual questions to a jury.
    \item There are three elements of duress: (1) immediate threat of death or 
    serious bodily injury, (2) well-grounded fear that the threat will be 
    carried out, and (3) no reasonable opportunity to escape.
    \item The necessity defense was not available because the situation was 
    not caused by physical forces and the defendant did not act to promote the 
    general welfare.
    \item Coyle, dissenting and concurring: defendant failed to establish the 
    immediacy and inescapability needed for the excuse defense.
\end{enumerate}

\paragraph{Duress and Murder: \emph{People v. Anderson}}
~\\\\
Duress is never a defense to murder (at least in California).

\begin{enumerate}
    \item The defendant, a father, was coerced by another father to murder a 
    woman suspected of molesting their children.
    \item The defendant argued that he killed under duress.
    \item The jury convicted defendant of first-degree murder and kidnapping.
    \item A California statute prevents the duress defense for any crime 
    ``punishable by death.'' When the statute was passed, all forms of murder 
    were punishable by death. Now, only first degree murder is punishable by 
    death. The court here chose to retain the original scope. It held that 
    duress is never a defense to murder.
    \item The defendant also argued that duress negated the malice necessary 
    to find murder, and therefore the charge should be reduced to 
    manslaughter. The court held that this would require a new form of 
    voluntary manslaughter, and that only the legislature can make this 
    designation.
    \item The dissent argued that the statutory language should apply to 
    crimes currently punishable by death, not at the time of the drafting 
    of the original statute.
\end{enumerate}

\subsubsection{Intoxication}

\begin{enumerate}
    \item \textbf{Voluntary intoxication}:
        \begin{enumerate}
            \item Early common law: \textbf{Not a defense at common law}. The intoxicated 
            defendant ``shall have no privilege by this \textbf{voluntarily 
            contracted madness}, but shall have the same judgment as if he 
            were in his right senses.''\footnote{Sir Matthew Hale.}
            \item Current common law: some jurisdictions only allow voluntary 
            intoxication to mitigate punishment for graded specific intent 
            crimes (but it cannot completely exculpate the defendant). Others 
            allow it to be a completely exculpatory defense for any specific 
            intent crimes.
            \item \emph{Montana v. Egelhoff}: the Supreme Court held that 
            states can preclude the defendant from using intoxication to 
            negate mens rea without violating due process.
            \item MPC \S\ 2.08:
            \begin{itemize}
                \item ``[I]ntoxication of the actor is not a defense unless it 
                negatives an element of the offense.''
                \item When recklessness establishes an element of the offense, if the 
                actor, due to self-induced intoxication, is unaware of a risk of which 
                he would have been unaware had he been sober, such unawareness is 
                immaterial.
            \end{itemize}
        \end{enumerate}
    \item \textbf{Involuntary intoxication}:
    \begin{enumerate}
        \item Common law:
        \begin{enumerate}
            \item Intoxication is involuntary if coerced, the result of an 
            innocent mistake, unexpected from a prescribed medication within 
            the prescribed dose, or ``pathological intoxication'' (an 
            unforeseen psychotic reaction to a substance). 
            \item Proof of involuntary intoxication can negate the mens rea 
            for both general and specific intent crimes.
        \end{enumerate}
        \item MPC \S\ 2.08(4):
        \begin{enumerate}
            \item ``Intoxication that (a) is not self-induced or (b) is 
            pathological is an affirmative defense if by reason of such 
            intoxication the actor at the time of his conduct lacks 
            substantial capacity either to appreciate its criminality 
            [wrongfulness] or to conform his conduct to the requirements of 
            law.''
        \end{enumerate}
    \end{enumerate}
\end{enumerate}

\paragraph{Specific Intent Crimes: \emph{United States v. Veach}}
~\\\\
Intoxication is a defense only to specific intent crimes.

\begin{enumerate}
    \item The defendant was drunk and got involved in a car accident. He made 
    death threats against the park rangers who arrested him.
    \item The jury convicted him of one count of resisting a federal law 
    enforcement officer and two counts of threatening to assault and murder a 
    law enforcement officer.
    \item The court held that intoxication is a defense when it negates the 
    mens rea of a crime.
    \item Resisting an officer is a general intent crime, so intoxication is 
    not a defense.
    \item Threatening an officer, however, is a specific intent crime, so the 
    defendant should have been allowed to present the defense to a jury.
\end{enumerate}

\subsubsection{Insanity}

\begin{enumerate}
    \item Mental illness in medicine is a broad spectrum. Mental illness in 
    law is a binary state. Sometimes, mental illness in medicine is 
    insufficient to establish mental illness in law.
    \item Insanity can only be raised for \textbf{recognized mental 
    disorders}. Other mental incapacity claims (e.g., inability to form intent 
    because of toxic exposure) fall under diminished capacity.
    \item \textbf{``Deific decree''}: ``God told me to do it,'' and the actor 
    realized that their conduct was socially unacceptable. The MPC allows 
    this to establish insanity because the he lacked substantial capacity to 
    appreciate wrongfulness. The M'Naghten test does not allow it because the 
    actor knows it is wrong (and so many M'Naghten jurisdictions include 
    specific exceptions for deific decrees, but not for general religious 
    beliefs).
    \item \textbf{Insanity tests}: see \emph{Johnson} below.
\end{enumerate}

\paragraph{\emph{United States v. Freeman}}
~\\\\
Insanity is available as an excuse defense.

\begin{enumerate}
    \item ``...none of the three asserted purposes of criminal 
    law---rehabilitation, deterrence, and retribution---is satisfied when the 
    truly irresponsible are punished.''\footnote{Casebook p. 616.}
    \item The insanity defense aims to ``draw a line between the bad and the 
    mad.''\footnote{Casebook p. 617.}
\end{enumerate}

\paragraph{\emph{State v. Johnson}}
~\\\\
There are several competing tests for determining whether a defendant 
qualifies for the insanity defense. The M'Naghten and MPC tests are the most 
popular. Some jurisdictions only use one prong of M'Naghten.

\begin{enumerate}
    \item A legal standard has to reflect community values, incorporate 
    scientific understanding, and preserve the factfinder's authority to 
    render a decision.
    \item \textbf{Right-wrong test}: whether a defendant has ``knowledge of 
    good or evil.''\footnote{Casebook p. 619.}
    \item \textbf{M'Naghten rule}: to establish the defense, the defendant     
    must have suffered from a mental disease such that he (1) did ``not know 
    the nature and quality of the act he was doing, or (2) if he did know 
    it, that he did not know what he was doing was wrong.''\footnote{Casebook 
    p. 620.} Criticisms include:
    \begin{enumerate}
        \item It recognizes only cognitive impairments, but not volitional or 
        emotional impairments.
        \item Its ``all-or-nothing approach'' requires total incapacity of 
        cognition.
        \item It severely restricts expert testimony by calling for an ethical 
        judgment.
    \end{enumerate}
    \item \textbf{Irresistable impulse test}: ''courts inquire into both the 
    cognitive and volitional components of the defendant's 
    behavior.''\footnote{Casebook p. 621.} A person is insane if (1) he acted 
    from an irresistible or uncontrollable impulse, (2) he was unable to 
    choose between right and wrong behavior, and (3) his will was destroyed 
    such that his actions were beyond his control. Criticisms:
    \begin{enumerate}
        \item Like M'Naghten, it takes an absolutist view of the capacity to 
        know.
        \item It suggests that a crime must have been committed ``in a sudden 
        and explosive fit.''
    \end{enumerate}
    \item \textbf{\emph{Durham}/Product test}: ``an accused is not criminally 
    responsible if his unlawful act was the product of a mental disease or 
    mental defect.'' The main problem was that expert witnesses usurped the 
    jury's function.  The DC Court of Appeals, which originally introduced the 
    test in 1954, repealed it in 1972.
    \item \textbf{MPC/ALI test}: acknowledges that volitional and cognitive 
    impairments are important. MPC \S\ 4.01 allows the insanity defense when 
    (1) because of a mental disease or defect, ``the defendant lacked 
    substantial capacity to appreciate the criminality [wrongfulness] of his 
    conduct, and (2) ``lacked substantial capacity to conform his conduct to 
    the requirements of law.''\footnote{Casebook p. 622.}
    \item (The M'Naghten rule has been upheld as constitutional because no 
    test has been widely accepted as the baseline, so employing one test 
    instead of others does not pose a due process issue. It's not yet clear 
    whether abolishing the insanity defense is 
    unconstitutional.\footnote{Casebook pp. 624--25.})
\end{enumerate}

\paragraph{Criminality and Wrongfulness: \emph{State v. Wilson}}
~\\\\
If the defendant genuinely believed that society would have viewed his actions 
as criminal but not wrongful, he can claim insanity.

\begin{enumerate}
    \item Wilson killed Jack Peters because of delusions about mind control. 
    There was no question that he was mentally ill. The question was whether 
    he was criminally insane.
    \item The jury rejected his insanity claim and convicted him of murder.  
    \item The question on appeal was the meaning of ``wrongfulness'' under 
    Connecticut General Statutes \S\ 53a--13(a) (modeled on MPC \S\ 4.01).
    \item Wilson argued that wrongfulness has a moral element, so that the 
    accused is not guilty if he believes his act was not morally wrong, even 
    if he believed it was criminal. The trial court refused to give this 
    instruction.
    \item The appellate court addressed two questions: (1) what is the meaning 
    of wrongfulness, and (2) was the defendant's requested instruction 
    necessary in this case?
    \item The court noted three features of the MPC definition:
    \begin{enumerate}
        \item It includes both a cognitive and a volitional component.
        \item It focuses on the defendant's appreciation of, not just 
        knowledge of, the wrongfulness of his conduct. This accounts for cases 
        where the defendant is aware of the wrongfulness but is not affected 
        by it.
        \item It allows legislatures to choose between ``criminality'' and 
        ``wrongfulness.''
    \end{enumerate}
    \item Defendant contended that the insanity defense instructions must 
    define morality in purely personal terms.
    \item The state contended that the defendant must be held to morality 
    according to a social standard, unless he lacked the capacity to 
    appreciate the social moral standard.
    \item Court found that the state's version ``does not sufficiently account 
    for a delusional defendant's own distorted perception of society's moral 
    standards.''\footnote{Casebook p. 634.} If the defendant believed that 
    society would not have condemned his actions under the circumstances as he 
    understood them, he can claim insanity. If he did not believe that society 
    would accept his actions, however, the defense is not available.
    \item The defendant \emph{can} believe that his acts are criminal without 
    believing that they are wrongful.
    \item The court held that the defendant presented sufficient evidence for 
    a jury to have found insanity.
    \item Justice Katz, concurring: the test as the majority interpreted it 
    might exclude defendants who adhere to a personal code of morality because 
    of their mental illness.
    \item Justice McDonald, dissenting: we should not excuse people who kill 
    even though they know society does not condone the killing. The common 
    sense of juries will hopefully mitigate the impact of the majority's 
    ruling. The defendant's conduct showed that he could have been deterred.
\end{enumerate}

\subsubsection{Diminished Capacity}

\begin{enumerate}
    \item There are two commmn law variants of the diminished capacity defense: the 
    \textbf{mens rea} and \textbf{partial responsibility} variants.
    \begin{enumerate}
        \item The \textbf{``mens rea'' variant}:
        \begin{enumerate}
            \item If the prosecution cannot prove the mens rea, the defendant must 
            be acquitted. It is a failure of proof defense.
            \item Claiming no mens rea because of a mental disorder is not the 
            same as claiming legal insanity. The defendant is not claiming partial 
            responsibility, but ``straightforwardly denying the prosecution's 
            prima facie case.''\footnote{Casebook p. 657.}
        \end{enumerate}
        \item The \textbf{``partial responsibility'' variant}:
        \begin{enumerate}
            \item Partial responsibility \emph{is} a form of lesser legal 
            insanity. The defendant claims less culpability and argues he should 
            be convicted of a lesser crime or punished less severely. It is a 
            mitigating excuse defense.
        \end{enumerate}
        \item Limitations on the common law approach: (1) most allow the mens rea 
        variant, but only for specific intent crimes, and (2) a minority allow 
        the partial responsibility variant, but only to mitigate murder to 
        manslaughter.
    \end{enumerate}
    \item The MPC contains two diminished capacity tests:
    \begin{enumerate}
        \item \textbf{Mens rea}, \S\ 4.02(1): ``Evidence that the defendant 
        suffered from a mental disease or defect is admissible whenever it is 
        relevant to prove that the defendant did or did not have a state of 
        mind which is an element of the offense.''
        \begin{enumerate}
            \item Can be used to negate the intent for \emph{any} crime---not just 
            specific intent crimes.
        \end{enumerate}
        \item \textbf{EMED in manslaughter} (a variant of the common law 
        partial responsibility test), \S\ 210.3(1)(b): ``Homicide constitutes 
        manslaughter when: (b) a homicide that would otherwise be murder is 
        committed under the influence of extreme mental or emotional 
        disturbance for which there is a reasonable explanation or excuse.''
        \item Both allow capital cases to be mitigated to imprisonment.
    \end{enumerate}
    \item The California Supreme Court held in several cases that mental 
    illness could negate premeditation, deliberation, and sometimes even 
    malice aforethought. The legislature abolished the defense after the 
    Twinkie Defense affair, so \textbf{diminished capacity is no longer a 
    defense in California}. Other jurisdictions also do not recognize it.
    \item A rationale for the diminished capacity defense is the 
    \textbf{``continuum of competence,''} which allows a defense where the 
    full insanity defense is not available---e.g., the \textbf{Twinkie 
    Defense}.
    \item States that have adopted the MPC's EMED standard have mostly also 
    adopted the partial responsibility model of diminished 
    capacity.\footnote{Casebook p. 659.}
\end{enumerate}

\paragraph{\emph{Clark v. Arizona}}
~\\\\
Mental-disease evidence can only be introduced as part of an insanity defense, 
not a diminished capacity defense---though states may adopt the opposite rule 
without running afoul of due process.

\begin{enumerate}
    \item Clark was circling the block in his truck. Officer Moritz responded 
    and Clark shot him. Clark suffered from schizophrenia and had talked about 
    wanting to kill a police officer.
    \item Clark claimed mental illness in an attempt to rebut evidence of the 
    mens rea for intentionally or knowingly killing a police officer. The 
    trial court did not allow the defense, holding under the \emph{Mott} rule 
    that only insanity could negate the mens rea and convicting Clark of 
    first-degree murder.
    \item The Supreme Court identified three types of evidence: 
    ``observation'' evidence (e.g., witness testimony), ``mental-disease'' 
    evidence, and ``capacity'' evidence (indicating the defendant's capacity 
    for cognition and moral judgment). \emph{Mott} restricted the latter two 
    types.
    \item Clark argued that the \emph{Mott} restrictions violated due process.
    \item The Court here, Justice Souter, held that mental-disease and 
    capacity evidence can be restricted to their bearing on the insanity 
    defense. It identified several dangers of relying on capacity evidence:
    \begin{enumerate}
        \item It might mask internal debates in professional psychology about 
        the mental disease.
        \item It could mislead jurors to believe that the defendant lacked 
        certain capacities when in fact he did not.
        \item Expert witnesses might inappropriately substitute their own 
        opinions on capacity for mental-disease diagnoses.
    \end{enumerate}
    \item Other states are free to adopt the opposite rule.
    \item Justice Kennedy, dissenting:
    \begin{enumerate}
        \item Clark's evidence of lack of capacity was ``critical and 
        reliable.''\footnote{Casebook p. 667.}
        \item ``Simply put, knowledge relies on cognition, and cognition can 
        be affected by schizophrenia.''
        \item The issue is not whether mental illness is an excuse, but 
        whether it ``made him unaware that he was shooting a police 
        officer.''\footnote{Casebook p. 668.}
        \item The risk of speculative testimony from expert witnesses does not 
        explain why evidence of mental illness can \emph{never} be used.
        \item The risk of jury confusion is unconvincing because juries 
        constantly deal with complex issues.
        \item The Court confuses the insanity defense with the question of 
        intent.
        \item The fact that state and defense experts drew different 
        conclusions about Clark's mental illness made the evidence contested 
        but not misleading.
    \end{enumerate}
\end{enumerate}

\subsubsection{Infancy}

\begin{enumerate}
    \item Infancy is a capacity defense.
    \item Common law: there was a conclusive presumption of incapacity for 
    children under the age of seven. There was a rebuttable presumption of 
    capacity between seven and fourteen. See \emph{Devon T} below.
    \item The MPC transfers infancy cases to juvenile courts.\footnote{MPC \S\ 
    4.10.}
\end{enumerate}

\paragraph{\emph{In re Devon T}}
~\\\\
There is a presumption of incapacity for young children that diminishes as 
they get older.

\begin{enumerate}
    \item The incapacity to distinguish right from wrong in the M'Naghten test 
    is a characteristic of many defenses in addition to insanity---infancy, 
    mental illness and retardation, and involuntary intoxication.
    \item \emph{Doli capax}: capable of criminal intent. \emph{Doli incapax}: 
    incapable.\footnote{Casebook p. 673.}
    \item The juvenile appellant---13 years, 10 months, and 2 weeks old---was 
    charged with possession of heroin with intent to distribute. The city 
    circuit court found him delinquent.
    \item The common law rule was that children below seven did not have the 
    capacity to form criminal intent, children above fourteen had full 
    capacity, and those in between fell on a spectrum in which the presumption 
    of criminal capacity is rebuttable.
    \item Early juvenile courts did not allow the infancy defense because the 
    courts were purely rehabilitative and did not take moral blameworthiness 
    into account. The recent shift towards punishment required juvenile courts 
    to take blameworthiness into account. The infancy defense is therefore now 
    available.
    \item The state was required to prove the capacity to form criminal 
    intent. According to the common law rule, the defendant had a 98.2\% 
    capacity to form criminal intent. Moreover, he showed clear awareness that 
    what he was doing was wrong by trying to conceal it.
    \item The court held that the state successfully overcame ``the slight 
    residual weight of the presumption of incapacity due to 
    infancy.''\footnote{Casebook p. 678.}
\end{enumerate}
