\section{Defenses}

\subsection{Justification}

\subsubsection{Self Defense}

\paragraph{\emph{United States v. Peterson}}

\begin{enumerate}
    \item Victim was stealing the windshield wipers from the defendant's car. Defendant got a pistol and said, ``If you move, I will shoot.'' Victim picked up a wrench and walked towared the defendant with the wrench raised. Defendant told him to ``not take another step.'' Victim kept walking and defendant shot him in the face, killing him instantly.
    \item Trial court: indicted for second-degree murder, convicted by jury of manslaughter.
    \item Defendant argues that two jury instructions were erroneous: (1) whether he was the aggressor in the altercation before the homicide, and (2) whether he could have retrated without jeopardizing his safety.
    \item Defendant argued that the killing was excusable.
    \item At common law, there would be no excuse defense. % TODO: add common law criteria and compare to MPC. pp. 500-501 and MPC 3.04
    \item On the first point, the court found that ``the evidence plainly presented an issue of fact as to whether Peterson's conduct was an invitation to and provocation o the encounter which ended in the fatal shot.''\footnote{p. 504.}
    \item On the second point, the court found that the instruction on retreat was proper. The defendant countered that the ``castle'' doctrine does not require a defendant to retreat from his own house. The court rejects this argument on the grounds that the doctrine ``can be invoked only by one who is without fault in bringing the conflict on.''\footnote{Casebook p. 505.}
    \item Affirmed.
\end{enumerate}

\paragraph{Reasonable Belief: \emph{People v. Goetz}}

\begin{enumerate}
    \item 
    \item Trial court: indictments for attempted murder, assault, and weapons possession dismissed on erroneous jury instructions on defense of justification. It found that the prosecutor erred in instructing jurors to consider how a reasonable man would have acted under the circumstances. The statutory test for justification should be ``wholly subjective.''\footnote{Casebook p. 513.}
    \item Appellate court: affirmed dismissal.\footnote{Casebook p. 513.}
    \item The prosecutor, the trial court, and a plurality of the appellate court read the New York justification statute to mean that the defendant's belief must have been ``reasonable to him.'' This is consistent with the MPC, on which the NY criminal statutory reforms of 1961 were based.
    \item However, the court here argued that NY does not follow the MPC here. The legislature intended to keep the objective reasonableness requirement.
    \item Court of Appeals here: reversed; indictments reinstated.
\end{enumerate}

% \paragraph{\emph{State v. Wanrow}}
% 
% \begin{enumerate}
%     \item todo
% \end{enumerate}
% 
% \paragraph{\emph{State v. Norman}}
% 
% \begin{enumerate}
%      \item todo
% \end{enumerate}

\subsubsection{Necessity}

\paragraph{MPC 2.02 and ALI Commentary}

\begin{itemize}
    \item (1) Conduct to avoid harm is justified if:
    \begin{itemize}
        \item (a) Harm avoided is greater.
        \item (b) There are no explicit exceptions for this particular situation.
        \item (c) There is no legislative purpose to exclude the claimed justification.
    \end{itemize}
    \item (2) If the actor recklessly or negligently caused the situation, the justification offense cannot be used to defend against an offense for which recklessness or negligence estasblishes culpability.
\end{itemize}

\paragraph{\em{Nelson v. State}}

\begin{enumerate}
    \item Nelson's truck became bogged down. He and two companions took a dump truck from the Highway Department Yard to help pull out the truck, and it also got stuck. A man named Curly helped them take a front-end loader from the yard, which also got stuck.
    \item The trial court and appellate court affirmed convictions of reckless destruction of personal property and joyriding.
    \item Nelson argued that the jury instruction on the necessity defense should have been based on a subjective interpretation of the emergency situation. The court here agreed, but held that there was not a reasonable apprehension of emergency (since the truck remained stable for 12 hours and one of the actors slept in it). Moreover, there were lawful alternatives. Affirmed.
\end{enumerate}

\paragraph{\em{United States v. Schoon}}

\begin{enumerate}
    \item The three defendants disrupted an IRS office out of protest of US policies in El Salvador. The district court denied them the necessity defense.
    \item There are four elements of the necessity defense:
    \begin{enumerate}
        \item Actors chose the lesser evil.
        \item They acted to prevent imminent harm.
        \item They ``reasonably anticipated a direct causal relationship between their conduct and the harm to be averted.''\footnote{Casebook p. 567.}
        \item They had no legal alternatives.
    \end{enumerate}
    \item The necessity defense is not available if any of the four elements are lacking.
    \item The court distinguished between indirect and direct civil disobedience. This case involved indirect civil disobedience.
    \item Necessity is a utilitarian defense because it aims to maximize social welfare.
    \item The court reasoned:
    \begin{enumerate}
        \item Balance of harms: law or policy ``cannot constitute a legally cognizable harm.''
        \item Causal relationshiop: indirect civil disobedience is ``unlikely to abate the evil.''
        \item Legal alternatives: ``legal alternatives will never be deemed exhausted when the harm can be mitigated by congressional action.''\footnote{Casebook p. 570.}
    \end{enumerate}
    \item Affirmed.
    \item Judge Fernandez, concurring: ``I am not so sure that this defense of justification should be grounded on utilitarian theory alone.''
\end{enumerate}

\paragraph{\em{The Queen v. Dudley and Stephens}}

\begin{enumerate}
    \item Facts: see p. 6.
    \item Was the killing of Richard Parker necessary and justified, or was it murder?
    \item Bracton and Hale: the only justified killing is in self-defense.
    \item \emph{United States v. Holmes}: If killing some would save others, and there is no imminent danger, those to die must be chosen by lot.
    \item Bacon: if you're on a lifeboat, and another passenger would sink it, you can justifiably push him away.
    \item The court here found murder. ``...a man has no right to declare temptation to be an excuse, though he might himself have yielded to it...''\footnote{Casebook p. 575.}
\end{enumerate}

\subsection{Excuse}

\begin{enumerate}
    \item Theories of excuse:
    \begin{enumerate}
        \item \textbf{Utilitarian}: Bentham:``identify situations in which conduct is nondeterrable, so that punishment would be so much unnecessary evil.'' Hart: excuses maximize the effect of a person's choices within the framework of coercive law.
        \item \textbf{Causation}: ``a person should not be blamed for her conduct if it was caused by factors outside her control.''\footnote{Casebook p. 581.}
        \item \textbf{Character}: punishment should be proportional to a wrongdoer's moral desert, and moral desert should be measured by the actor's overall character.
        \item \textbf{Free Choice}: people should only be punished for actions they freely committed. An actor is free if he has the capacity and opportunity to (1) understand the facts, (2) appreciate that the conduct violates society's mores, and (3) conform her conduct to the dictates of the law.\footnote{Casebook p. 582.}
    \end{enumerate}
\end{enumerate}

\subsubsection{Duress}

\paragraph{\emph{United States v. Contento-Pachon}}

\begin{enumerate}
    \item Defendant was coerced into smuggling cocaine into the U.S.
    \item Trial court excluded evidence of duress.
    \item Appellate court here held that there was sufficient evidence of duress to send the factual questions to a jury.
    \item Three elements of duress: (1) immediate threat of death or serious bodily injury, (2) well-grounded fear that the threat will be carried out, and (3) no reasonable opportunity to escape.
    \item Necessity defense was not available because the situation was not caused by physical forces and he did not act to promote the general welfare.
    \item Coyle, dissenting and concurring: defendant failed to establish the immediacy and inescapability needed for the excuse defense.
\end{enumerate}

\paragraph{\emph{People v. Anderson}}

\begin{enumerate}
    \item Defendant, a father, was coerced by another father to murder a woman suspected of molesting their children.
    \item Jury convicted defendant of first-degree murder and kidnapping.
    \item Defendant argued that he killed under duress.
    \item California statute prevents the duress defense for any crime ``punishable by death.'' When the statute was passed, all forms of murder were punishable by death. Now, only first degree murder is punishable by death. The court here chose to retain the original scope. Holding: \textbf{duress is never a defense to murder}.
    \item Defendant also argued that duress negated the malice necessary to murder, and therefore the charge should be reduced to manslaughter. The court held that this would require a new form of voluntary manslaughter, and that only the legislature can make this designation.
    \item The dissent argued that the statutory language should apply to crimes contemporarily punishable by death, not at the time of the drafting of the original statute.
\end{enumerate}

\subsubsection{Intoxication}

\paragraph{\emph{United States v. Veach}}

\begin{enumerate}
    \item Defendant was drunk and got involved in a car accident. He made death threats against the park rangers who arrested him.
    \item Jury convicted him of one count of resisting a federal law enforcement officer and two counts of threatening to assault and murder a law enforcement officer.
    \item Court here: intoxication is a defense when it negates the mens rea of a crime.
    \item Resisting an officer is a general intent crime, so intoxication is not a defense.
    \item Threatening an officer, however, is a specific intent crime, so the defendant should have been allowed to present the defense to a jury.
\end{enumerate}

\subsubsection{Insanity}

\paragraph{\emph{United States v. Freeman}}

\begin{enumerate}
    \item ``...none of the three asserted purposes of criminal law---rehabilitation, deterrence, and retribution---is satisfied when the truly irresponsible are punished.''\footnote{Casebook p. 616.}
    \item The insanity defense aims to ``draw a line between the bad and the mad.''\footnote{Casebook p. 617.}
\end{enumerate}

\paragraph{\emph{State v. Johnson}}

\begin{enumerate}
    \item A legal standard has to reflect community values, incorporate scientific understanding, and preserve the factfinder's authority to render a decision.
    \item \textbf{Right-wrong test}: whether a defendant has ``knowledge of good or evil.''\footnote{Casebook p. 619.}
    \item \textbf{M'Naghten rule}: to establish the defense, the defendant must have suffered from a mental disease such that he did ``not know the nature and quality of the act he was doing, or if he did know it, that he did not know what he was doing was wrong.''\footnote{Casebook p. 620.} Criticisms include:
    \begin{enumerate}
        \item It recognizes only cognitive impairments, but not volitional or emotional impairments.
        \item Its ``all-or-nothing approach'' requires total incapacity of cognition.
        \item It severely restricts expert testimony by calling for an ethical judgment.
    \end{enumerate}
    \item \textbf{Irresistable impulse test}: ''courts inquire into both the cognitive and volitional components of the defendant's behavior.''\footnote{Casebook p. 621.} Criticisms:
    \begin{enumerate}
        \item Like M'Naghten, it takes an absolutist view of the capacity to know.
        \item It suggests that a crime must have been committed ``in a sudden and explosive fit.''
    \end{enumerate}
    \item \textbf{Product test}: ``an accused is not criminally responsible if his unlawful act was the product of a mental disease or mental defect.'' RThe main problem was that expert witnesses usurped the jury's function. The DC Court of Appeals, which originally introduced the test in 1954, repealed it in 1972.
    \item \textbf{MPC}: acknowledges that volitional and cognitive impairments are important. It also uses common vocabulary.
    \item (The M'Naghten rule has been upheld as constitutional because no test has been widely accepted as the baseline, so employing one test instead of others does not pose a due process issue. It's not yet clear whether abolishing the insanity defense is unconstitutional.\footnote{Casebook pp. 624--25.})
\end{enumerate}

\paragraph{ALI Commentary to § 4.01}

\begin{enumerate}
    \item MPC allows the defense when (1) because of a mental disease or defect, ``the defendant lacked substantial capacity to appreciate the criminality [wrongfulness] of his conduct, and (2) ``lacked substantial capacity to conform his conduct to the requirements of law.''\footnote{Casebook p. 622.}
\end{enumerate}

\paragraph{\emph{State v. Wilson}}

\begin{enumerate}
    \item Wilson killed Jack Peters because of delusions about mind control (details on p. 632).
    \item Jury rejected the insanity claim and convicted him of murder. 
    \item Question on appeal: the meaning of ``wrongfulness'' under Connecticut General Statutes § 53a--13(a) (modeled on MPC § 4.01).
    \item Wilson argued that wrongfulness has a moral element, so that the accused is not guilty if he believes his act was not morally wrong, even if he believed it was criminal. The trial court refused to give this instruction.
    \item The appellate court addressed two questions: (1) what is the meaning of wrongfulness, and (2) was the defendant's requested instruction necessary in this case?
    \item The court noted three features of the MPC definition:
    \begin{enumerate}
        \item It includes both a cognitive and a volitional component.
        \item It focuses on the defendant's appreciation of, not just knowledge of, the wrongfulness of his conduct. This accounts for cases where the defendant is aware of the wrongfulness but is not affected by it.
        \item It allows legislatures to choose between ``criminality'' and ``wrongfulness.''
    \end{enumerate}
    \item Defendant contended that the insanity defense instructions must define morality in purely personal terms.
    \item The state contended that the defendant must be held to morality according to a social standard, unless he lacked the capacity to appreciate the social moral standard.
    \item Court found that the state's version ``does not sufficiently account for a delusional defendant's own distorted perception of society's moral standards.''\footnote{Casebook p. 634.} If the defendant believed that society would not have condemned his actions under the circumstances as he understood them, he can claim insanity. If he did not believe that society would accept his actions, however, the defense is not available.
    \item The defendant \emph{can} believe that his acts are criminal without believing that they are wrongful.
    \item The court held that the defendant presented sufficient evidence for a jury to have found insanity.
    \item Justice Katz, concurring: the test as the majority interpreted it might exclude defendants who adhere to a personal code of morality because of their mental illness.
    \item Justice McDonald, dissenting: we should not excuse people who kill even though they know society does not condone the killing. The common sense of juries will hopefully mitigate the impact of the majority's ruling.
\end{enumerate}
