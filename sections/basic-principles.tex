\section{Basic Principles of Criminal Law}

\subsection{Introduction}

\begin{enumerate}
    \item Henry Hart argues that criminal law is a method with five features:
    \begin{enumerate}
        \item It operates by a series of commands (``don't kill or steal'').
        \item A community makes the commands binding.
        \item There are sanctions for disobeying the commands.
        \item The distinction between civil and criminal sanctions is that criminal violations draw a community's moral condemnation.
        \item Violations are punished.
    \end{enumerate}
    \item Murray: laws are framed as conditions (``if you do x, then y''---e.g., punishment), emphasizing agency and choice.
    \item \emph{nulla poena sine lege}: no punishment without law authorizing it.

    \item Sources of criminal law:
    \begin{enumerate}
        \item Codification (statutes, administrative rules, etc.).
        \item Common law (based on the English system, as distinct from a civil-law system).
        \item Case law.
        \item Model Penal Code.
    \end{enumerate}
    \item What distinguishes criminal punishment?
    \begin{enumerate}
        \item Criminal penalties can restrain personal liberty (but civil penalties don't).
        \item Moral stigma.
        \item Judgment is collective---it isn't about two parties.\footnote{See Schelling, ``Ethics, Law, and the Exercise of Self-Command.''}
    \end{enumerate}
    \item \textbf{Probable cause} is necessary to make an arrest.
    \item \textbf{Indictment} by a grand jury is usually necessary before a case can go to trial.
    \item Sixth Amendment guarantees a right to a speedy and public trial, by an impartial jury.
    \item \textbf{Due process clauses} in the Fifth and Fourteenth amendments guarantee persuasion \textbf{beyond a reasonable doubt} (as determined by the factfinder).
    \item What does it mean to prove something ``beyond a reasonable doubt?'' \emph{Owens v. State}: Driver was found drunk and asleep behind the wheel of a running car in a private driveway. Circumstantial evidence gives equal weight to two interpretations of the facts: either he had just arrived (guilty) or had not yet left (not guilty). If each interpretation is equally likely, the factfinder could not fairly choose the guilty option beyond a reasonable doubt. But after analyzing the evidence, the court finds ``the totality of the circumstances are, in the last analysis, inconsistent with a reasonable hypothesis of innocence.'' The court affirms the conviction of driving while intoxicated.
    \item Can you satisfy the burden of proof with only circumstantial evidence?
    \item What is required to meet the reasonable doubt standard?
    \item How should a judge instruct a jury on the definition of ``reasonable doubt''?
\end{enumerate}

\subsection{Principles of Punishment}

\begin{enumerate}
    \item Some types of punishment: prison, fines, community service, shaming.
    \item Two key questions:
    \begin{enumerate}
        \item Who should be punished?
        \item How much punishment is appropriate?
    \end{enumerate}
    \item Two predominant (and non-mutuallly-exclusive) theories of punishment: \textbf{retributivism} and \textbf{utilitarianism}
\end{enumerate}

\subsubsection{Utilitarianism}

Punishment is justified because it's useful.

\begin{enumerate}
    \item Jeremy Bentham: the \textbf{principle of utility} evaluates actions in light of their effect on the happiness of the interested party. Laws aim to augment a community's total happiness.
    \item Kent Greenawalt: ``Since punishment involves pain, it can be justified only if it accomplishes enough good consequences to outweigh this harm.''\footnote{Casebook p. 35.} The consequences of an action determine its morality.
    \item Benefits of utilitarian punishment:
    \begin{enumerate}
        \item General deterrence (i.e., discourage an action from occuring in a community).
        \item Specific deterrence (i.e., discourage a specific person from doing something).
        \item Incapacitation.
        \item Reform.
    \end{enumerate}
\end{enumerate}

\subsubsection{Retributivism}

Punishment is justified because criminals deserve it.

\begin{enumerate}
    \item Michael Moore: ``the desert of an offender is a sufficient reason to punish him or her.''\footnote{Casebook p. 39}
    \item Immanuel Kant: penal law is a categorical imperative.
    \item James Fitzjames Stephen: \textbf{assaultive retribution} holds that hatred and vengeance in the name of morality are socially beneficial; criminals are ``noxious insects.''\footnote{Casebook p. 42.}
    \item Herbert Morris: \textbf{protective retribution} holds that rules exist to provide collective benefit; they guards against unfair advantage for freeriders; if somebody cheats, punishment evens the score.
    \item Jeffrey G. Murphy \& Jean Hampton: wrongdoers implictly place their own value above their victims'; ``retributive punishment is the defeat of the wrongdoer at the hands of the victim.''\footnote{Casebook p. 46.}
\end{enumerate}

\subsubsection{Justifying Punishment}

\paragraph{\emph{The Queen v. Dudley and Stephens}}

Dudley, Stephens, Brooks, and Parker were castaways on a boat 1600 miles from the Cape of Good Hope. They quickly ran out of food and water. After twenty days, Dudley and Stephens decided to kill and eat Parker (with Brooks dissenting). They ate Parker’s body for four days, at which point they were rescued and brought to trial for murder.

The case highlights the differences between retributive and utilitarian theories of justice. Parker was weak and unlikely to have survived the last four days if he hadn’t been killed. Dudley and Stephens likely wouldn’t have survived, either. Moreover, Dudley and Stephens had family responsibilities, while Parker was a drifter. A retributive response would hold that Dudley and Stephens are morally culpable and should be found guilty regardless of the mitigating factors. A utilitarian response would find them not guilty on the recognition of a net benefit for all parties involved.

\paragraph{\emph{People v. Du}}

The defendant, Soon Ja Du, a 51-year-old woman, owned a liquor store in LA. A 15-year-old girl, Latasha Harlins, in the store put a bottle of orange juice in her backpack. It's not clear whether she intended to pay. A fight ensued, in which Du was injured. As Harlins was leaving, Du pulled out a gun (which had been previously stolen, heavily modified, and then recovered) and shot Harlins in the back of the head. She testified that she did not intend to kill Harlins. The jury rejected this defense, convicting her of voluntary manslaughter.

Du's probation officer concluded ``would be most unlikely to repeat this or any other crime.'' The sentencing court sentenced Du to ten years, but suspended the sentence and placed her o n probation. She wrote, ``it is my opinion that justice is never served when public opinion, prejudice, revenge or unwarranted sympathy are considered by a sentencing court in resolving a case.'' She tests Du's case against seven goals of sentencing:

\begin{enumerate}
    \item Protect society.
    \item Punish the defendant.
    \item Encourage the defendant to lead a law-abiding life.
    \item Deter others.
    \item Incapacitation.
    \item Secure restitution for the victim.
    \item Seek uniformity in sentencing.
\end{enumerate}

None of these reasons is sufficient to justify prison time. The only somewhat convincing motivation for prison time is the strong presumption against probation when guns are involved. But this is an unusual case, she concludes, ``which overcomes the statutory presumption against probation.''

\subsection{Proportionality of Punishment}
