\section{Accomplice Liability}

\begin{enumerate}
    \item An accomplice is liable for the principal's crimes if the accomplice 
    intentionally assists the principal in committing the criminal conduct.
    \item \textbf{Accomplice liability is not a substantive offense} on its 
    own. Rather, accomplice liability is a theory of \textbf{derivative 
    liability}.
    \item Accomplices are prosecuted for the principal's crime.
    \item MPC accomplice liability is in \S\ 2.06.
    \item Can knowledge of a crime establish intent to aid its commission?
    \begin{enumerate}
        \item The majority view is \textbf{no}. Complicity requires 
        participation. There must be a ``community of purpose.''
        \item The MPC also requires purpose to encourage the substantive 
        offense. Knowledge is insufficient.
        \item A person is not an accomplice unless he shares the criminal 
        intent of the principal.
    \end{enumerate}
    \item \textbf{Conduct}: Accomplice liability requires that the person 
    intended to promote the principal's \emph{conduct}, but not necessarily 
    the end \emph{result}. See \emph{Riley} below.
    \item Under both the common law majority view and the MPC, you can be 
    liable as an accomplice for the consequences of unintentional acts, so long 
    as the act of assistance was \textbf{undertaken with the same level of 
    culpability as required for the underlying offense}. I.e., an accomplice 
    can be liable if she acts with the \textbf{same mental state as the 
    principal}---e.g., if a taxi passenger negligently encourages the driver 
    to drive negligently and he causes an accident, she is liable.  But if he 
    drives recklessly, she is not liable.
    \item Under old common law rules, accomplice liability was entirely tied 
    to the success of the principal. The accomplice could only be convicted if 
    the principal was convicted. Now, the defendant can still be convicted as 
    an accomplice, as long as \emph{a} principal committed the crime.
    \item \textbf{Justification and excuse}: If the principal is acquitted on 
    an \emph{excuse} defense, the accomplice can still be liable. If the 
    principal is acquitted on a \emph{justification}, there has been no crime 
    and there is no basis for accomplice liability. See \emph{Lopez} below.
    \item \textbf{Legislative exemption}: ``Where the Legislature has dealt 
    with crimes which necessarily involve the joint action of two or more 
    persons, and where no punishment at all is provided for the conduct, or 
    misconduct, of one of the participants, the party whose participation is 
    not denounced by statute cannot be charged with criminal conduct on either 
    a conspiracy or aiding and abetting theory.'' I.e., a victim cannot aid 
    the commission of the crime that victimizes him. MPC \S\ 2.06(6)(a) is in 
    accord. See \emph{Megan R.} below.
    \item \textbf{Abandonment}:
    \begin{enumerate}
        \item Common law: an accomplice can abandon the criminal enterprise, 
        but he ``must communicate his withdrawal to the principal and make bona 
        fide efforts to neutralize the effect of his prior 
        assistance.''\footnote{\emph{Understanding Criminal Law} p. 484.}
        \item MPC: a person is not an accomplice if he ``(1) terminates his 
        assistsance, (2) gives timely warning to the police of the impending 
        offense, or (3) in some other manner attempts to prevent the 
        commission of the crime.''\footnote{\emph{Understanding Criminal Law} 
        p. 494; MPC \S\ 2.06(6).}
    \end{enumerate}
\end{enumerate}

\subsection{General Principles}

\subsubsection{Common Law Terminology: \emph{State v. Ward}}

\begin{enumerate}
    \item \textbf{Principal in the first degree}: one who actually commits a 
    crime.
    \item \textbf{Principal in the second degree}: one who ``aided, counseled, 
    commanded, or encouraged the commission [of the act] in his 
    presence...''---e.g., lookout, getaway driver.\footnote{Casebook p. 849}.
    \item \textbf{Accessory before the fact}: same as principal in the second 
    degree, but not physically present---e.g., casing the bank, buying 
    disguises.
    \item \textbf{Accessory after the fact}: one who renders assistance in 
    hindering detection, arrest, trial, or punishment.
    \begin{enumerate}
        \item Accessories after the fact cannot be tried before the principal 
        is tried.
    \end{enumerate}
    \item Modern equivalents:
    \begin{enumerate}
        \item \textbf{Principal} = first-degree principal.
        \item \textbf{Accomplice} = second-degree principal or accessory 
        before the fact. Accessories after the fact are not accomplices under 
        the modern approach because they are less culpable, but they may be 
        subject to liability for misprision. % TODO
    \end{enumerate}
\end{enumerate}

\subsubsection{Theoretical Foundations for Derivative Liability}

\begin{enumerate}
    \item An accomplice (under common law, a principal in the second degree or 
    an accessory before the fact) is not guilty of ``aiding and abetting'' but 
    rather is \textbf{guilty of the substantive crime itself}. The 
    ``elements'' (mens rea and actus reus) of an accomplice's act are 
    convenient shorthand but not rigorously defined.
    \item Themes in accomplice liability:
    \begin{enumerate}
        \item What makes a person an accomplice so as to justify holding that 
        person liable for the completed crime?
        \item Are there cases where it is possible to convict a second party of a 
        more serious offense than the primary party?
        \item When can an accomplice avoid liability despite participation in 
        criminal activity?
    \end{enumerate}
\end{enumerate}

\subsubsection{\emph{People v. Hoselton}}

Aiders and abettors must share the criminal intent of the principal.

\begin{enumerate}
    \item Hoselton was trespassing with several friends on a barge. They broke 
    into a storage unit and stole several tools. Hoselton was unaware of his 
    friends' intent to steal until they opened the door. He went to a car at 
    the other end of the barge. His friends loaded the stolen goods into the 
    car and drove him directly home.
    \item Hoselton was convicted of entering without breaking a vessel with 
    intent to commit larceny.
    \item In a voluntary interview, Hoselton was asked, ``Were you keeping a 
    lookout?'' and he responded, ``You could say that. I just didn't want to 
    go down in there.''
    \item The appellate court held that an aider and abettor (a principal in 
    the second degree) must share the criminal intent of the principal in the 
    first degree. It found that Hoselton's voluntary statement was 
    insufficient to establish criminal intent.
    \item Reversed.
\end{enumerate}

\subsubsection{Kerman, \emph{Orange is the New Black}}

\begin{enumerate}
    \item See above, p. 55.
\end{enumerate}

\subsection{Mens Rea}

\subsubsection{\emph{People v. Lauria}}

\begin{enumerate}
    \item See above, p. 57.
\end{enumerate}

\subsubsection{No Principal, and Conduct vs. Result: \emph{Riley v. State}}

Accomplice liability requires that the person intended to promote the 
principal's \emph{conduct}, but not necessarily the end \emph{result}. This 
accounts for cases where the principal is not identifiable.

\begin{enumerate}
    \item Riley and Portella opened fire on an unsuspecting crowd, wounding 
    two of them. It was not clear who fired the wounding shots.
    \item The jury found Riley guilty as an accomplice.
    \item In \emph{Echols}, the Alaska appellate court held that accomplice 
    liability requires intent to commit the target crime.
    \item Here, the court reversed the \emph{Echols} rule. It noted that the 
    Alaska statute was based closely on MPC \S\ 2.06. MPC \S\ 2.06(4) 
    indicates that an accomplice has the requisite mens rea if he acts with 
    the culpability sufficient for the target offense. Accomplice liability 
    requires that the person intended to promote the principal's 
    \emph{conduct}, but not necessarily the end 
    \emph{result}.\footnote{Casebook p. 861--62.}
    \item Affirmed.
\end{enumerate}

\subsubsection{Natural-and-Probable Consequences Doctrine: \emph{State v. 
Linscott}}

An accomplice is liable for the reasonably foreseeable consequences of the 
criminal undertaking.

\begin{enumerate}
    \item Linscott and several others intended to rob Grenier at his home. One 
    of the other robbers, Fuller, fired a shot which killed Grenier.
    \item At trial, Linscott argued that Fuller often carried a gun with him 
    because he was a hunter. He argued further that he had no intent of 
    causing a death in the course of the robbery.
    \item The trial court found him guilty of robbery and, by accomplice 
    liability, guilty of murder. Although Linscott did not intend to kill 
    Grenier, the murder was a reasonably foreseeable consequence of the 
    robbery.
    \item The appellate court rejected Linscott's argument that the 
    foreseeable consequence rule violated due process. Affirmed.
    \item Juries must follow a four-step process in deciding whether to apply 
    the natural-and-probable-consequences doctrine:\footnote{Casebook p. 866.}
    \begin{enumerate}
        \item Did the primary party commit the target offense (or an inchoate 
        version)?
        \item Was the secondary party an accomplice? 
        \item Did the primary party commit \emph{another} crime beyond the 
        target offense?
        \item Were the additional crimes reasonable and foreseeable 
        consequences of the criminal acts?
        \item If the answer is yes to all of the above, the doctrine applies.
    \end{enumerate}
\end{enumerate}

\subsection{Actus Reus}

\subsubsection{Slackjaw Deadeye: \emph{State v. V.T.}}

Active encouragement is required to establish accomplice liability. A 
comforting or encouraging presence can suffice in some cases (e.g., standing   
lookout or providing moral support).

\begin{enumerate}
    \item V.T., Moose, and Joey were staying at a relative's house. The 
    relative's camcorder went missing and turned up soon after at a pawnshop. 
    The camcorder contained a video of Moose calling a friend and discussed 
    pawning the stolen camcorder. V.T. ``never spoke or gestured during any of 
    this footage.''\footnote{Casebook p. 869.}
    \item V.T. was charged with two counts of theft. The judge found him 
    guilty of misdemeanor theft of the camcorder.
    \item The appellate court held that V.T.'s passive presence alone was insufficient 
    to establish accomplice liability. The state was required to show some 
    kind of active encouragement. Reversed.
\end{enumerate}

\subsubsection{Small Encouragement: \emph{Wilcox v. Jeffery}}

Sometimes very small encouragement suffices, especially when politics are 
involved.

\begin{enumerate}
    \item Wilcox ran \emph{Jazz Illustrated} magazine. Coleman Hawkins entered 
    the country without permission to take employment. Hawkins gave a 
    performance that Wilcox attended and wrote about for his magazine.
    \item The court found him guilty of aiding and abetting. It noted that 
    ``his presence and his payment to go there [to Hawkins's show] was an 
    encouragement.''\footnote{Casebook p. 873.}
\end{enumerate}

\subsection{Liability of Principals and Accomplices}

\subsubsection{Innocent Agency Doctrine: \emph{Bailey v. Commonwealth}}

I'm a [non-culpable] fool to do your dirty work.

\begin{enumerate}
    \item Bailey and Murdock were drunk and taunting each other on citizens' 
    band radio. Bailey convinced Murdock to stand on his porch with his gun. 
    He then anonymously called the police on Murdock. Murdock opened fire. The 
    police fired back, killing him.
    \item The trial court instructed the jury that it should find Bailey 
    guilty of involuntary manslaughter if he acted with callous disregard for 
    human life and that he proximately caused Murdock's death. The jury 
    convicted him.
    \item The appellate court affirmed, holding that ``Bailey undertook to 
    cause Murdock harm and used the police to accomplish that 
    purpose.''\footnote{Casebook p. 882.} Murdock's firing on the police was a 
    reasonably foreseeable intervening cause.
    \item The ``innocent agency doctrine'' holds a defendant liable if he 
    uses an innocent agent to commit the offense.
\end{enumerate}

\subsubsection{Justification: \emph{United States v. Lopez}}

If the target offense was justified then the principal did not commit a 
criminal act and there can be no accomplice liability.

\begin{enumerate}
    \item McIntosh landed a helicopter in a prison to help his girlfriend, 
    Lopez, escape. Before trial, McIntosh and Lopez indicated intent to raise 
    a ``necessity/duress'' defense based on threats to Lopez's life. McIntosh 
    requested a jury instruction that if Lopez acted under necessity/duress, 
    McIntosh could not be guilty of aiding and abetting.\footnote{Casebook p. 
    885.}
    \item The court held that Lopez's necessity defense was a justification, 
    not an excuse. If the jury found Lopez not guilty, the principal would 
    have committed no criminal act, and therefore there cannot be accomplice 
    liability. McIntosh is entitled to his jury instruction.
\end{enumerate}

\subsubsection{More Serious Crime than the Principal's: \emph{People v. McCoy}}

An accomplice can be convicted of a more serious crime than the principal's 
crime because their mental states are distinct.

\begin{enumerate}
    \item McCoy and Lakey were tried for first-degree murder in a drive-by 
    shooting. McCoy shot the victim. He claimed self-defense, which the jury 
    rejected.
    \item The appellate court overturned on the ground that the jury 
    instructions on self-defense were inadequate. It also reversed Lakey's 
    murder conviction on the ground that an accomplice cannot be convicted of 
    a more serious offense than the principal.
    \item The California Supreme Court held that an aider and abettor's mental 
    state is distinct from the principal's mental state and might well be 
    more culpable (e.g., Iago and Othello). Reversed.
    % todo: self-defense or self-defense? edit throughout
\end{enumerate}

\subsection{Limitations on Accomplice Liability}

\subsubsection{Aiding Your Own Victimization? \emph{In re Megan R.}}

A victim of a crime cannot be an accomplice to its commission.

\begin{enumerate}
    \item Oscar Rodriguez and Megan R. broke into the home of Joani Rodriguez 
    to have sex. Megan was 14 years old. The juvenile court convicted her of 
    burglary (breaking and entering into the dwelling of another at night with 
    the intent to commit a felony therein).
    \item On appeal, Megan argued that she could not logically aid the crime 
    of her own statutory rape.\footnote{Cf. \emph{Gebardi}, above, where the 
    court found that a woman could not be found guilty as a co-conspirator to 
    violate the Mann Act.} The court agreed. Reversed.
\end{enumerate}

\subsubsection{Renunciation: \emph{State v. Formella}}

To counter complicity in a crime, an accomplice must take affirmative steps to 
end his involvement. Passive withdrawal is not enough.

\begin{enumerate}
    \item Formella and friends met a group of students in a school hallway 
    after school had dismissed. The group asked them to stand lookout while 
    they stole math exams. While the group was away, Formella decided to 
    abandon the project and went outside to the parking lot. Later, someone 
    informed the school about the theft. Formella was charged and convicted of 
    the theft based on accomplice liability.
    \item Formella argued that he was not an accomplice because he ended his 
    involvement with the crime. The relevant statute required that he would 
    not be held liable as an accomplice if (among other requirements) ``he 
    wholly deprived his complicity of effectiveness in the commission of the 
    crime.''\footnote{Casebook p. 892.} The court held that to counter his 
    prior complicity, Formella must have acted affirmatively. Passive 
    withdrawal is insufficient. Affirmed.
\end{enumerate}
